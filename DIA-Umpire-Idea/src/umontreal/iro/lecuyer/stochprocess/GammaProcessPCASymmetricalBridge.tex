\defmodule {GammaProcessPCASymmetricalBridge}

Same as \class{GammaProcessPCABridge}, but uses the fast inversion method
for the symmetrical beta distribution, proposed by L'Ecuyer and Simard
\cite {rLEC06a}, to accelerate the generation of the beta random variables.
This class works only in the case where the number of intervals is a power of
2 and all these intervals are of equal size.

\bigskip\hrule\bigskip

%%%%%%%%%%%%%%%%%%%%%%%%%%%%%%%%%%%%%%%%%%%%%%%%%%%%%%%%%%%%%%%%%%
\begin{code}
\begin{hide}
/*
 * Class:        GammaProcessPCASymmetricalBridge
 * Description:
 * Environment:  Java
 * Software:     SSJ
 * Copyright (C) 2001  Pierre L'Ecuyer and Universite de Montreal
 * Organization: DIRO, Universite de Montreal
 * @authors      Jean-Sebastien Parent and Maxime Dion
 * @since        july 2008

 * SSJ is free software: you can redistribute it and/or modify it under
 * the terms of the GNU General Public License (GPL) as published by the
 * Free Software Foundation, either version 3 of the License, or
 * any later version.

 * SSJ is distributed in the hope that it will be useful,
 * but WITHOUT ANY WARRANTY; without even the implied warranty of
 * MERCHANTABILITY or FITNESS FOR A PARTICULAR PURPOSE.  See the
 * GNU General Public License for more details.

 * A copy of the GNU General Public License is available at
   <a href="http://www.gnu.org/licenses">GPL licence site</a>.
 */
\end{hide}
package umontreal.iro.lecuyer.stochprocess;\begin{hide}
import umontreal.iro.lecuyer.rng.*;
import umontreal.iro.lecuyer.probdist.*;
import umontreal.iro.lecuyer.randvar.*;

\end{hide}

public class GammaProcessPCASymmetricalBridge extends GammaProcessPCABridge \begin{hide} {
\end{hide}
\end{code}
%%%%%%%%%%%%%%%%%%%%%%%%%%%%%%%%%%%%%%%%%%%%%%%%%%%%%%%%%%%%%%%%
\subsubsection* {Constructors}
\begin{code}

   public GammaProcessPCASymmetricalBridge (double s0, double mu, double nu,
                                            RandomStream stream) \begin{hide} {
        super (s0, mu, nu,  stream);
   }\end{hide}
\end{code}
\begin{tabb}
Constructs a new \texttt{GammaProcessPCASymmetricalBridge}
with parameters $\mu = \texttt{mu}$, $\nu = \texttt{nu}$ and initial
value $S(t_{0}) = \texttt{s0}$.
The \externalclass{umontreal.iro.lecuyer.rng}{RandomStream} \texttt{stream}
is used in the
\externalclass{umontreal.iro.lecuyer.randvar}{GammaGen}
and in the \externalclass{umontreal.iro.lecuyer.randvar}{BetaSymmetricalGen}.
These two generators use inversion to generate random numbers.  The first
uniform random number generated by \texttt{stream} is used for the gamma, and the
other $d-1$ for the beta's.
\end{tabb}

%%%%%%%%%%%%%%%%%%%%%%%%%%%%%%%%%%%%%%
%%\subsubsection* {Methods}
\begin{code}\begin{hide}

   public double[] generatePath (double[] uniform01)  {
    // uniformsV[] of size d+1, but element 0 never used.
        double[] uniformsV = new double[d+1];

    // generate BrownianMotion PCA path
        double[] BMPCApath = BMPCA.generatePath(uniform01);
        int oldIndexL;
        int newIndex;
        int oldIndexR;
        double temp, y;
    // Transform BMPCA points to uniforms using an inverse bridge.
        for (int j = 0; j < 3*(d-1); j+=3) {
            oldIndexL   = BMBridge.wIndexList[j];
            newIndex    = BMBridge.wIndexList[j + 1];
            oldIndexR   = BMBridge.wIndexList[j + 2];

            temp = BMPCApath[newIndex] - BMPCApath[oldIndexL];
            temp -= (BMPCApath[oldIndexR] - BMPCApath[oldIndexL]) *
                                          BMBridge.wMuDt[newIndex];
            temp /= BMBridge.wSqrtDt[newIndex];
            uniformsV[newIndex] = NormalDist.cdf01(temp);
        }
    double dT = BMPCA.t[d] - BMPCA.t[0];
    uniformsV[d] = NormalDist.cdf01( ( BMPCApath[d] - BMPCApath[0] - BMPCA.mu*dT )/
                     ( BMPCA.sigma * Math.sqrt(dT) ) );


    // Reconstruct the bridge for the GammaProcess from the Brownian uniforms.
    // Here we have to hope that the bridge is implemented in the
    // same order for the Brownian and Gamma processes.

        path[0] = x0;
        path[d] = x0 + GammaDist.inverseF(mu2dTOverNu, muOverNu, 10, uniformsV[d]);
        for (int j = 0; j < 3*(d-1); j+=3) {
            oldIndexL   = wIndexList[j];
            newIndex    = wIndexList[j + 1];
            oldIndexR   = wIndexList[j + 2];

            y = BetaSymmetricalDist.inverseF(bMu2dtOverNuL[newIndex], uniformsV[newIndex]);

            path[newIndex] = path[oldIndexL] +
        (path[oldIndexR] - path[oldIndexL]) * y;
        }
        //resetStartProcess();
        observationIndex   = d;
        observationCounter = d;
        return path;
    }


   public double[] generatePath()  {
        double[] u = new double[d];
        for(int i =0; i < d; i++)
            u[i] = stream.nextDouble();
        return generatePath(u);
    }


    // code taken from GammaSymmetricalBridge to check time is power of 2,
    // as it is required for the symmetrical bridge.
   protected void init () {
        super.init ();
        if (observationTimesSet) {

            /* Testing to make sure number of observations n = 2^k */
                int x = d;
            int y = 1;
            while (x>1) {
            x = x / 2;
            y = y * 2;
            }
            if (y != d) throw new IllegalArgumentException
            ( "GammaSymmetricalBridgeProcess:"
                +"Number 'n' of observation times is not a power of 2" );
       }
    }
}
\end{hide}
\end{code}
