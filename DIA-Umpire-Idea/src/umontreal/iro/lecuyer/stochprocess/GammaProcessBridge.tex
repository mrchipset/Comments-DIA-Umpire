\defmodule {GammaProcessBridge}

This class represents a gamma process
$\{ S(t) = G(t; \mu, \nu) : t \geq 0 \}$ with mean parameter $\mu$ and
variance parameter $\nu$, sampled using the \emph{gamma bridge} method
(see for example \cite{fRIB02a,fAVR03a}).
This is analogous to the bridge sampling used in
\class{BrownianMotionBridge}.

Note that gamma bridge sampling requires not only gamma variates, but also
\emph{beta} variates. The latter generally take a longer time to generate
than the former. The class \texttt{GammaSymmetricalBridgeProcess} provides
a faster implementation when the number of observation times
is a power of two.

The warning from class \class{BrownianMotionBridge} applies verbatim
to this class.

\bigskip\hrule\bigskip

%%%%%%%%%%%%%%%%%%%%%%%%%%%%%%%%%%%%%%%%%%%%%%%%%%%%%%%%%%%%%%%%%%
\begin{code}
\begin{hide}
/*
 * Class:        GammaProcessBridge
 * Description:
 * Environment:  Java
 * Software:     SSJ
 * Copyright (C) 2001  Pierre L'Ecuyer and Universite de Montreal
 * Organization: DIRO, Universite de Montreal
 * @author       Pierre Tremblay and Jean-Sebastien Parent
 * @since        July 2003

 * SSJ is free software: you can redistribute it and/or modify it under
 * the terms of the GNU General Public License (GPL) as published by the
 * Free Software Foundation, either version 3 of the License, or
 * any later version.

 * SSJ is distributed in the hope that it will be useful,
 * but WITHOUT ANY WARRANTY; without even the implied warranty of
 * MERCHANTABILITY or FITNESS FOR A PARTICULAR PURPOSE.  See the
 * GNU General Public License for more details.

 * A copy of the GNU General Public License is available at
   <a href="http://www.gnu.org/licenses">GPL licence site</a>.
 */
\end{hide}
package umontreal.iro.lecuyer.stochprocess;\begin{hide}
import umontreal.iro.lecuyer.rng.*;
import umontreal.iro.lecuyer.probdist.*;
import umontreal.iro.lecuyer.randvar.*;
import umontreal.iro.lecuyer.util.Num;
\end{hide}

public class GammaProcessBridge extends GammaProcess \begin{hide} {
    protected BetaGen      Bgen;
    protected double       mu2OverNu,
                           mu2dTOverNu;
    protected double[]     bMu2dtOverNuL,  // For precomputations for G Bridge
                           bMu2dtOverNuR;
    protected int[]        wIndexList;
    protected int          bridgeCounter = -1; // Before 1st observ
\end{hide}
\end{code}
%%%%%%%%%%%%%%%%%%%%%%%%%%%%%%%%%%%%%%%%%%%%%%%%%%%%%%%%%%%%%%%%
\subsubsection* {Constructors}
\begin{code}

   public GammaProcessBridge (double s0, double mu, double nu,
                              RandomStream stream) \begin{hide} {
        this (s0, mu, nu, new GammaGen (stream, new GammaDist (1.0)),
                          new BetaGen (stream, new BetaDist (1.0, 1.0)));
    }\end{hide}
\end{code}
\begin{tabb} Constructs a new \texttt{GammaProcessBridge} with parameters
$\mu = \texttt{mu}$, $\nu = \texttt{nu}$ and initial value $S(t_{0}) = \texttt{s0}$.
Uses \texttt{stream} to generate the gamma and beta variates by inversion.
\end{tabb}
\begin{code}

   public GammaProcessBridge (double s0, double mu, double nu,
                              GammaGen Ggen, BetaGen Bgen) \begin{hide} {
        super (s0, mu, nu, Ggen);
        this.Bgen = Bgen;
        this.Bgen.setStream(Ggen.getStream()); // to avoid confusion in streams
        this.stream = Ggen.getStream();
    }\end{hide}
\end{code}
\begin{tabb} Constructs a new \texttt{GammaProcessBridge}. Uses the
random variate generators \texttt{Ggen} and \texttt{Bgen} to generate the gamma
and beta variates, respectively. Note that both generator uses the
same \externalclass{umontreal.iro.lecuyer.rng}{RandomStream}. Furthermore, the
parameters of the
\externalclass{umontreal.iro.lecuyer.randvar}{GammaGen} and
\externalclass{umontreal.iro.lecuyer.randvar}{BetaGen} objects are not
important since the implementation forces the generators to use
the correct parameters.
(as defined in \cite[page 7]{fRIB02a}).
\end{tabb}
%%%%%%%%%%%%%%%%%%%%%%%%%%%%%%%%%%%%%%
\subsubsection* {Methods}
\begin{code}\begin{hide}

   public double nextObservation()  {
        double s;
        if (bridgeCounter == -1) {
            s = x0 + Ggen.nextDouble(stream, mu2dTOverNu, muOverNu);
            if (s <= x0)
                s = setLarger (x0);
            bridgeCounter    = 0;
            observationIndex = d;
        } else {
            int j = bridgeCounter * 3;
            int oldIndexL = wIndexList[j];
            int newIndex  = wIndexList[j + 1];
            int oldIndexR = wIndexList[j + 2];

            double y =  Bgen.nextDouble(stream, bMu2dtOverNuL[newIndex],
                                         bMu2dtOverNuR[newIndex], 0.0, 1.0);
            s = path[oldIndexL] +
              (path[oldIndexR] - path[oldIndexL]) * y ;
           // make sure the process is strictly increasing
           if (s <= path[oldIndexL])
               s = setLarger (path, oldIndexL, oldIndexR);
            bridgeCounter++;
            observationIndex = newIndex;
        }
        observationCounter = bridgeCounter + 1;
        path[observationIndex] = s;
        return s;
    }

   public double nextObservation (double nextT) {
        double s;
        if (bridgeCounter == -1) {
            t[d] = nextT;
            mu2dTOverNu = mu2OverNu * (t[d] - t[0]);
            s = x0 + Ggen.nextDouble(stream, mu2dTOverNu, muOverNu);
            if (s <= x0)
                s = setLarger (x0);
            bridgeCounter    = 0;
            observationIndex = d;
        } else {
            int j = bridgeCounter * 3;
            int oldIndexL = wIndexList[j];
            int newIndex  = wIndexList[j + 1];
            int oldIndexR = wIndexList[j + 2];

            t[newIndex] = nextT;
            bMu2dtOverNuL[newIndex] = mu2OverNu
                                      * (t[newIndex] - t[oldIndexL]);
            bMu2dtOverNuR[newIndex] = mu2OverNu
                              * (t[oldIndexR] - t[newIndex]);

            double y = Bgen.nextDouble(stream, bMu2dtOverNuL[newIndex],
                                         bMu2dtOverNuR[newIndex], 0.0, 1.0);

            s = path[oldIndexL] +
              (path[oldIndexR] - path[oldIndexL]) * y ;
            // make sure the process is strictly increasing
            if (s <= path[oldIndexL])
               s = setLarger (path, oldIndexL, oldIndexR);
            bridgeCounter++;
            observationIndex = newIndex;
        }
        observationCounter = bridgeCounter + 1;
        path[observationIndex] = s;
        return s;
    }

   public double[] generatePath (double[] uniform01) {
        int oldIndexL, oldIndexR, newIndex;
        double y;

        path[d] = x0 + GammaDist.inverseF (mu2dTOverNu, muOverNu, 10, uniform01[0]);
        for (int j = 0; j < 3*(d-1); j+=3) {
            oldIndexL   = wIndexList[j];
            newIndex    = wIndexList[j + 1];
            oldIndexR   = wIndexList[j + 2];

            y = BetaDist.inverseF(bMu2dtOverNuL[newIndex], bMu2dtOverNuR[newIndex], 8, uniform01[1 + j/3]);

            path[newIndex] = path[oldIndexL] +
              (path[oldIndexR] - path[oldIndexL]) * y;
            // make sure the process is strictly increasing
            if (path[newIndex] <= path[oldIndexL])
               setLarger (path, oldIndexL, newIndex, oldIndexR);
        }
        //resetStartProcess();
        observationIndex   = d;
        observationCounter = d;
        return path;
    }

    public double[] generatePath() {
        int oldIndexL, oldIndexR, newIndex;
        double y;

        path[d] = x0 + Ggen.nextDouble(stream, mu2dTOverNu, muOverNu);
        for (int j = 0; j < 3*(d-1); j+=3) {
            oldIndexL   = wIndexList[j];
            newIndex    = wIndexList[j + 1];
            oldIndexR   = wIndexList[j + 2];

            y = Bgen.nextDouble(stream, bMu2dtOverNuL[newIndex], bMu2dtOverNuR[newIndex], 0.0, 1.0);
            path[newIndex] = path[oldIndexL] +
              (path[oldIndexR] - path[oldIndexL]) * y;
           // make sure the process is strictly increasing
           if (path[newIndex] <= path[oldIndexL])
               setLarger (path, oldIndexL, newIndex, oldIndexR);
        }
        //resetStartProcess();
        observationIndex   = d;
        observationCounter = d;
        return path;
    }

   public void resetStartProcess() {
        observationIndex   = 0;
        observationCounter = 0;
        bridgeCounter = -1;
    }

   protected void init() {
        super.init();
        if (observationTimesSet) {

        // Quantities for gamma bridge process
        bMu2dtOverNuL = new double[d+1];
        bMu2dtOverNuR = new double[d+1];
        wIndexList  = new int[3*d];

        int[] ptIndex = new int[d+1];
        int   indexCounter = 0;
        int   newIndex, oldLeft, oldRight;

        ptIndex[0] = 0;
        ptIndex[1] = d;

        mu2OverNu   = mu * mu / nu;
        mu2dTOverNu = mu2OverNu * (t[d] - t[0]);

        for (int powOfTwo = 1; powOfTwo <= d/2; powOfTwo *= 2) {
            /* Make room in the indexing array "ptIndex" */
            for (int j = powOfTwo; j >= 1; j--) { ptIndex[2*j] = ptIndex[j]; }

            /* Insert new indices and Calculate constants */
            for (int j = 1; j <= powOfTwo; j++) {
                oldLeft  = 2*j - 2;
                oldRight = 2*j;
                newIndex = (int) (0.5*(ptIndex[oldLeft] + ptIndex[oldRight]));

                bMu2dtOverNuL[newIndex] = mu * mu
                                   * (t[newIndex] - t[ptIndex[oldLeft]]) / nu;
                bMu2dtOverNuR[newIndex] = mu * mu
                                  * (t[ptIndex[oldRight]] - t[newIndex]) / nu;

                ptIndex[oldLeft + 1]       = newIndex;
                wIndexList[indexCounter]   = ptIndex[oldLeft];
                wIndexList[indexCounter+1] = newIndex;
                wIndexList[indexCounter+2] = ptIndex[oldRight];

                indexCounter += 3;
            }
        }
        /* Check if there are holes remaining and fill them */
        for (int k = 1; k < d; k++) {
            if (ptIndex[k-1] + 1 < ptIndex[k]) {
            // there is a hole between (k-1) and k.

                bMu2dtOverNuL[ptIndex[k-1]+1] = mu * mu
                                  * (t[ptIndex[k-1]+1] - t[ptIndex[k-1]]) / nu;
                bMu2dtOverNuR[ptIndex[k-1]+1] = mu * mu
                                  * (t[ptIndex[k]] - t[ptIndex[k-1]+1]) / nu;

                wIndexList[indexCounter]   = ptIndex[k]-2;
                wIndexList[indexCounter+1] = ptIndex[k]-1;
                wIndexList[indexCounter+2] = ptIndex[k];
                indexCounter += 3;
            }
        }
        }
    }\end{hide}

   public void setStream (RandomStream stream)\begin{hide} {
        super.setStream(stream);
        this.Bgen.setStream(stream);
        this.stream = stream;
}\end{hide}
\end{code}
\begin{tabb}
Resets the \externalclass{umontreal.iro.lecuyer.rng}{RandomStream}
of the \externalclass{umontreal.iro.lecuyer.randvar}{GammaGen} and
the \externalclass{umontreal.iro.lecuyer.randvar}{BetaGen} to \texttt{stream}.
\end{tabb}
\begin{code}
\begin{hide}
}
\end{hide}
\end{code}
