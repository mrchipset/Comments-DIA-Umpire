\defmodule {JohnsonSBDist}

Extends the class \class{ContinuousDistribution} for
the {\em Johnson $S_B$\/} distribution\latex{ \cite{tJOH49a,sLAW00a,tFLY06a}}
with shape parameters $\gamma$ and $\delta > 0$, location parameter $\xi$,
and scale parameter $\lambda>0$.
Denoting $t=(x-\xi)/\lambda$ and $z = \gamma + \delta\ln (t/(1-t))$, the density is
\eq
 f(x) = \html{\delta  e^{-z^2/2}/(}\latex{\frac {\delta e^{-z^2/2}}}
 {\lambda t(1 - t)\sqrt{2\pi}}
 \html{)}, \qquad
  \mbox { for } \xi < x < \xi+\lambda,     \eqlabel{eq:JohnsonSB-density}
\endeq
and 0 elsewhere.  The distribution function is
\eq
 F(x) = \Phi (z),
     \mbox { for } \xi < x < \xi+\lambda,   \eqlabel{eq:JohnsonSB-dist}
\endeq
where $\Phi$ is the standard normal distribution function.
The inverse distribution function is
\eq
  F^{-1}(u) = \xi + \lambda \left(1/\left(1+e^{-v(u)}\right)\right) \qquad\mbox{for }  0 \le u \le 1,
                                      \eqlabel{eq:JohnsonSB-inverse}
\endeq
where
\eq
  v(u) = [\Phi^{-1}(u) - \gamma]/\delta.
\endeq

This class relies on the methods  \clsexternalmethod{}{NormalDist}{cdf01}{} and
  \clsexternalmethod{}{NormalDist}{inverseF01}{}
of \class{NormalDist} to approximate $\Phi$ and $\Phi^{-1}$.

\bigskip\hrule

%%%%%%%%%%%%%%%%%%%%%%%%%%%%%%%%%%%%%%%%
\begin{code}
\begin{hide}
/*
 * Class:        JohnsonSBDist
 * Description:  Johnson S_B distribution
 * Environment:  Java
 * Software:     SSJ
 * Copyright (C) 2001  Pierre L'Ecuyer and Universite de Montreal
 * Organization: DIRO, Universite de Montreal
 * @author
 * @since

 * SSJ is free software: you can redistribute it and/or modify it under
 * the terms of the GNU General Public License (GPL) as published by the
 * Free Software Foundation, either version 3 of the License, or
 * any later version.

 * SSJ is distributed in the hope that it will be useful,
 * but WITHOUT ANY WARRANTY; without even the implied warranty of
 * MERCHANTABILITY or FITNESS FOR A PARTICULAR PURPOSE.  See the
 * GNU General Public License for more details.

 * A copy of the GNU General Public License is available at
   <a href="http://www.gnu.org/licenses">GPL licence site</a>.
 */
\end{hide}
package umontreal.iro.lecuyer.probdist;\begin{hide}
import umontreal.iro.lecuyer.util.Num;
\end{hide}

public class JohnsonSBDist extends JohnsonSystem\begin{hide} {
   // m_psi is used in computing the mean and the variance
   private double m_psi = -1.0e100;


   private static double getMeanPsi (double gamma, double delta,
                           double xi, double lambda, double[] tpsi) {
      // Returns the theoretical mean of t = (x - xi)/lambda;
      // also compute psi and returns it in tpsi[0], since
      // it is used in computing the mean and the variance

      final int NMAX = 10000;
      final double EPS = 1.0e-15;

      double a1 = 1.0/(2*delta*delta);
      double a2 = (1.0 - 2*delta*gamma)/(2*delta*delta);
      double a3 = (gamma - 1./delta)/delta;
      int n = 0;
      double tem = 1;
      double sum = 0;
      double v;
      while (Math.abs(tem) > EPS* Math.abs(sum) && n < NMAX) {
         ++n;
         v = Math.exp(-n*gamma/delta) + Math.exp(n*a3);
         tem = Math.exp(-n*n*a1) * v / (1 + Math.exp(-2*n*a1));
      //   tem = Math.exp(-n*n*a1) * Math.cosh(n*a2) / Math.cosh(n*a1);
         sum += tem;
      }
      if (n >= NMAX)
         System.err.println ("JohnsonSBDist:  possible lack of accuracy on mean");
      double A = (0.5 + sum) / (delta);

      a1 = Math.PI * Math.PI * delta * delta;
      a2 = Math.PI * delta * gamma;
      int j;
      n = 0;
      tem = 1;
      sum = 0;
      while (Math.abs(tem) > EPS*Math.abs(sum) && n < NMAX) {
         ++n;
         j = 2*n - 1;
         tem = Math.exp(-j*j*a1/2.0) * Math.sin(j*a2) / Math.sinh(j*a1);
         sum += tem;
      }
      if (n >= NMAX)
         System.err.println ("JohnsonSBDist:  possible lack of accuracy on mean");
      double B = 2.0* Math.PI * delta * sum;

      a1 = 2*Math.PI * Math.PI * delta * delta;
      a2 = 2*Math.PI * delta * gamma;
      n = 0;
      tem = 1;
      sum = 0;
      while (Math.abs(tem) > EPS*Math.abs(sum) && n < NMAX) {
         ++n;
         tem = Math.exp(-n*n*a1) * Math.cos(n*a2);
         sum += tem;
      }
      if (n >= NMAX)
         System.err.println ("JohnsonSBDist:  possible lack of accuracy on mean");
      double C = 1 + 2.0 * sum;

      double D = Math.sqrt(2*Math.PI) * Math.exp(gamma* gamma / 2.0);
      double tmean = (A - B) / (C*D);
      tpsi[0] = C*D;
      return tmean;
   }
\end{hide}
\end{code}
%%%%%%%%%%%%%%%%%%%%%%%%%%%%%%%%%%%%%%%%%
\subsubsection* {Constructor}

\begin{code}

   public JohnsonSBDist (double gamma, double delta,
                         double xi, double lambda)\begin{hide} {
      super (gamma, delta, xi, lambda);
      setLastParams(xi, lambda);
   }\end{hide}
\end{code}
  \begin{tabb} Constructs a \texttt{JohnsonSBDist} object
   with shape parameters $\gamma$ and $\delta$,
   location parameter $\xi$ and scale parameter $\lambda$.
  \end{tabb}

%%%%%%%%%%%%%%%%%%%%%%%%%%%%%%%%%%%
\subsubsection* {Methods}
\begin{hide}
\begin{code}

   private void setLastParams(double xi, double lambda) {
      supportA = xi;
      supportB = xi + lambda;
   }

   public double density (double x) {
      return density (gamma, delta, xi, lambda, x);
   }

   public double cdf (double x) {
      return cdf (gamma, delta, xi, lambda, x);
   }

   public double barF (double x) {
      return barF (gamma, delta, xi, lambda, x);
   }

   public double inverseF (double u) {
      return inverseF (gamma, delta, xi, lambda, u);
   }

   public double getMean() {
      return JohnsonSBDist.getMean (gamma, delta, xi, lambda);
   }

   public double getVariance() {
      return JohnsonSBDist.getVariance (gamma, delta, xi, lambda);
   }

   public double getStandardDeviation() {
      return JohnsonSBDist.getStandardDeviation (gamma, delta, xi, lambda);
   }
\end{code}
\end{hide}\begin{code}

   public static double density (double gamma, double delta,
                                 double xi, double lambda, double x)\begin{hide} {
      if (lambda <= 0)
         throw new IllegalArgumentException ("lambda <= 0");
      if (delta <= 0)
         throw new IllegalArgumentException ("delta <= 0");
      if (x <= xi || x >= (xi+lambda))
         return 0.0;
      double y = (x - xi)/lambda;
      double z = gamma + delta*Math.log (y/(1.0 - y));
      return delta/(lambda*y*(1.0 - y)*Math.sqrt (2.0*Math.PI))*
           Math.exp (-z*z/2.0);
   }\end{hide}
\end{code}
\begin{tabb} Returns the density function (\ref{eq:JohnsonSB-density}).
\end{tabb}
\begin{code}

   public static double cdf (double gamma, double delta,
                             double xi, double lambda, double x)\begin{hide} {
      if (lambda <= 0)
         throw new IllegalArgumentException ("lambda <= 0");
      if (delta <= 0)
         throw new IllegalArgumentException ("delta <= 0");
      if (x <= xi)
         return 0.0;
      if (x >= xi+lambda)
         return 1.0;
      double y = (x - xi)/lambda;
      double z = gamma + delta*Math.log (y/(1.0 - y));
      return NormalDist.cdf01 (z);
   }\end{hide}
\end{code}
 \begin{tabb}
  Returns the distribution function (\ref{eq:JohnsonSB-dist}).
 \end{tabb}
\begin{code}

   public static double barF (double gamma, double delta,
                              double xi, double lambda, double x)\begin{hide} {
      if (lambda <= 0)
         throw new IllegalArgumentException ("lambda <= 0");
      if (delta <= 0)
         throw new IllegalArgumentException ("delta <= 0");
      if (x <= xi)
         return 1.0;
      if (x >= xi+lambda)
         return 0.0;
      double y = (x - xi)/lambda;
      double z = gamma + delta*Math.log (y/(1.0 - y));
      return NormalDist.barF01 (z);
   }\end{hide}
\end{code}
  \begin{tabb}
  Returns the complementary distribution.
 \end{tabb}
\begin{code}

   public static double inverseF (double gamma, double delta,
                                  double xi, double lambda, double u)\begin{hide} {
      if (lambda <= 0)
         throw new IllegalArgumentException ("lambda <= 0");
      if (delta <= 0)
         throw new IllegalArgumentException ("delta <= 0");
      if (u > 1.0 || u < 0.0)
          throw new IllegalArgumentException ("u not in [0,1]");

      if (u >= 1.0)    // if u == 1, in fact
          return xi+lambda;
      if (u <= 0.0)    // if u == 0, in fact
          return xi;

      double z = NormalDist.inverseF01 (u);
      double v = (z - gamma)/delta;

      if (v >= Num.DBL_MAX_EXP*Num.LN2)
            return xi + lambda;
      if (v <= Num.DBL_MIN_EXP*Num.LN2)
            return xi;

      v = Math.exp (v);
      return (xi + (xi+lambda)*v)/(1.0 + v);
   }\end{hide}
\end{code}
  \begin{tabb}
  Returns the inverse of the distribution (\ref{eq:JohnsonSB-inverse}).
 \end{tabb}
\begin{code}

   public static double[] getMLE (double[] x, int n,
                                  double xi, double lambda) \begin{hide} {
      if (n <= 0)
         throw new IllegalArgumentException ("n <= 0");
      final double LN_EPS = Num.LN_DBL_MIN - Num.LN2;
      double[] ftab = new double[n];
      double sum = 0.0;
      double t;

      for (int i = 0; i < n; i++) {
         t = (x[i] - xi) / lambda;
         if (t <= 0.)
            ftab[i] = LN_EPS;
         else if (t >= 1 - Num.DBL_EPSILON)
            ftab[i] = Math.log (1. / Num.DBL_EPSILON);
         else
            ftab[i] = Math.log (t/(1. - t));
         sum += ftab[i];
      }
      double empiricalMean = sum / n;

      sum = 0.0;
      for (int i = 0; i < n; i++) {
         t = ftab[i] - empiricalMean;
         sum += t * t;
      }
      double sigmaf = Math.sqrt(sum / n);

      double[] param = new double[2];
      param[0] = -empiricalMean / sigmaf;
      param[1] = 1.0 / sigmaf;

      return param;
   }\end{hide}
\end{code}
\begin{tabb}
  Estimates the parameters $(\gamma,\delta)$ of the Johnson $S_B$ distribution,
   using the maximum likelihood method, from the $n$ observations
   $x[i]$, $i = 0, 1,\ldots, n-1$.
   Parameters $\xi = \texttt{xi}$ and $\lambda = \texttt{lambda}$ are known.
   The estimated parameters are returned in a two-element
    array  in the order: [$\gamma$, $\delta$].
   \begin{detailed}
   The maximum likelihood estimators are the values $(\hat\gamma , \hat\delta)$
   that satisfy the equations \cite{tFLY06a}
   $\hat\gamma = -\bar f / s_f$ and $\hat\delta = 1/s_f$,
   where $f = \ln (t/(1-t))$, $\bar f$ is the sample mean of the $f_i$, and
   \[
   s_f = \sqrt{\frac 1n \sum_{i=0}^{n-1} (f_i - \bar f)^2},
   \]
   with $f_i = \ln (t_i/(1-t_i))$.
   \end{detailed}
\end{tabb}
\begin{htmlonly}
   \param{x}{the list of observations to use to evaluate parameters}
   \param{n}{the number of observations to use to evaluate parameters}
   \param{xi}{parameter $\xi$}
   \param{lambda}{parameter $\lambda$}
   \return{returns the parameters [$\hat{\gamma}$, $\hat{\delta}$]}
\end{htmlonly}
\begin{code}

   public static JohnsonSBDist getInstanceFromMLE (double[] x, int n,
                                                   double xi, double lambda)\begin{hide} {
      double parameters[] = getMLE (x, n, xi, lambda);
      return new JohnsonSBDist (parameters[0], parameters[1], xi, lambda);
   }\end{hide}
\end{code}
\begin{tabb}
   Creates a new instance of a \texttt{JohnsonSBDist} object
   using the maximum likelihood method based
   on the $n$ observations $x[i]$, $i = 0, 1, \ldots, n-1$.
   Given the parameters $\xi = \texttt{xi}$ and $\lambda = \texttt{lambda}$,
   the parameters $\gamma$ and $\delta$ are estimated from the observations.
\end{tabb}
\begin{htmlonly}
   \param{x}{the list of observations to use to evaluate parameters}
   \param{n}{the number of observations to use to evaluate parameters}
   \param{xi}{parameter $\xi$}
   \param{lambda}{parameter $\lambda$}
\end{htmlonly}
\begin{code}

   public static double getMean (double gamma, double delta,
                                 double xi, double lambda)\begin{hide} {
      if (lambda <= 0)
         throw new IllegalArgumentException ("lambda <= 0");
      if (delta <= 0)
         throw new IllegalArgumentException ("delta <= 0");
      double[] tpsi = new double[1];
      double mu = getMeanPsi (gamma, delta, xi, lambda, tpsi);
      return xi + lambda * mu;
   }\end{hide}
\end{code}
  \begin{tabb}
  Returns the mean \latex{\cite{tJOH49a}}
   of the Johnson $S_B$ distribution with parameters
   $\gamma$, $\delta$, $\xi$ and $\lambda$.
 \end{tabb}
\begin{code}

   public static double getVariance (double gamma, double delta,
                                     double xi, double lambda)\begin{hide} {
      if (lambda <= 0.0)
         throw new IllegalArgumentException ("lambda <= 0");
      if (delta <= 0.0)
         throw new IllegalArgumentException ("delta <= 0");

      final int NMAX = 10000;
      final double EPS = 1.0e-15;

      double a1 = 1.0/(2.0*delta*delta);
      double a2 = (1.0 - 2.0*delta*gamma)/(2.0*delta*delta);
      double a3 = (gamma - 1./delta)/delta;
      double v;
      int n = 0;
      double tem = 1;
      double sum = 0;
      while (Math.abs(tem) > EPS*Math.abs(sum) && n < NMAX) {
         ++n;
         v = Math.exp(-n*gamma/delta) - Math.exp(n*a3);
         tem = n*Math.exp(-n*n*a1) * v / (1 + Math.exp(-2*n*a1));
       //  tem = n*Math.exp(-n*n*a1) * Math.sinh(n*a2) / Math.cosh(n*a1);
         sum += tem;
      }
      if (n >= NMAX)
         System.err.println ("JohnsonSBDist:  possible lack of accuracy on variance");
      double A = -sum / (delta*delta);

      a1 = Math.PI * Math.PI * delta * delta;
      a2 = Math.PI * delta * gamma;
      int j;
      n = 0;
      tem = 1;
      sum = 0;
      while (Math.abs(tem) > EPS*Math.abs(sum) && n < NMAX) {
         ++n;
         j = 2*n - 1;
         tem = j*Math.exp(-j*j*a1/2.0) * Math.cos(j*a2) / Math.sinh(j*a1);
         sum += tem;
      }
      if (n >= NMAX)
         System.err.println ("JohnsonSBDist:  possible lack of accuracy on variance");
      double B = 2.0* a1 * sum;

      a1 = 2*Math.PI * Math.PI * delta * delta;
      a2 = 2*Math.PI * delta * gamma;
      n = 0;
      tem = 1;
      sum = 0;
      while (Math.abs(tem) > EPS*Math.abs(sum) && n < NMAX) {
         ++n;
         tem = n * Math.exp(-n*n*a1) * Math.sin(n*a2);
         sum += tem;
      }
      if (n >= NMAX)
         System.err.println ("JohnsonSBDist:  possible lack of accuracy on variance");
      double C = - 4.0 * Math.PI * delta * sum;

      double D = Math.sqrt(2*Math.PI) * Math.exp(0.5 * gamma* gamma);
      double[] tpsi = new double[1];
      double mu = getMeanPsi (gamma, delta, xi, lambda, tpsi);

      double tvar = mu *(1 - delta * gamma) - mu *mu +
                    delta / tpsi[0] * (A - B - mu * C * D);
      return lambda*lambda*tvar;

   }\end{hide}
\end{code}
\begin{tabb}  Returns the variance \latex{\cite{tFLY06a}}
   of the Johnson $S_B$ distribution with parameters $\gamma$, $\delta$, $\xi$
   and $\lambda$.
\end{tabb}
\begin{htmlonly}
   \return{the variance of the Johnson $S_B$ distribution.}
\end{htmlonly}
\begin{code}

  public static double getStandardDeviation (double gamma, double delta,
                                             double xi, double lambda)\begin{hide} {
      return Math.sqrt (JohnsonSBDist.getVariance (gamma, delta, xi, lambda));
   }\end{hide}
\end{code}
\begin{tabb}  Returns the standard deviation of the Johnson $S_B$
   distribution with parameters $\gamma$, $\delta$, $\xi$, $\lambda$.
\end{tabb}
\begin{htmlonly}
   \return{the standard deviation of the Johnson $S_B$ distribution}
\end{htmlonly}
\begin{code}

   public void setParams (double gamma, double delta,
                          double xi, double lambda)\begin{hide} {
      setParams0(gamma, delta, xi, lambda);
      setLastParams(xi, lambda);
   }\end{hide}
\end{code}
  \begin{tabb}
  Sets the value of the parameters $\gamma$, $\delta$, $\xi$ and
  $\lambda$ for this object.
 \end{tabb}
\begin{code}\begin{hide}
}\end{hide}
\end{code}
