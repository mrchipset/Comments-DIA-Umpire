\defmodule {PoissonDist}

Extends the class \class{DiscreteDistributionInt} for the
{\em Poisson\/} distribution
\cite[page 325]{sLAW00a} with mean $\lambda\ge 0$.
The mass function is
\begin{htmlonly}
\eq
  p(x)  = e^{-\lambda} \lambda^x/(x!),
  \qquad\mbox{for } x=0,1,\dots
\endeq
\end{htmlonly}%
\begin{latexonly}%
\begin{eqnarray}
  p(x) & =& \frac{e^{-\lambda} \lambda^x}{x!},
  \qquad\mbox{for } x=0,1,\dots \label{eq:fmass-Poisson}
\end{eqnarray}
\end{latexonly}%
and the distribution function is
\begin{htmlonly}
\eq
  F(x)  =  e^{-\lambda} \sum_{j=0}^x\; \lambda^j/(j!),
          \qquad\mbox{for } x=0,1,\dots .
\endeq
\end{htmlonly}
\begin{latexonly}\begin{eqnarray}
  F(x) & =&  e^{-\lambda} \sum_{j=0}^x\; \frac{\lambda^j}{j!},
          \qquad\mbox{for } x=0,1,\dots . \label{eq:FPoisson}
\end{eqnarray}
\end{latexonly}%
% The static methods do not require creating an object but always recompute
% everything from scratch.
If one has to compute $p(x)$ and/or $F(x)$ for several values of $x$
% call these functions several times
with the same $\lambda$, where $\lambda$ is not too large, then it is more
efficient to instantiate an object and use the non-static methods, since
the functions will then be computed once and kept in arrays.

For the static methods that compute $F(x)$ and $\bar{F}(x)$,
we exploit the relationship
$F(x) = 1 - G_{x+1}(\lambda)$, where $G_{x+1}$ is the \emph{gamma}
distribution function with parameters $(\alpha,\lambda) = (x+1, 1)$.
%
\hpierre{We should explain a bit more how things are done;
         e.g., use of the gamma function, etc.}


\bigskip\hrule

%%%%%%%%%%%%%%%%%%%%%%%%%%%%%%%%%%%%%%%%
\begin{code}
\begin{hide}
/*
 * Class:        PoissonDist
 * Description:  Poisson distribution
 * Environment:  Java
 * Software:     SSJ
 * Copyright (C) 2001  Pierre L'Ecuyer and Universite de Montreal
 * Organization: DIRO, Universite de Montreal
 * @author
 * @since

 * SSJ is free software: you can redistribute it and/or modify it under
 * the terms of the GNU General Public License (GPL) as published by the
 * Free Software Foundation, either version 3 of the License, or
 * any later version.

 * SSJ is distributed in the hope that it will be useful,
 * but WITHOUT ANY WARRANTY; without even the implied warranty of
 * MERCHANTABILITY or FITNESS FOR A PARTICULAR PURPOSE.  See the
 * GNU General Public License for more details.

 * A copy of the GNU General Public License is available at
   <a href="http://www.gnu.org/licenses">GPL licence site</a>.
 */
\end{hide}
package umontreal.iro.lecuyer.probdist;\begin{hide}
import umontreal.iro.lecuyer.util.*;\end{hide}

public class PoissonDist extends DiscreteDistributionInt\begin{hide} {

   private double lambda;
\end{hide}
\end{code}
\unmoved\begin{detailed}
%%%%%%%%%%%%%%%%%%%%%%%%%%%%%%%%%%%%%%%%%
\subsubsection* {Constant}

\begin{code}

   public static double MAXLAMBDA = 100000;
\end{code}
 \begin{tabb} The value of the parameter  $\lambda$ above which
  the tables are {\em not\/} precomputed by the constructor.
\end{tabb}
\end{detailed}

%%%%%%%%%%%%%%%%%%%%%%%%%%%%%%%%%%%%%%%%%
\subsubsection* {Constructor}
\begin{code}

   public PoissonDist (double lambda)\begin{hide} {
      setLambda (lambda);
   }\end{hide}
\end{code}
  \begin{tabb}
   Creates an object that contains
   the probability and distribution functions, for the Poisson
   distribution with parameter \texttt{lambda}, which are
   computed and stored in dynamic arrays inside that object.
 \end{tabb}

%%%%%%%%%%%%%%%%%%%%%%%%%%%%%%%%%%%
\subsubsection* {Methods}

\begin{code}\begin{hide}

   public double prob (int x) {
      if (x < 0)
         return 0.0;
      if (pdf == null)
         return prob (lambda, x);
      if (x > xmax || x < xmin)
         return prob (lambda, x);
      return pdf[x - xmin];
   }

   public double cdf (int x) {
      double Sum = 0.0;
      int j;

      if (x < 0)
         return 0.0;
      if (lambda == 0.0)
         return 1.0;

      /* For large lambda, we use the Chi2 distribution according to the exact
         relation, with 2x + 2 degrees of freedom

         cdf (lambda, x) = 1 - chiSquare (2x + 2, 2*lambda)

         which equals also 1 - gamma (x + 1, lambda) */
      if (cdf == null)
         return GammaDist.barF (x + 1.0, 15, lambda);

      if (x >= xmax)
         return 1.0;

      if (x < xmin) {
         // Sum a few terms to get a few decimals far in the lower tail. One
         // could also call GammaDist.barF instead.
         final int RMAX = 20;
         int i;
         double term = prob(lambda, x);
         Sum = term;
         i = x;
         while (i > 0 && i >= x - RMAX) {
            term = term * i / lambda;
            i--;
            Sum += term;
         }
         return Sum;
      }

      if (x <= xmed)
         return cdf[x - xmin];
      else
         // We keep the complementary distribution in the upper part of cdf
         return 1.0 - cdf[x + 1 - xmin];
   }


   public double barF (int x) {
      /*
       * poisson (lambda, x) = 1 - cdf (lambda, x - 1)
       */

      if (x <= 0)
         return 1.0;

      /* For large lambda,  we use the Chi2 distribution according to the exact
         relation, with 2x + 2 degrees of freedom

         cdf (lambda, x) = 1 - ChiSquare.cdf (2x + 2, 2*lambda)
         cdf (lambda, x) = 1 - GammaDist.cdf (x + 1, lambda)
       */

      if (cdf == null)
         return GammaDist.cdf ((double)x, 15, lambda);

      if (x > xmax)
//         return GammaDist.cdf ((double)x, 15, lambda);
         return PoissonDist.barF(lambda, x);
      if (x <= xmin)
         return 1.0;
      if (x > xmed)
         // We keep the complementary distribution in the upper part of cdf
         return cdf[x - xmin];
      else
         return 1.0 - cdf[x - 1 - xmin];
   }


   public int inverseFInt (double u) {
      if ((cdf == null) || (u <= EPSILON))
         return inverseF (lambda, u);
      return super.inverseFInt (u);
   }

   public double getMean() {
      return PoissonDist.getMean (lambda);
   }

   public double getVariance() {
      return PoissonDist.getVariance (lambda);
   }

   public double getStandardDeviation() {
      return PoissonDist.getStandardDeviation (lambda);
   }\end{hide}

   public static double prob (double lambda, int x)\begin{hide} {
      if (x < 0)
         return 0.0;

      if (lambda >= 100.0) {
         if ((double) x >= 10.0*lambda)
            return 0.0;
      } else if (lambda >= 3.0) {
         if ((double) x >= 100.0*lambda)
            return 0.0;
      } else {
         if ((double) x >= 200.0*Math.max(1.0, lambda))
            return 0.0;
      }

      final double LAMBDALIM = 20.0;
      double Res;
      if (lambda < LAMBDALIM && x <= 100)
         Res = Math.exp (-lambda)*Math.pow (lambda, x)/Num.factorial (x);
      else {
         double y = x*Math.log (lambda) - Num.lnGamma (x + 1.0) - lambda;
         Res = Math.exp (y);
      }
      return Res;
   }\end{hide}
\end{code}
 \begin{tabb}  Computes and returns the Poisson probability
  $p(x)$ for $\lambda = $ \texttt{lambda}\html{.}\latex{, as
 defined in (\ref{eq:fmass-Poisson})}.
\begin{detailed}
 If $\lambda\ge 20$, this (static) method uses the logarithm of the gamma
  function, defined in (\ref{eq:Gamma}), to estimate the density.
\end{detailed}
 \end{tabb}
\begin{code}

   public static double cdf (double lambda, int x)\begin{hide} {
   /*
    * On our machine, computing a value using gamma is faster than the
    * naive computation for lambdalim > 200.0, slower for lambdalim < 200.0
    */
      if (lambda < 0.0)
        throw new IllegalArgumentException ("lambda < 0");
      if (lambda == 0.0)
         return 1.0;
      if (x < 0)
         return 0.0;

      if (lambda >= 100.0) {
         if ((double) x >= 10.0*lambda)
            return 1.0;
      } else {
         if ((double) x >= 100.0*Math.max(1.0, lambda))
            return 1.0;
      }

      /* If lambda > LAMBDALIM, use the Chi2 distribution according to the
         exact relation, with 2x + 2 degrees of freedom

         poisson (lambda, x) = 1 - chiSquare (2x + 2, 2*lambda)

         which also equals 1 - gamma (x + 1, lambda) */

      final double LAMBDALIM = 200.0;
      if (lambda > LAMBDALIM)
         return GammaDist.barF (x + 1.0, 15, lambda);

      if (x >= lambda)
         return 1 - PoissonDist.barF(lambda, x+1);

      // Naive computation: sum all prob. from i = x
      double sum = 1;
      double term = 1;
      for(int j = 1; j <= x; j++) {
         term *= lambda/j;
         sum += term;
      }
      return sum*Math.exp(-lambda);
   }\end{hide}
\end{code}
  \begin{tabb}  Computes and returns the value of the Poisson
  distribution function $F(x)$ for $\lambda = $ \texttt{lambda}\html{.}\latex{,
as defined in (\ref{eq:FPoisson}).}
\begin{detailed}
  To compute $F(x)$, all non-negligible terms of the sum are added
  if $\lambda \le 200$; otherwise, the relationship
  $F_\lambda (x) = 1 - G_{x + 1}(\lambda)$ is used,
  where $G_{x+1}$ is the gamma distribution function with parameter
  $\alpha = x+1$ (see \class{GammaDist}).
\end{detailed}
 \hrichard{Cette documentation est ici, parce que ce paragraphe s'applique
    seulement \`a la version \texttt{static}.}
 \hpierre{Suggestions: lorsque $x>\lambda$, calculer plutot
   les termes de $1-F (x)$ ? Et si $\lambda > 200$ mais $x$ est petit,
   ne vaut-il pas mieux utiliser la somme? Peut-etre mettre la limite
   sur $x$ a la place? }
 \hrichard{ Si $\lambda$ est grand, on risque d'avoir d\'epassement de
   capacit\'e avec le facteur $e^{-\lambda}$ et l'autre facteur aussi,
   si on somme les termes
   explicitement. D'ailleurs, avec grand $\lambda$, la probabilit\'e
   d'avoir des petits $x$ est n\'egligeable. C'est pourquoi j'ai mis la
   limite sur $\lambda$ et non sur $x$.\\ Pour $x>\lambda$, la queue
   de Poisson \'etant plus allong\'ee vers le haut que vers le bas, il
   n'est pas \'evident que ce serait plus efficace de calculer
   $1-F (x)$,  et ce serait beaucoup plus compliqu\'e.}
 \hpierre{Et pourquoi prendre $10^5$ au lieu de 150 dans PoissonDist?
   Si on a 5 a 10 appels a faire, il me semble qu'appeler ``gamma''
   est beaucoup plus efficace que de calculer des tableaux de taille
   $10^5$? }
 \hrichard{Dans le cas o\`u $\lambda = 10^5$, on ne calcule pas des
   tableaux de taille $10^5$, mais 2 tableaux de taille 4849. Les
   autres termes sont $< 10^{-16}$ et ne sont ni calcul\'es ni
   conserv\'es.}
 \end{tabb}
\begin{code}

   public static double barF (double lambda, int x)\begin{hide} {
      if (lambda < 0)
         throw new IllegalArgumentException ("lambda < 0");
      if (x <= 0)
         return 1.0;

      if (lambda >= 100.0) {
         if ((double) x >= 10.0*lambda)
            return 0.0;
      } else {
         if ((double) x >= 100 + 100.0*Math.max(1.0, lambda))
            return 0.0;
      }

      /* If lambda > LAMBDALIM, we use the Chi2 distribution according to the
         exact relation, with 2x + 2 degrees of freedom

         cdf (lambda, x) = 1 - ChiSquare.cdf (2x + 2, 2*lambda)

         which also equals   1 - GammaDist.cdf (x + 1, lambda) */

      final double LAMBDALIM = 200.0;
      if (lambda > LAMBDALIM)
         return GammaDist.cdf ((double)x, 15, lambda);

      if (x <= lambda)
         return 1.0 - PoissonDist.cdf(lambda, x - 1);

      // Naive computation: sum all prob. from i = x to i = oo
      double term, sum;
      final int IMAX = 20;

      // Sum at least IMAX prob. terms from i = s to i = oo
      sum = term = PoissonDist.prob(lambda, x);
      int i = x + 1;
      while (term > EPSILON || i <= x + IMAX) {
         term *= lambda/i;
         sum += term;
         i++;
      }
      return sum;
   } \end{hide}
\end{code}
  \begin{tabb}  Computes and returns the value of the complementary Poisson
   distribution function, for $\lambda = $ \texttt{lambda}.
%   Computes and adds the non-negligible terms in the tail.
   \emph{WARNING:} The complementary distribution function is defined as
    $\bar F(x) = P[X \ge x]$.
 \end{tabb}
\begin{code}

   public static int inverseF (double lambda, double u)\begin{hide} {
      if (u < 0.0 || u > 1.0)
         throw new IllegalArgumentException ("u is not in range [0,1]");
      if (lambda < 0.0)
         throw new IllegalArgumentException ("lambda < 0");
      if (u >= 1.0)
         return Integer.MAX_VALUE;
      if (u <= prob (lambda, 0))
         return 0;
      int i;

      final double LAMBDALIM = 700.0;
      if (lambda < LAMBDALIM) {
         double sumprev = -1.0;
         double term = Math.exp(-lambda);
         double sum = term;
         i = 0;
         while (sum < u && sum > sumprev) {
            i++;
            term *= lambda / i;
            sumprev = sum;
            sum += term;
         }
         return i;

      } else {
         i = (int)lambda;
         double term = PoissonDist.prob(lambda, i);
         while ((term >= u) && (term > Double.MIN_NORMAL)) {
            i /= 2;
            term = PoissonDist.prob (lambda, i);
         }
         if (term <= Double.MIN_NORMAL) {
            i *= 2;
            term = PoissonDist.prob (lambda, i);
            while (term >= u && (term > Double.MIN_NORMAL)) {
               term *= i / lambda;
               i--;
            }
         }
         int mid = i;
         double sum = term;
         double termid = term;

         while (term >= EPSILON*u && i > 0) {
            term *= i / lambda;
            sum += term;
            i--;
         }

         term = termid;
         i = mid;
        double prev = -1;
        if (sum < u) {
            while ((sum < u) && (sum > prev)) {
               i++;
               term *= lambda / i;
               prev = sum;
               sum += term;
            }
         } else {
            // The computed CDF is too big so we substract from it.
            sum -= term;
            while (sum >= u) {
               term *= i / lambda;
               i--;
               sum -= term;
            }
         }
      }
      return i;
   }\end{hide}
\end{code}
\begin{tabb} Performs a linear search to get the inverse function without
   precomputed tables.
\end{tabb}
\begin{code}

   public static double[] getMLE (int[] x, int n)\begin{hide} {
      if (n <= 0)
         throw new IllegalArgumentException ("n <= 0");

      double parameters[];
      parameters = new double[1];
      double sum = 0.0;
      for (int i = 0; i < n; i++) {
         sum += x[i];
      }

      parameters[0] = (double) sum / (double) n;
      return parameters;
   }\end{hide}
\end{code}
\begin{tabb}
   Estimates the parameter $\lambda$ of the Poisson distribution
   using the maximum likelihood method, from the $n$ observations
   $x[i]$, $i = 0, 1, \ldots, n-1$.
%   \begin{detailed}
   The maximum likelihood estimator $\hat{\lambda}$ satisfy the equation
   $\hat{\lambda} = \bar{x}_n$,
   where  $\bar{x}_n$ is the average of $x[0], \ldots, x[n-1]$
   (see \cite[page 326]{sLAW00a}).
%   \end{detailed}
\end{tabb}
\begin{htmlonly}
   \param{x}{the list of observations used to evaluate parameters}
   \param{n}{the number of observations used to evaluate parameters}
   \return{returns the parameter [$\hat{\lambda}$]}
\end{htmlonly}
\begin{code}

   public static PoissonDist getInstanceFromMLE (int[] x, int n)\begin{hide} {
      double parameters[] = getMLE (x, n);
      return new PoissonDist (parameters[0]);
   }\end{hide}
\end{code}
\begin{tabb}
   Creates a new instance of a Poisson distribution with parameter $\lambda$
   estimated using the maximum likelihood method based on the $n$ observations
   $x[i]$, $i = 0, 1, \ldots, n-1$.
\end{tabb}
\begin{htmlonly}
   \param{x}{the list of observations to use to evaluate parameters}
   \param{n}{the number of observations to use to evaluate parameters}
\end{htmlonly}
\begin{code}

   public static double getMean (double lambda)\begin{hide} {
      if (lambda < 0.0)
       throw new IllegalArgumentException ("lambda < 0");

      return lambda;
   }\end{hide}
\end{code}
\begin{tabb}  Computes and returns the mean $E[X] = \lambda$ of the
   Poisson distribution with parameter $\lambda$.
\end{tabb}
\begin{htmlonly}
   \return{the mean of the Poisson distribution $E[X] = \lambda$}
\end{htmlonly}
\begin{code}

   public static double getVariance (double lambda)\begin{hide} {
      if (lambda < 0.0)
       throw new IllegalArgumentException ("lambda < 0");

      return lambda;
   }\end{hide}
\end{code}
\begin{tabb}  Computes and returns the variance $= \lambda$
   of the Poisson distribution with parameter $\lambda$.
\end{tabb}
\begin{htmlonly}
   \return{the variance of the Poisson distribution
    $\mbox{Var}[X] = \lambda$}
\end{htmlonly}
\begin{code}

   public static double getStandardDeviation (double lambda)\begin{hide} {
      if (lambda < 0.0)
       throw new IllegalArgumentException ("lambda < 0");

      return Math.sqrt (lambda);
   }\end{hide}
\end{code}
\begin{tabb}  Computes and returns the standard deviation of the
   Poisson distribution with parameter $\lambda$.
\end{tabb}
\begin{htmlonly}
   \return{the standard deviation of the Poisson distribution}
\end{htmlonly}
\begin{code}

   public double getLambda()\begin{hide} {
      return lambda;
   }
\end{hide}
\end{code}
\begin{tabb}
   Returns the $\lambda$ associated with this object.
\end{tabb}
\begin{code}

   public void setLambda (double lambda)\begin{hide} {
      supportA = 0;
      if (lambda < 0.0)
         throw new IllegalArgumentException ("lambda < 0");
      this.lambda = lambda;

      // For lambda > MAXLAMBDAPOISSON, we do not use pre-computed arrays
      if (lambda > MAXLAMBDA) {
         pdf = null;
         cdf = null;
         return;
      }

      double epsilon;
      int i, mid, Nmax;
      int imin, imax;
      double sum;
      double[] P;    // Poisson probability terms
      double[] F;    // Poisson cumulative probabilities

      // In theory, the Poisson distribution has an infinite range. But
      // for i > Nmax, probabilities should be extremely small.
      Nmax = (int)(lambda + 16*(2 + Math.sqrt (lambda)));
      P = new double[1 + Nmax];

      mid = (int)lambda;
      epsilon = EPSILON * EPS_EXTRA/prob (lambda, mid);
      // For large lambda, mass will lose a few digits of precision
      // We shall normalize by explicitly summing all terms >= epsilon
      sum = P[mid] = 1.0;

      // Start from the maximum and compute terms > epsilon on each side.
      i = mid;
      while (i > 0 && P[i] > epsilon) {
         P[i - 1] = P[i]*i/lambda;
         i--;
         sum += P[i];
      }
      xmin = imin = i;

      i = mid;
      while (P[i] > epsilon) {
         P[i + 1] = P[i]*lambda/(i + 1);
         i++;
         sum += P[i];
         if (i >= Nmax - 1) {
            Nmax *= 2;
            double[] nT = new double[1 + Nmax];
            System.arraycopy (P, 0, nT, 0, P.length);
            P = nT;
         }
      }
      xmax = imax = i;
      F = new double[1 + Nmax];

      // Renormalize the sum of probabilities to 1
      for (i = imin; i <= imax; i++)
         P[i] /= sum;

      // Compute the cumulative probabilities until F >= 0.5, and keep them in
      // the lower part of array, i.e. F[s] contains all P[i] for i <= s
      F[imin] = P[imin];
      i = imin;
      while (i < imax && F[i] < 0.5) {
         i++;
         F[i] = P[i] + F[i - 1];
      }
      // This is the boundary between F and 1 - F in the CDF
      xmed = i;

      // Compute the cumulative probabilities of the complementary distribution
      // and keep them in the upper part of the array. i.e. F[s] contains all
      // P[i] for i >= s
      F[imax] = P[imax];
      i = imax - 1;
      do {
         F[i] = P[i] + F[i + 1];
         i--;
      } while (i > xmed);

       /* Reset imin because we lose too much precision for a few terms near
      imin when we stop adding terms < epsilon. */
      i = imin;
      while (i < xmed && F[i] < EPSILON)
         i++;
      xmin = imin = i;

      /* Same thing with imax */
      i = imax;
      while (i > xmed && F[i] < EPSILON)
         i--;
      xmax = imax = i;

      pdf = new double[imax + 1 - imin];
      cdf = new double[imax + 1 - imin];
      System.arraycopy (P, imin, pdf, 0, imax-imin+1);
      System.arraycopy (F, imin, cdf, 0, imax-imin+1);
   }\end{hide}
\end{code}
\begin{tabb}
   Sets the $\lambda$ associated with this object.
\end{tabb}
\begin{code}

   public double[] getParams ()\begin{hide} {
      double[] retour = {lambda};
      return retour;
   }\end{hide}
\end{code}
\begin{tabb}
   Return a table containing the parameter of the current distribution.
\end{tabb}
\begin{hide}\begin{code}

   public String toString () {
      return getClass().getSimpleName() + ": lambda = " + lambda;
   }
\end{code}
\begin{tabb}
   Returns a \texttt{String} containing information about the current distribution.
\end{tabb}\end{hide}
\begin{code}\begin{hide}
}\end{hide}
\end{code}
