\defmodule {FrechetDist}

Extends the class \class{ContinuousDistribution} for the  \emph{Fr\'echet}
distribution \cite[page 3]{tJOH95b}, with location parameter $\delta$, scale
 parameter $\beta > 0$, and shape parameter $\alpha > 0$, where we use
 the notation $z = (x-\delta)/\beta$. It has density
\begin{htmlonly}
\eq
    f (x) = \alpha e^{-z^{-\alpha}} /  ( \beta z^{\alpha +1} ),
              \qquad \mbox{for } x > \delta
\endeq
\end{htmlonly}
\begin{latexonly}
\[
f (x) =
 \frac{\alpha e^{-z^{-\alpha}}}  {\beta z^{\alpha +1}},
 \qquad  \mbox{for } x > \delta
\]
\end{latexonly}
and distribution function
\[
F(x) =
    e^{-z^{-\alpha}},  \qquad \mbox{for } x > \delta.
\]
Both the density and the distribution are  0 for $x \le \delta$.

The mean is given by
\begin{htmlonly}
\eq
E[X] = \delta + \beta \Gamma(1 - 1/\alpha),
\endeq
\end{htmlonly}
\begin{latexonly}
\[
E[X] = \delta + \beta \Gamma\!\left(1 - \frac1\alpha\right),
\]
\end{latexonly}
where $\Gamma(x)$ is the gamma function. The variance is
\begin{htmlonly}
\eq
\mbox{Var}[X] = \beta^2 [\Gamma(1 - 2/\alpha) - (\Gamma(1 - 1/\alpha))^2].
\endeq
\end{htmlonly}
\begin{latexonly}
\[
\mbox{Var}[X] = \beta^2 \left[\Gamma\!\left(1 - \frac2\alpha\right) -
 \Gamma^2\!\left(1 - \frac1\alpha\right)\right].
\]
\end{latexonly}


\bigskip\hrule

%%%%%%%%%%%%%%%%%%%%%%%%%%%%%%%%%%%%%%%%
\begin{code}
\begin{hide}
/*
 * Class:        FrechetDist
 * Description:  Fréchet distribution
 * Environment:  Java
 * Software:     SSJ 
 * Copyright (C) 2001  Pierre L'Ecuyer and Universite de Montreal
 * Organization: DIRO, Universite de Montreal
 * @author       
 * @since

 * SSJ is free software: you can redistribute it and/or modify it under
 * the terms of the GNU General Public License (GPL) as published by the
 * Free Software Foundation, either version 3 of the License, or
 * any later version.

 * SSJ is distributed in the hope that it will be useful,
 * but WITHOUT ANY WARRANTY; without even the implied warranty of
 * MERCHANTABILITY or FITNESS FOR A PARTICULAR PURPOSE.  See the
 * GNU General Public License for more details.

 * A copy of the GNU General Public License is available at
   <a href="http://www.gnu.org/licenses">GPL licence site</a>.
 */
\end{hide}
package umontreal.iro.lecuyer.probdist;
\begin{hide}
import umontreal.iro.lecuyer.util.*;
import optimization.*;
import umontreal.iro.lecuyer.functions.MathFunction;
\end{hide}
public class FrechetDist extends ContinuousDistribution\begin{hide} {
   private double delta;
   private double beta;
   private double alpha;


   private static class Optim implements Lmder_fcn {
      protected double[] x;
      protected int n;

      public Optim (double[] x, int n) {
         this.n = n;
         this.x = x;
      }

      public void fcn (int m, int n, double[] par, double[] fvec, double[][] fjac, int iflag[])
      {
         if (par[1] <= 0.0 || par[2] <= 0.0) {
            final double BIG = 1.0e100;
            fvec[1] = BIG;
            fvec[2] = BIG;
            fvec[3] = BIG;
            return;
         }

         double sum1, sum2, sumb, sum4, sum5;
         double z, w, v;
         double alpha = par[1];
         double beta = par[2];
         double mu = par[3];

         if (iflag[1] == 1) {
            sum1 = sum2 = sumb = sum4 = sum5 = 0;
            for (int i = 0; i < n; i++) {
               z = (x[i] - mu) / beta;
               sum1 += 1.0 / z;
               v = Math.pow(z, -alpha);
               sum2 += v / z;
               sumb += v;
               w = Math.log(z);
               sum4 += w;
               sum5 += v * w;
            }

            fvec[2] = sumb - n;   // eq. for beta
            fvec[3] = (alpha + 1) * sum1 - alpha * sum2;   // eq. for mu
            fvec[1] = n / alpha + sum5 - sum4;   // eq. for alpha

         } else if (iflag[1] == 2) {
            throw new IllegalArgumentException ("iflag = 2");
            // The 3 X 3 Jacobian must be calculated and put in fjac
         }
      }
   }


   private static class Function implements MathFunction {
      private int n;
      private double[] x;
      private double delta;
      public double sumxi;
      public double dif;

      public Function (double[] y, int n, double delta) {
         this.n = n;
         this.x = y;
         this.delta = delta;
         double xmin = Double.MAX_VALUE;
         for (int i = 0; i < n; i++) {
            if ((y[i] < xmin) && (y[i] > delta))
               xmin = y[i];
         }
         dif = xmin - delta;
      }

      public double evaluate (double alpha) {
         if (alpha <= 0.0) return 1.0e100;
         double v, w;
         double sum1 = 0, sum2 = 0, sum3 = 0;
         for (int i = 0; i < n; i++) {
            if (x[i] <= delta)
               continue;
            v = Math.log(x[i] - delta);
            w = Math.pow(dif / (x[i] - delta), alpha);
            sum1 += v;
            sum2 += w;
            sum3 += v * w;
         }

         sum1 /= n;
         sumxi = sum2 / n;
         return 1 / alpha + sum3 / sum2 - sum1;
      }
   }\end{hide}
\end{code}
%%%%%%%%%%%%%%%%%%%%%%%%%%%%%%%%%%%%%%%%%
\subsubsection* {Constructors}

\begin{code}

   public FrechetDist (double alpha)\begin{hide} {
      setParams (alpha, 1.0, 0.0);
   }\end{hide}
\end{code}
  \begin{tabb} Constructor for the standard \emph{Fr\'echet}
   distribution with parameters  $\beta$ = 1 and $\delta$ = 0.
   \end{tabb}
\begin{code}

   public FrechetDist (double alpha, double beta, double delta)\begin{hide} {
      setParams (alpha, beta, delta);
   }\end{hide}
\end{code}
  \begin{tabb} Constructs a \texttt{FrechetDist} object with parameters
  $\alpha$ = \texttt{alpha},  $\beta$ = \texttt{beta} and  $\delta$ = \texttt{delta}.
   \end{tabb}

%%%%%%%%%%%%%%%%%%%%%%%%%%%%%%%%%%%
\subsubsection* {Methods}
\begin{code}\begin{hide}

   public double density (double x) {
      return density (alpha, beta, delta, x);
   }

   public double cdf (double x) {
      return cdf (alpha, beta, delta, x);
   }

   public double barF (double x) {
      return barF (alpha, beta, delta, x);
   }

   public double inverseF (double u) {
      return inverseF (alpha, beta, delta, u);
   }

   public double getMean() {
      return getMean (alpha, beta, delta);
   }

   public double getVariance() {
      return getVariance (alpha, beta, delta);
   }

   public double getStandardDeviation() {
      return getStandardDeviation (alpha, beta, delta);
   }\end{hide}

   public static double density (double alpha, double beta, double delta,
                                 double x)\begin{hide} {
      if (beta <= 0)
         throw new IllegalArgumentException ("beta <= 0");
      if (alpha <= 0)
         throw new IllegalArgumentException ("alpha <= 0");
      final double z = (x - delta)/beta;
      if (z <= 0.0)
         return 0.0;
      double t = Math.pow (z, -alpha);
      return  alpha * t * Math.exp (-t) / (z * beta);
   }\end{hide}
\end{code}
\begin{tabb} Computes and returns the density function.
\end{tabb}
\begin{code}

   public static double cdf (double alpha, double beta, double delta,
                             double x)\begin{hide} {
      if (beta <= 0)
         throw new IllegalArgumentException ("beta <= 0");
      if (alpha <= 0)
         throw new IllegalArgumentException ("alpha <= 0");
      final double z = (x - delta)/beta;
      if (z <= 0.0)
         return 0.0;
      double t = Math.pow (z, -alpha);
      return  Math.exp (-t);
   }\end{hide}
\end{code}
 \begin{tabb}
  Computes and returns  the distribution function.
 \end{tabb}
\begin{code}

   public static double barF (double alpha, double beta, double delta,
                              double x)\begin{hide} {
      if (beta <= 0)
         throw new IllegalArgumentException ("beta <= 0");
      if (alpha <= 0)
         throw new IllegalArgumentException ("alpha <= 0");
      final double z = (x - delta)/beta;
      if (z <= 0.0)
         return 1.0;
      double t = Math.pow (z, -alpha);
      return  -Math.expm1 (-t);
   }\end{hide}
\end{code}
 \begin{tabb}
  Computes and returns  the complementary distribution function $1 - F(x)$.
 \end{tabb}
\begin{code}

   public static double inverseF (double alpha, double beta, double delta,
                                  double u)\begin{hide} {
      if (u < 0.0 || u > 1.0)
         throw new IllegalArgumentException ("u not in [0, 1]");
      if (beta <= 0)
         throw new IllegalArgumentException ("beta <= 0");
      if (alpha <= 0)
         throw new IllegalArgumentException ("alpha <= 0");
      if (u >= 1.0)
         return Double.POSITIVE_INFINITY;
      if (u <= 0.0)
         return delta;
      double t = Math.pow (-Math.log (u), 1.0/alpha);
      if (t <= Double.MIN_NORMAL)
         return Double.MAX_VALUE;
      return delta + beta / t;
   }\end{hide}
\end{code}
  \begin{tabb}
  Computes and returns the inverse distribution function.
 \end{tabb}
\begin{code}

   public static double[] getMLE (double[] x, int n, double delta)\begin{hide} {
      if (n <= 1)
         throw new IllegalArgumentException ("n <= 1");

      Function func = new Function (x, n, delta);
      double a = 1e-4;
      double b = 1.0e12;
      double alpha = RootFinder.brentDekker (a, b, func, 1e-12);
      double par[] = new double[2];
      par[0] = alpha;
      par[1] = func.dif * Math.pow (func.sumxi, -1.0/alpha);
      return par;
   }\end{hide}
\end{code}
\begin{tabb}
   Given $\delta =$ \texttt{delta}, estimates the parameters $(\alpha, \beta)$
  of the \emph{Fr\'echet} distribution
   using the maximum likelihood method with the $n$ observations
   $x[i]$, $i = 0, 1,\ldots, n-1$. The estimates are returned in a two-element
    array, in regular order: [$\alpha$, $\beta$].
   \begin{detailed}
   The maximum likelihood estimators are the values $(\hat\alpha, \hat\beta)$
   that satisfy the equations:
   \begin{eqnarray*}
      \hat\beta & = &  \left(\frac1n \sum_{i=0}^{n-1} (x_i - \delta)^{-\hat\alpha}\right)^{\!\!-1/\hat\alpha} \\[0.5em]
  \frac1n \sum_{i=0}^{n-1} \ln(x_i - \delta)  & = &   \frac 1 {\hat\alpha}  +
                              \frac{\sum_{i=0}^{n-1} (x_i - \delta)^{-\hat\alpha}\ln(x_i - \delta)}
                               {\sum_{i=0}^{n-1} (x_i - \delta)^{-\hat\alpha}}.
   \end{eqnarray*}
   \end{detailed}
\end{tabb}
\begin{htmlonly}
   \param{x}{the list of observations used to evaluate parameters}
   \param{n}{the number of observations used to evaluate parameters}
   \param{delta}{location parameter}
   \return{returns the parameters [$\hat{\alpha}$, $\hat{\beta}$]}
\end{htmlonly}
\begin{code}

   public static FrechetDist getInstanceFromMLE (double[] x, int n,
                                                 double delta)\begin{hide} {
      double par[] = getMLE (x, n, delta);
      return new FrechetDist (par[0], par[1], delta);
   }\end{hide}
\end{code}
\begin{tabb}
Given $\delta =$ \texttt{delta}, creates a new instance of a \emph{Fr\'echet}
distribution with parameters $\alpha$ and $\beta$ estimated using the
maximum likelihood method based on the $n$ observations $x[i]$,
$i = 0, 1, \ldots, n-1$.
\end{tabb}
\begin{htmlonly}
   \param{x}{the list of observations to use to evaluate parameters}
   \param{n}{the number of observations to use to evaluate parameters}
   \param{delta}{location parameter}
\end{htmlonly}
% \begin{code}
%
%    public static double[] getMLE (double[] x, int n)\begin{hide} {
%       if (n <= 1)
%          throw new IllegalArgumentException ("n <= 1");
%     throw new UnsupportedOperationException("NOT IMPLEMENTED");
%     // TO DO:  The Jacobian must be calculated and put in class Optim above
%     // Then it will be finished.
%        final int m = 3;
%       double[] param = new double[m + 1];
%       param[1] = param[2] = param[3] = 1;
%
%       double[] fvec = new double [m + 1];
%       double[][] fjac = new double[m + 1][m + 1];
%       int[] iflag = new int[2];
%       int[] info = new int[2];
%       int[] ipvt = new int[m + 1];
%       Optim system = new Optim (x, n);
%
%       Minpack_f77.lmder1_f77 (system, m, m, param, fvec, fjac, 1e-5, info, ipvt);
%
%       double par[] = new double[m];
%       par[0] = param[1];
%       par[1] = param[2];
%       par[2] = param[3];
%
%       return par;
%    }\end{hide}
% \end{code}
% \begin{tabb}
%    Estimates the parameters $(\alpha, \beta, \delta)$
%   of the \emph{Fr\'echet} distribution
%    using the maximum likelihood method with the $n$ observations
%    $x[i]$, $i = 0, 1,\ldots, n-1$. The estimates are returned in a three-element
%     array, in regular order: [$\alpha$, $\beta$, $\delta$].
%    \begin{detailed}
%    The maximum likelihood estimators are the values $(\hat\alpha, \hat\beta, \hat{\delta})$
%    that satisfy the equations:
%    \begin{eqnarray*}
%       n & = &  \sum_{i=0}^{n-1} z_i^{-\hat\alpha}, \\[0.5em]
%      (\hat\alpha + 1)\sum_{i=0}^{n-1} z_i^{-1}   & = &
%           \hat\alpha \sum_{i=0}^{n-1} z_i^{-\hat\alpha - 1}, \\[0.5em]
%    \sum_{i=0}^{n-1} \ln (z_i)  & = &   \frac n {\hat\alpha}  +
%                               {\sum_{i=0}^{n-1} z_i^{-\hat\alpha}\ln(z_i)},
%    \end{eqnarray*}
%    where we use the notation $z_i = (x_i - \hat\delta) / \hat\beta$.
%    \end{detailed}
% \end{tabb}
% \begin{htmlonly}
%    \param{x}{the list of observations used to evaluate parameters}
%    \param{n}{the number of observations used to evaluate parameters}
%    \return{returns the parameters [$\hat{\alpha}$, $\hat{\beta}$, $\hat{\delta}$]}
% \end{htmlonly}
% \begin{code}
%
%    public static FrechetDist getInstanceFromMLE (double[] x, int n)\begin{hide} {
%       double par[] = getMLE (x, n);
%       return new FrechetDist (par[0], par[1], par[2]);
%    }\end{hide}
% \end{code}
% \begin{tabb}
% Creates a new instance of a \emph{Fr\'echet}
% distribution with parameters $\alpha$, $\beta$ and $\delta$ estimated using the
% maximum likelihood method based on the $n$ observations $x[i]$,
% $i = 0, 1, \ldots, n-1$.
% \end{tabb}
% \begin{htmlonly}
%    \param{x}{the list of observations to use to evaluate parameters}
%    \param{n}{the number of observations to use to evaluate parameters}
% \end{htmlonly}
\begin{code}

   public static double getMean (double alpha, double beta, double delta)\begin{hide} {
      if (beta <= 0)
         throw new IllegalArgumentException ("beta <= 0");
      if (alpha <= 1)
         throw new IllegalArgumentException ("alpha <= 1");
      double t = Num.lnGamma(1.0 - 1.0/alpha);
      return delta + beta * Math.exp(t);
   }\end{hide}
\end{code}
\begin{tabb}  Returns the mean of the \emph{Fr\'echet} distribution with
 parameters $\alpha$, $\beta$ and $\delta$.
\end{tabb}
\begin{htmlonly}
   \return{the mean}
\end{htmlonly}
\begin{code}

   public static double getVariance (double alpha, double beta,
                                     double delta)\begin{hide} {
      if (beta <= 0)
         throw new IllegalArgumentException ("beta <= 0");
      if (alpha <= 2)
         throw new IllegalArgumentException ("alpha <= 2");
      double t = Num.lnGamma(1.0 - 1.0/alpha);
      double mu = Math.exp(t);
      double v = Math.exp(Num.lnGamma(1.0 - 2.0/alpha));
      return beta * beta * (v - mu * mu);
   }\end{hide}
\end{code}
\begin{tabb}  Returns the variance of the \emph{Fr\'echet} distribution with parameters
   $\alpha$, $\beta$ and $\delta$.
\end{tabb}
\begin{htmlonly}
   \return{the variance}
\end{htmlonly}
\begin{code}

   public static double getStandardDeviation (double alpha, double beta,
                                              double delta)\begin{hide} {
      return  Math.sqrt(getVariance (alpha, beta, delta));
   }\end{hide}
\end{code}
\begin{tabb}  Returns the standard deviation of the \emph{Fr\'echet} distribution
with parameters $\alpha$, $\beta$ and $\delta$.
\end{tabb}
\begin{htmlonly}
   \return{the standard deviation}
\end{htmlonly}
\begin{code}

   public double getAlpha()\begin{hide} {
      return alpha;
   }\end{hide}
\end{code}
  \begin{tabb} Returns the parameter $\alpha$  of this object.
  \end{tabb}
\begin{code}

   public double getBeta()\begin{hide} {
      return beta;
   }\end{hide}
\end{code}
  \begin{tabb} Returns the parameter $\beta$  of this object.
  \end{tabb}
\begin{code}

   public double getDelta()\begin{hide} {
      return delta;
   }
\end{hide}
\end{code}
 \begin{tabb}
   Returns the parameter $\delta$ of this object.
 \end{tabb}
\begin{code}

   public void setParams (double alpha, double beta, double delta)\begin{hide} {
      if (beta <= 0)
         throw new IllegalArgumentException ("beta <= 0");
      if (alpha <= 0)
         throw new IllegalArgumentException ("alpha <= 0");
      this.delta  = delta;
      this.beta = beta;
      this.alpha = alpha;
   }\end{hide}
\end{code}
 \begin{tabb}
   Sets the parameters  $\alpha$, $\beta$  and $\delta$ of this object.
 \end{tabb}
\begin{code}

   public double[] getParams()\begin{hide} {
      double[] retour = {alpha, beta, delta};
      return retour;
   }\end{hide}
\end{code}
\begin{tabb}
   Return an array containing the parameters of the current object
    in regular order: [$\alpha$, $\beta$, $\delta$].
\end{tabb}
\begin{hide}\begin{code}

   public String toString ()\begin{hide} {
      return getClass().getSimpleName() + " : alpha = " + alpha + ", beta = " + beta + ", delta = " + delta;
   }\end{hide}
\end{code}
\begin{tabb}
   Returns a \texttt{String} containing information about the current distribution.
\end{tabb}\end{hide}
\begin{code}\begin{hide}
}\end{hide}
\end{code}
