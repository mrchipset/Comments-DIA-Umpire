\defmodule {PascalDist}

The \emph{Pascal} distribution is a special case of the
\emph{negative binomial\/} distribution
\cite[page 324]{sLAW00a} with
parameters $n$ and $p$, where $n$ is a positive
integer and $0\le p\le 1$.
Its mass function is
\begin{htmlonly}
\eq
  p(x) = \mbox{nCr}(n + x - 1, x) p^n (1 - p)^{x},
    \qquad\mbox{for } x = 0, 1, 2, \ldots
\endeq
where nCr is defined in \class{BinomialDist}.
\end{htmlonly}
\begin{latexonly}
\eq
  p(x) = \binom{n + x - 1}{x} p^n (1 - p)^{x},
    \qquad\mbox{for } x = 0, 1, 2, \ldots \eqlabel{eq:fmass-pascal}
\endeq
\end{latexonly}
This $p(x)$ can be interpreted as the probability of having $x$ failures
before the $n$th success in a sequence of independent Bernoulli trials
with  probability of success $p$.
For $n=1$, this gives the \emph{geometric} distribution.


\bigskip\hrule

%%%%%%%%%%%%%%%%%%%%%%%%%%%%%%%%%%%%%%%%
\begin{code}
\begin{hide}
/*
 * Class:        PascalDist
 * Description:  Pascal distribution
 * Environment:  Java
 * Software:     SSJ 
 * Copyright (C) 2001  Pierre L'Ecuyer and Universite de Montreal
 * Organization: DIRO, Universite de Montreal
 * @author       
 * @since

 * SSJ is free software: you can redistribute it and/or modify it under
 * the terms of the GNU General Public License (GPL) as published by the
 * Free Software Foundation, either version 3 of the License, or
 * any later version.

 * SSJ is distributed in the hope that it will be useful,
 * but WITHOUT ANY WARRANTY; without even the implied warranty of
 * MERCHANTABILITY or FITNESS FOR A PARTICULAR PURPOSE.  See the
 * GNU General Public License for more details.

 * A copy of the GNU General Public License is available at
   <a href="http://www.gnu.org/licenses">GPL licence site</a>.
 */
\end{hide}
package umontreal.iro.lecuyer.probdist;
\begin{hide}
import umontreal.iro.lecuyer.util.RootFinder;
import umontreal.iro.lecuyer.functions.MathFunction;
\end{hide}

public class PascalDist extends NegativeBinomialDist\begin{hide} {
   private static final double EPSI = 1.0E-10;

   private static class Function implements MathFunction {
      protected int m;
      protected int max;
      protected double mean;
      protected int[] Fj;

      public Function (int m, int max, double mean, int[] Fj) {
         this.m = m;
         this.max = max;
         this.mean = mean;
         this.Fj = new int[Fj.length];
         System.arraycopy(Fj, 0, this.Fj, 0, Fj.length);
      }

      public double evaluate (double p) {
         double sum = 0.0;
         double s = (p * mean) / (1.0 - p);

         for (int j = 0; j < max; j++)
            sum += Fj[j] / (s + (double) j);

         return sum + m * Math.log (p);
      }

      public double evaluateN (int n, double p) {
         double sum = 0.0;

         for (int j = 0; j < max; j++)
            sum += Fj[j] / (n + j);

         return sum + m * Math.log (p);
      }
   }
\end{hide}
\end{code}

%%%%%%%%%%%%%%%%%%%%%%%%%%%%%%%%%%%%%%%%%
\subsubsection* {Constructor}
\begin{code}

   public PascalDist (int n, double p)\begin{hide} {
      setParams (n, p);
   }\end{hide}
\end{code}
 \begin{tabb}
   Creates an object that contains the probability
   terms (\ref{eq:fmass-pascal}) and the distribution function for
   the Pascal distribution with parameter $n$ and $p$.
 \end{tabb}

%%%%%%%%%%%%%%%%%%%%%%%%%%%%%%%%%%%
\subsubsection* {Methods}
\begin{code}

   public static double[] getMLE (int[] x, int m)\begin{hide} {
      if (m <= 0)
         throw new IllegalArgumentException ("m <= 0");

      double sum = 0.0;
      int max = Integer.MIN_VALUE;
      for (int i = 0; i < m; i++)
      {
         sum += x[i];
         if (x[i] > max)
            max = x[i];
      }
      double mean = (double) sum / (double) m;

      double var = 0.0;
      for (int i = 0; i < m; i++)
         var += (x[i] - mean) * (x[i] - mean);
      var /= (double) m;

      if (mean >= var)
           throw new UnsupportedOperationException("mean >= variance");

      int[] Fj = new int[max];
      for (int j = 0; j < max; j++) {
         int prop = 0;
         for (int i = 0; i < m; i++)
            if (x[i] > j)
               prop++;

         Fj[j] = prop;
      }

      double[] parameters = new double[2];
      Function f = new Function (m, max, mean, Fj);

      parameters[1] = RootFinder.brentDekker (EPSI, 1 - EPSI, f, 1e-5);
      if (parameters[1] >= 1.0)
         parameters[1] = 1.0 - 1e-15;

      parameters[0] = Math.round ((parameters[1] * mean) / (1.0 - parameters[1]));
      if (parameters[0] == 0)
          parameters[0] = 1;

      return parameters;
   }\end{hide}
\end{code}
\begin{tabb}
   Estimates the parameter $(n, p)$ of the Pascal distribution
   using the maximum likelihood method, from the $m$ observations
   $x[i]$, $i = 0, 1, \ldots, m-1$. The estimates are returned in a two-element
    array, in regular order: [$n$, $p$].
   \begin{detailed}
   The maximum likelihood estimators are the values $(\hat{n}$, $\hat{p})$
  that  satisfy the equations
   \begin{eqnarray*}
      \frac{\hat{n}(1 - \hat{p})}{\hat{p}} & = & \bar{x}_m\\
      \ln (1 + \hat{p}) & = & \sum_{j=1}^{\infty} \frac{F_j}{(\hat{n} + j - 1)}
   \end{eqnarray*}
   where  $\bar x_m$ is the average of $x[0],\dots,x[m-1]$, and
   $F_j = \sum_{i=j}^{\infty} f_i$ = proportion of $x$'s which are
   greater than or equal to $j$
   \cite[page 132]{tJOH69a}.
   \end{detailed}
\end{tabb}
\begin{htmlonly}
   \param{x}{the list of observations used to evaluate parameters}
   \param{m}{the number of observations used to evaluate parameters}
   \return{returns the parameters [$\hat{n}$, $\hat{p}$]}
\end{htmlonly}
\begin{code}

   public static PascalDist getInstanceFromMLE (int[] x, int m)\begin{hide} {
      double parameters[] = getMLE (x, m);
      return new PascalDist ((int) parameters[0], parameters[1]);
   }\end{hide}
\end{code}
\begin{tabb}
   Creates a new instance of a Pascal distribution with parameters $n$ and
   $p$ estimated using the maximum likelihood method based on the $m$
   observations $x[i]$, $i = 0, 1, \ldots, m-1$.
\end{tabb}
\begin{htmlonly}
   \param{x}{the list of observations to use to evaluate parameters}
   \param{m}{the number of observations to use to evaluate parameters}
\end{htmlonly}
\begin{code}

   public int getN1()\begin{hide} {
      return (int) (n + 0.5);
   }
   \end{hide}
\end{code}
\begin{tabb} Returns the parameter $n$ of this object.
\end{tabb}
\begin{code}

   public void setParams (int n, double p)\begin{hide} {
      super.setParams ((double) n, p);
   }\end{hide}
\end{code}
\begin{tabb} Sets the parameter $n$ and $p$ of this object.
\end{tabb}

\begin{hide}\begin{code}

   public String toString ()\begin{hide} {
      return getClass().getSimpleName() + " : n = " + getN1() + ", p = " + getP();
   }\end{hide}
\end{code}
\begin{tabb}
   Returns a \texttt{String} containing information about the current distribution.
\end{tabb}\end{hide}
\begin{code}\begin{hide}
}\end{hide}
\end{code}
