\defmodule {CramerVonMisesDist}

Extends the class \class{ContinuousDistribution} for the
Cram\'er-von Mises distribution (see \cite{tDUR73a,tSTE70a,tSTE86b}).
Given a sample of $n$ independent uniforms $U_i$ over $[0,1]$,
the Cram\'er-von Mises statistic $W_n^2$  is defined by
\begin{htmlonly}
  \begin {equation}
     W_n^2 = {1/12n} +
            \sum_{j=1}^n \left(U_{(j)} - {(j-0.5) / n}\right)^2,
  \end {equation}
  where the $U_{(j)}$ are the $U_i$ sorted in increasing order.
  The  distribution function (the cumulative probabilities)
  is defined as $F_n(x) = P[W_n^2 \le x]$.
\end{htmlonly}%
\begin{latexonly}%
  \begin {equation}
     W_n^2 = \frac{1}{12n} +
            \sum_{j=1}^n \left(U_{(j)} - \frac{(j-0.5)}{n}\right)^2,
                                                   \eqlabel {eq:CraMis}
  \end {equation}
  where the $U_{(j)}$ are the $U_i$ sorted in increasing order.
  The  distribution function (the cumulative probabilities)
  is defined as $F_n(x) = P[W_n^2 \le x]$.
\end{latexonly}%

\bigskip\hrule

%%%%%%%%%%%%%%%%%%%%%%%%%%%%%%%%%%%%%%%%
\begin{code}
\begin{hide}
/*
 * Class:        CramerVonMisesDist
 * Description:  Cramér-von Mises distribution
 * Environment:  Java
 * Software:     SSJ 
 * Copyright (C) 2001  Pierre L'Ecuyer and Universite de Montreal
 * Organization: DIRO, Universite de Montreal
 * @author       
 * @since

 * SSJ is free software: you can redistribute it and/or modify it under
 * the terms of the GNU General Public License (GPL) as published by the
 * Free Software Foundation, either version 3 of the License, or
 * any later version.

 * SSJ is distributed in the hope that it will be useful,
 * but WITHOUT ANY WARRANTY; without even the implied warranty of
 * MERCHANTABILITY or FITNESS FOR A PARTICULAR PURPOSE.  See the
 * GNU General Public License for more details.

 * A copy of the GNU General Public License is available at
   <a href="http://www.gnu.org/licenses">GPL licence site</a>.
 */
\end{hide}
package umontreal.iro.lecuyer.probdist;
\begin{hide}
import umontreal.iro.lecuyer.util.*;
import umontreal.iro.lecuyer.functions.MathFunction;
\end{hide} 

public class CramerVonMisesDist extends ContinuousDistribution\begin{hide} {
   protected int n;

   private static class Function implements MathFunction {
      protected int n;
      protected double u;

      public Function (int n, double u) {
         this.n = n;
         this.u = u;
      }

      public double evaluate (double x) {
         return u - cdf(n,x);
      }
   }
\end{hide}
\end{code}
%%%%%%%%%%%%%%%%%%%%%%%%%%%%%%%%%%%%%%%%%
\subsubsection* {Constructor}

\begin{code}

   public CramerVonMisesDist (int n)\begin{hide} {
      setN (n);
   }\end{hide}
\end{code}
\begin{tabb}
 Constructs a {\em Cram\'er-von Mises\/} distribution for a sample of size $n$.
\end{tabb}

%%%%%%%%%%%%%%%%%%%%%%%%%%%%%%%%%%%
\subsubsection* {Methods}

\begin{code}\begin{hide}

   public double density (double x) {
      return density (n, x);
   }

   public double cdf (double x) {
      return cdf (n, x);
   }

   public double barF (double x) {
      return barF (n, x);
   }

   public double inverseF (double u) {
      return inverseF (n, u);
   }

   public double getMean() {
      return CramerVonMisesDist.getMean (n);
   }

   public double getVariance() {
      return CramerVonMisesDist.getVariance (n);
   }

   public double getStandardDeviation() {
      return CramerVonMisesDist.getStandardDeviation (n);
   }\end{hide}

   public static double density (int n, double x)\begin{hide} {
      if (n <= 0)
        throw new IllegalArgumentException ("n <= 0");

      if (x <= 1.0/(12.0*n) || x >= n/3.0)
         return 0.0;

      if (n == 1)
         return 1.0 / Math.sqrt (x - 1.0/12.0);

      if (x <= 0.002 || x > 3.95)
         return 0.0;

      throw new UnsupportedOperationException("density not implemented.");
   }\end{hide}
\end{code}
\begin{tabb} Computes the density function
  for a {\em Cram\'er-von Mises\/} distribution with parameter $n$.
\end{tabb}
\begin{code}

   public static double cdf (int n, double x)\begin{hide} {
      if (n <= 0)
         throw new IllegalArgumentException ("n <= 0");

      if (n == 1) {
         if (x <= 1.0/12.0)
            return 0.0;
         if (x >= 1.0/3.0)
            return 1.0;
         return 2.0*Math.sqrt (x - 1.0/12.0);
      }

      if (x <= 1.0/(12.0*n))
         return 0.0;

      if (x <= (n + 3.0)/(12.0*n*n)) {
         double t = Num.lnFactorial(n) - Num.lnGamma (1.0 + 0.5*n) +
            0.5*n*Math.log (Math.PI*(x - 1.0/(12.0*n)));
         return Math.exp(t);
      }

      if (x <= 0.002)
         return 0.0;
      if (x > 3.95 || x >= n/3.0)
         return 1.0;

      final double EPSILON = Num.DBL_EPSILON;
      final int JMAX = 20;
      int j = 0;
      double Cor, Res, arg;
      double termX, termS, termJ;

      termX = 0.0625/x;            // 1 / (16x)
      Res = 0.0;

      final double A[] = {
         1.0,
         1.11803398875,
         1.125,
         1.12673477358,
         1.1274116945,
         1.12774323743,
         1.1279296875,
         1.12804477649,
         1.12812074678,
         1.12817350091
      };

      do {
         termJ = 4*j + 1;
         arg = termJ*termJ*termX;
         termS = A[j]*Math.exp (-arg)*Num.besselK025 (arg);
         Res += termS;
         ++j;
      } while (!(termS < EPSILON || j > JMAX));

      if (j > JMAX)
         System.err.println ("cramerVonMises: iterations have not converged");
      Res /= Math.PI*Math.sqrt (x);

      // Empirical correction in 1/n
      if (x < 0.0092)
         Cor = 0.0;
      else if (x < 0.03)
         Cor = -0.0121763 + x*(2.56672 - 132.571*x);
      else if (x < 0.06)
         Cor = 0.108688 + x*(-7.14677 + 58.0662*x);
      else if (x < 0.19)
         Cor = -0.0539444 + x*(-2.22024 + x*(25.0407 - 64.9233*x));
      else if (x < 0.5)
         Cor = -0.251455 + x*(2.46087 + x*(-8.92836 + x*(14.0988 -
                  x*(5.5204 + 4.61784*x))));
      else if (x <= 1.1)
         Cor = 0.0782122 + x*(-0.519924 + x*(1.75148 +
               x*(-2.72035 + x*(1.94487 - 0.524911*x))));
      else
         Cor = Math.exp (-0.244889 - 4.26506*x);

      Res += Cor/n;
      // This empirical correction is not very precise, so ...
      if (Res <= 1.0)
         return Res;
      else
         return 1.0;
   }\end{hide}
\end{code}
\begin{tabb}
  Computes the Cram\'er-von Mises distribution function with parameter $n$.
  Returns an approximation of $P[W_n^2 \le x]$, where $W_n^2$ is the
  Cram\'er von Mises  statistic (see \cite{tSTE70a,tSTE86b,tAND52a,tKNO74a}).
  The approximation is based on the distribution function of $W^2 =
  \lim_{n\to\infty} W_n^2$, which has the following series expansion derived
  by Anderson and Darling \cite{tAND52a}:
   $$ \qquad
   P(W^2 \le x)  \ = \ \frac1{\pi\sqrt x} \sum_{j=0}^\infty (-1)^j 
   \binom{-1/2}{j} \sqrt{4j+1}\;\;
    {\rm exp}\left\{-\frac{(4j+1)^2}{16 x}\right\}
    {\rm K}_{1/4}\left(\frac{(4j+1)^2}{16 x}\right),
   $$
  where ${\rm K}_{\nu}$ is the  modified Bessel function of the 
  second kind.
  To correct for the deviation between $P(W_n^2\le x)$ and $P(W^2\le x)$,
  we add a correction in $1/n$, obtained empirically by simulation.
  For $n = 10$, 20, 40, the error is less than
  0.002, 0.001, and 0.0005, respectively, while for
  $n \ge 100$ it is less than 0.0005.
  For $n \to\infty$, we estimate that the method returns
  at least 6 decimal digits of precision.
  For $n = 1$, the method uses the exact distribution:
  $P(W_1^2 \le x) = 2 \sqrt {x - 1/12}$ for $1/12 \le x \le 1/3$.
 \end{tabb}
\begin{code}

   public static double barF (int n, double x)\begin{hide} {
      return 1.0 - cdf(n,x);
   }\end{hide}
\end{code}
\begin{tabb}
  Computes the complementary distribution function $\bar F_n(x)$
  with parameter $n$.
\end{tabb}
\begin{code}

   public static double inverseF (int n, double u)\begin{hide} {
      if (n <= 0)
         throw new IllegalArgumentException ("n <= 0");
      if (u < 0.0 || u > 1.0)
         throw new IllegalArgumentException ("u must be in [0,1]");

      if (u >= 1.0)
         return n/3.0;
      if (u <= 0.0)
         return 1.0/(12.0*n);

      if (n == 1)
         return 1.0/12.0 + 0.25*u*u;

      Function f = new Function (n,u);
      return RootFinder.brentDekker (0.0, 10.0, f, 1e-6);
   }\end{hide}
\end{code}
\begin{tabb}
  Computes $x = F_n^{-1}(u)$, where $F_n$ is the 
  {\em Cram\'er-von Mises\/} distribution with parameter $n$.
\end{tabb}
\begin{code}

   public static double getMean (int n)\begin{hide} {
      return 1.0 / 6.0;
   }\end{hide}
\end{code}
\begin{tabb} Returns the mean of the distribution with parameter $n$.
\end{tabb}
\begin{htmlonly}
   \return{the mean}
\end{htmlonly}
\begin{code}

   public static double getVariance (int n)\begin{hide} {
      return (4.0 * n - 3.0) / (180.0 * n ); 
   }\end{hide}
\end{code}
\begin{tabb} Returns the variance of the distribution with parameter $n$.
\end{tabb}
\begin{htmlonly}
   \return{variance}
\end{htmlonly}
\begin{code}

   public static double getStandardDeviation (int n)\begin{hide} {
     return Math.sqrt(getVariance(n));
   }\end{hide}
\end{code}
\begin{tabb} Returns the standard deviation of the distribution with
   parameter $n$.
\end{tabb}
\begin{htmlonly}
   \return{the standard deviation}
\end{htmlonly}
\begin{code}

   public int getN()\begin{hide} {
      return n;
   }\end{hide}
\end{code}
 \begin{tabb} Returns the parameter $n$ of this object.
 \end{tabb}
\begin{code}

   public void setN (int n)\begin{hide} {
      if (n <= 0)
         throw new IllegalArgumentException ("n <= 0");
      this.n = n;
      supportA = 1.0/(12.0*n);
      supportB = n/3.0;
   }\end{hide}
\end{code}
 \begin{tabb} Sets the parameter $n$ of this object.
 \end{tabb}
 \begin{code}

   public double[] getParams ()\begin{hide} {
      double[] retour = {n};
      return retour;
   }\end{hide}
\end{code}
\begin{tabb}
   Return an array containing the parameter $n$ of this object.
\end{tabb}
\begin{hide}\begin{code}

   public String toString ()\begin{hide} {
      return getClass().getSimpleName() + " : n = " + n;
   }\end{hide}
\end{code}
\begin{tabb}
   Returns a \texttt{String} containing information about the current distribution.
\end{tabb}\end{hide}
\begin{code}\begin{hide}
}\end{hide}
\end{code}
