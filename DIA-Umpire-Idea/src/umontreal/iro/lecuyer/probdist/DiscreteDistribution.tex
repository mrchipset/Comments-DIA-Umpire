\defmodule {DiscreteDistribution}

This class implements discrete distributions over a \emph{finite set of real numbers}
(also over \emph{integers} as a particular case).
% For discrete distributions over integers,
% see \externalclass{umontreal.iro.lecuyer.probdist}{DiscreteDistributionInt}.
%
We assume that the random variable $X$ of interest can take one of the
$n$ values $x_0 < \cdots < x_{n-1}$, which  \emph{must be sorted} by
increasing order.
$X$ can take the value $x_k$ with probability $p_k = P[X = x_k]$.
In addition to the methods specified in the interface
\externalclass{umontreal.iro.lecuyer.probdist}{Distribution},
a method that returns the probability $p_k$ is supplied.

% Note that the default implementation of the complementary distribution function
% returns \texttt{1.0 - cdf(x - 1)}, which is not accurate when $F(x)$ is near 1.


\bigskip\hrule

\begin{code}
\begin{hide}
/*
 * Class:        DiscreteDistribution
 * Description:  discrete distributions over a set of real numbers
 * Environment:  Java
 * Software:     SSJ
 * Copyright (C) 2001  Pierre L'Ecuyer and Universite de Montreal
 * Organization: DIRO, Universite de Montreal
 * @author
 * @since

 * SSJ is free software: you can redistribute it and/or modify it under
 * the terms of the GNU General Public License (GPL) as published by the
 * Free Software Foundation, either version 3 of the License, or
 * any later version.

 * SSJ is distributed in the hope that it will be useful,
 * but WITHOUT ANY WARRANTY; without even the implied warranty of
 * MERCHANTABILITY or FITNESS FOR A PARTICULAR PURPOSE.  See the
 * GNU General Public License for more details.

 * A copy of the GNU General Public License is available at
   <a href="http://www.gnu.org/licenses">GPL licence site</a>.
 */
\end{hide}
package umontreal.iro.lecuyer.probdist;\begin{hide}

import java.util.Formatter;
import java.util.Locale;\end{hide}


public class DiscreteDistribution implements Distribution\begin{hide} {
  /*
     For better precision in the tails, we keep the cumulative probabilities
     (F) in cdf[x] for x <= xmed (i.e. cdf[x] is the sum off all the probabi-
     lities pr[i] for i <= x),
     and the complementary cumulative probabilities (1 - F) in cdf[x] for
     x > xmed (i.e. cdf[x] is the sum off all the probabilities pr[i]
     for i >= x).
  */

   protected double cdf[] = null;    // cumulative probabilities
   protected double pr[] = null;     // probability terms or mass distribution
   protected int xmin = 0;           // pr[x] = 0 for x < xmin
   protected int xmax = 0;           // pr[x] = 0 for x > xmax
   protected int xmed = 0;           // cdf[x] = F(x) for x <= xmed, and
                                     // cdf[x] = bar_F(x) for x > xmed
   protected int nVal;               // number of different values
   protected double sortedVal[];
   protected double supportA = Double.NEGATIVE_INFINITY;
   protected double supportB = Double.POSITIVE_INFINITY;
\end{hide}
\end{code}


%%%%%%%%%%%%%%%%%%%%%%%%%%%%%%%%%%%%%%%%%
\subsubsection* {Constructors}

\begin{code}\begin{hide}

   protected DiscreteDistribution () {}
   // Default constructor called by subclasses such as 'EmpiricalDist'
\end{hide}

   public DiscreteDistribution (double[] values, double[] prob, int n)\begin{hide} {
      init(n, values, prob);
   }\end{hide}
\end{code}
\begin{tabb} Constructs a discrete distribution over the $n$ values
 contained in array \texttt{values}, with probabilities given in array \texttt{prob}.
 Both arrays must have at least $n$ elements, the probabilities must
 sum to 1, and the values are assumed to be sorted by increasing order.
\end{tabb}
\begin{code}

   public DiscreteDistribution (int[] values, double[] prob, int n)\begin{hide} {
      double[] A = new double[n];
      for(int i=0; i<n; i++)
         A[i] = values[i];
      init(n, A, prob);
   }\end{hide}
\end{code}
\begin{tabb} Similar to
\method{DiscreteDistribution}{double[], double[], int}\texttt{(double[], double[], int)}.
\end{tabb}
\begin{code}

   @Deprecated
   public DiscreteDistribution (double[] params)\begin{hide} {
      if (params.length != 1+params[0]*2)
         throw new IllegalArgumentException("Wrong parameter size");

      int n =  (int)params[0];
      double[] val = new double[n];
      double[] prob = new double[n];

      //int indice = 1;
      System.arraycopy (params, 1, val, 0, n);
      System.arraycopy (params, n+1, prob, 0, n);
      init(n, val, prob);
    }\end{hide}
\end{code}
\begin{tabb}
   Constructs a discrete distribution whose parameters are given
   in a single ordered array: \texttt{params[0]} contains $n$, the number of
   values to consider. Then the next $n$ values of \texttt{params} are the
   values $x_j$, and the last $n$ values of \texttt{params}
   are the probabilities $p_j$.
\end{tabb}
\begin{hide}\begin{code}

   private void init(int n, double[] val, double[] prob) {
      int no = val.length;
      int np = prob.length;
      if (n <= 0)
         throw new IllegalArgumentException ("n <= 0");
      if (no < n || np < n)
         throw new IllegalArgumentException
         ("Size of arrays 'values' or 'prob' less than 'n'");

      nVal = n;
      pr = prob;

      // cdf
      sortedVal = new double[nVal];
      System.arraycopy (val, 0, sortedVal, 0, nVal);

      supportA = sortedVal[0];
      supportB = sortedVal[nVal - 1];
      xmin = 0;
      xmax = nVal - 1;

      /* Compute the cumulative probabilities until F >= 0.5, and keep them in
         the lower part of cdf */
      cdf = new double[nVal];
      cdf[0] = pr[0];
      int i = 0;
      while (i < xmax && cdf[i] < 0.5) {
         i++;
         cdf[i] = pr[i] + cdf[i - 1];
      }
      // This is the boundary between F and barF in the CDF
      xmed = i;

      /* Compute the cumulative probabilities of the complementary
         distribution and keep them in the upper part of cdf. */
      cdf[nVal - 1] = pr[nVal - 1];
      i = nVal - 2;
      while (i > xmed) {
         cdf[i] = pr[i] + cdf[i + 1];
         i--;
      }
}\end{code}\end{hide}


%%%%%%%%%%%%%%%%%%%%%%%%%%%%%%%%%%%
\subsubsection* {Methods}

\begin{hide}
\begin{code}

   public double cdf (double x) {
      if (x < sortedVal[0])
         return 0.0;
      if (x >= sortedVal[nVal-1])
         return 1.0;
      if ((xmax == xmed) || (x < sortedVal[xmed+1])) {
         for (int i = 0; i <= xmed; i++)
            if (x >= sortedVal[i] && x < sortedVal[i+1])
               return cdf[i];
      } else {
         for (int i = xmed + 1; i < nVal-1; i++)
            if (x >= sortedVal[i] && x < sortedVal[i+1])
               return 1.0 - cdf[i+1];
      }
      throw new IllegalStateException();
   }
\end{code}
\begin{htmlonly}
   \param{x}{value at which the distribution function is evaluated}
   \return{the distribution function evaluated at \texttt{x}}
\end{htmlonly}
\begin{code}

   public double barF (double x) {
      if (x <= sortedVal[0])
         return 1.0;
      if (x > sortedVal[nVal-1])
         return 0.0;
      if ((xmax == xmed) || (x <= sortedVal[xmed+1])) {
         for (int i = 0; i <= xmed; i++)
            if (x > sortedVal[i] && x <= sortedVal[i+1])
               return 1.0 - cdf[i];
      } else {
         for (int i = xmed + 1; i < nVal-1; i++)
            if (x > sortedVal[i] && x <= sortedVal[i+1])
               return cdf[i + 1];
      }
      throw new IllegalStateException();
   }
\end{code}
\begin{htmlonly}
   \param{x}{value at which the complementary distribution function is evaluated}
   \return{the complementary distribution function evaluated at \texttt{x}}
\end{htmlonly}
\begin{code}

   public double inverseF (double u) {
      int i, j, k;

      if (u < 0.0 || u > 1.0)
         throw new IllegalArgumentException ("u not in [0,1]");
      if (u <= 0.0)
         return supportA;
      if (u >= 1.0)
         return supportB;

      // Remember: the upper part of cdf contains the complementary distribu-
      // tion for xmed < s <= xmax, and the lower part of cdf the
      // distribution for xmin <= s <= xmed

      if (u <= cdf[xmed - xmin]) {
         // In the lower part of cdf
         if (u <= cdf[0])
            return sortedVal[xmin];
         i = 0;
         j = xmed - xmin;
         while (i < j) {
            k = (i + j) / 2;
            if (u > cdf[k])
               i = k + 1;
            else
               j = k;
         }
      }
      else {
         // In the upper part of cdf
         u = 1 - u;
         if (u < cdf[xmax - xmin])
            return sortedVal[xmax];

         i = xmed - xmin + 1;
         j = xmax - xmin;
         while (i < j) {
            k = (i + j) / 2;
            if (u < cdf[k])
               i = k + 1;
            else
               j = k;
         }
         i--;
      }

      return sortedVal[i + xmin];
   }
\end{code}
\begin{htmlonly}
   \param{u}{value in the interval $(0,1)$ for which
             the inverse distribution function is evaluated}
   \return{the inverse distribution function evaluated at \texttt{u}}
   \exception{IllegalArgumentException}{if $u$ is  not in the interval $(0,1)$}
   \exception{ArithmeticException}{if the inverse cannot be computed,
     for example if it would give infinity in a theoretical context}
\end{htmlonly}
\end{hide}
\begin{code}

   public double getMean()\begin{hide} {
      double mean = 0.0;
      for (int i = 0; i < nVal; i++)
         mean += sortedVal[i] * pr[i];
      return mean;
   }\end{hide}
\end{code}
\begin{tabb}
   Computes the mean $E[X] = \sum_{i}^{} p_i x_i$ of the distribution.
\end{tabb}
\begin{code}

   public double getVariance()\begin{hide} {
      double mean = getMean();
      double variance = 0.0;
      for (int i = 0; i < nVal; i++)
         variance += (sortedVal[i] - mean) * (sortedVal[i] - mean) * pr[i];
      return (variance);
   }\end{hide}
\end{code}
\begin{tabb}
   Computes the variance $\mbox{Var}[X] = \sum_{i}^{} p_i (x_i - E[X])^2$
   of the distribution.
\end{tabb}
\begin{code}

   public double getStandardDeviation()\begin{hide} {
      return Math.sqrt (getVariance());
   }\end{hide}
\end{code}
\begin{tabb}
   Computes the standard deviation of the distribution.
\end{tabb}
\begin{code}

   public double[] getParams()\begin{hide} {
      double[] retour = new double[1+nVal*2];
      double sum = 0;
      retour[0] = nVal;
      System.arraycopy (sortedVal, 0, retour, 1, nVal);
      for(int i = 0; i<nVal-1; i++) {
         retour[nVal+1+i] = cdf[i] - sum;
         sum = cdf[i];
      }
      retour[2*nVal] = 1.0 - sum;

      return retour;
   }\end{hide}
\end{code}
\begin{tabb}
   Returns a table containing the parameters of the current distribution.
   This table is built in regular order, according to constructor
   \texttt{DiscreteDistribution(double[] params)} order.
\end{tabb}
\begin{code}

   public int getN()\begin{hide} {
      return nVal;
   }\end{hide}
\end{code}
\begin{tabb} Returns the number of possible values $x_i$.
\end{tabb}
\begin{code}

   public double prob (int i)\begin{hide} {
      if (i < 0 || i >= nVal)
         return 0.;
      return pr[i];
   }\end{hide}
\end{code}
\begin{tabb}  Returns $p_i$, the probability of
  the $i$-th value, for $0\le i<n$.
\end{tabb}
\begin{htmlonly}
   \param{i}{value number, $0\le i < n$}
   \return{the probability of value \texttt{i}}
\end{htmlonly}
\begin{code}

   public double getValue (int i)\begin{hide} {
      return sortedVal[i];
   }\end{hide}
\end{code}
\begin{tabb}
   Returns the $i$-th value $x_i$, for $0\le i<n$.
\end{tabb}
\begin{code}

   public double getXinf()\begin{hide} {
      return supportA;
   }\end{hide}
\end{code}
\begin{tabb} Returns the lower limit $x_0$ of the support of the distribution.
% The probability is 0 for all $x < x_0$.
\end{tabb}
\begin{htmlonly}
   \return{$x$ lower limit of support}
\end{htmlonly}
\begin{code}

   public double getXsup()\begin{hide} {
      return supportB;
   }\end{hide}
\end{code}
\begin{tabb} Returns the upper limit $x_{n-1}$ of the support of the distribution.
% The probability is 0 for all $x > x_{n-1}$.
\end{tabb}
\begin{htmlonly}
   \return{$x$ upper limit of support}
\end{htmlonly}
\begin{code}

   public String toString()\begin{hide} {
      StringBuilder sb = new StringBuilder ();
      Formatter formatter = new Formatter (sb, Locale.US);
      formatter.format ("%s%n", getClass ().getSimpleName ());
      formatter.format ("%s :      %s%n", "value", "cdf");
      for (int i = 0; i < nVal - 1; i++)
         formatter.format ("%f : %f%n", sortedVal[i], cdf[i]);
      formatter.format ("%f : %f%n", sortedVal[nVal-1], 1.0);
      return sb.toString ();
   }\end{hide}
\end{code}
\begin{tabb}
   Returns a \texttt{String} containing information about the current distribution.
\end{tabb}
\begin{code}\begin{hide}
}\end{hide}
\end{code}
