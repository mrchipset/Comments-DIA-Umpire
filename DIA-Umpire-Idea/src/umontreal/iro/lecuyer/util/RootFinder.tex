\defmodule {RootFinder}

This class provides methods to solve non-linear equations.

\bigskip\hrule

\begin{code}
\begin{hide}
/*
 * Class:        RootFinder
 * Description:  Provides methods to solve non-linear equations.
 * Environment:  Java
 * Software:     SSJ
 * Copyright (C) 2001  Pierre L'Ecuyer and Universite de Montreal
 * Organization: DIRO, Universite de Montreal
 * @author
 * @since

 * SSJ is free software: you can redistribute it and/or modify it under
 * the terms of the GNU General Public License (GPL) as published by the
 * Free Software Foundation, either version 3 of the License, or
 * any later version.

 * SSJ is distributed in the hope that it will be useful,
 * but WITHOUT ANY WARRANTY; without even the implied warranty of
 * MERCHANTABILITY or FITNESS FOR A PARTICULAR PURPOSE.  See the
 * GNU General Public License for more details.

 * A copy of the GNU General Public License is available at
   <a href="http://www.gnu.org/licenses">GPL licence site</a>.
 */
\end{hide}
package umontreal.iro.lecuyer.util;
   import umontreal.iro.lecuyer.functions.MathFunction;


public class RootFinder\begin{hide} {
   private static final double MINVAL = 5.0e-308;
   private RootFinder() {}\end{hide}
\end{code}

%%%%%%%%%%%%%%%%%%%%%%%
\subsubsection* {Methods}
\begin{code}

   public static double brentDekker (double a, double b,
                                     MathFunction f, double tol)\begin{hide} {
      final double EPS = 0.5E-15;
      final int MAXITER = 120;    // Maximum number of iterations
      double c, d, e;
      double fa, fb, fc;
      final boolean DEBUG = false;

      // Special case I = [b, a]
      if (b < a) {
         double ctemp = a;
         a = b;
         b = ctemp;
      }

      // Initialization
      fa = f.evaluate (a);
      fb = f.evaluate (b);
      c = a;
      fc = fa;
      d = e = b - a;
      tol += EPS + Num.DBL_EPSILON; // in case tol is too small

      if (Math.abs (fc) < Math.abs (fb)) {
         a = b;
         b = c;
         c = a;
         fa = fb;
         fb = fc;
         fc = fa;
      }

      int i;
      for (i = 0; i < MAXITER; i++) {
         double s, p, q, r;
         double tol1 = tol + 4.0 * Num.DBL_EPSILON * Math.abs (b);
         double xm = 0.5 * (c - b);
         if (DEBUG) {
            double err = Math.abs(fa - fb);
            System.out.printf("[a, b] = [%g, %g]   fa = %g,   fb = %g   |fa - fb| = %.2g%n",
                    a, b, fa, fb, err);
         }

         if (Math.abs (fb) <= MINVAL) {
            return b;
         }
         if (Math.abs (xm) <= tol1) {
            if (Math.abs (b) > MINVAL)
               return b;
            else
               return 0;
         }

         if ((Math.abs (e) >= tol1) && (Math.abs (fa) > Math.abs (fb))) {
            if (a != c) {
               // Inverse quadratic interpolation
               q = fa / fc;
               r = fb / fc;
               s = fb / fa;
               p = s * (2.0 * xm * q * (q - r) - (b - a) * (r - 1.0));
               q = (q - 1.0) * (r - 1.0) * (s - 1.0);
            } else {
               // Linear interpolation
               s = fb / fa;
               p = 2.0 * xm * s;
               q = 1.0 - s;
            }

            // Adjust signs
            if (p > 0.0)
               q = -q;
            p = Math.abs (p);

            // Is interpolation acceptable ?
            if (((2.0 * p) >= (3.0 * xm * q - Math.abs (tol1 * q)))
                  || (p >= Math.abs (0.5 * e * q))) {
               d = xm;
               e = d;
            } else {
               e = d;
               d = p / q;
            }
         } else {
            // Bisection necessary
            d = xm;
            e = d;
         }

         a = b;
         fa = fb;
         if (Math.abs (d) > tol1)
            b += d;
         else if (xm < 0.0)
            b -= tol1;
         else
            b += tol1;
         fb = f.evaluate (b);
         if (fb * (Math.signum (fc)) > 0.0) {
            c = a;
            fc = fa;
            d = e = b - a;
         } else {
            a = b;
            b = c;
            c = a;
            fa = fb;
            fb = fc;
            fc = fa;
         }
      }

      if (i >= MAXITER)
         System.err.println(" WARNING:  root finding does not converge");
      return b;
   }\end{hide}
\end{code}
\begin{tabb} Computes a root $x$ of the function in \texttt{f} using the
    Brent-Dekker method. The interval $[a, b]$ must contain the root $x$.
    The calculations are done with an approximate relative precision
    \texttt{tol}.  Returns $x$ such that $f(x) = 0$.
 \end{tabb}
\begin{htmlonly}
   \param{a}{left endpoint of initial interval}
   \param{b}{right endpoint of initial interval}
   \param{f}{the function which is evaluated}
   \param{tol}{accuracy goal}
   \return{the root $x$}
\end{htmlonly}
\begin{code}

   public static double bisection (double a, double b,
                                   MathFunction f, double tol)\begin{hide} {
      // Case I = [b, a]
      if (b < a) {
         double ctemp = a;
         a = b;
         b = ctemp;
      }
      double xa = a;
      double xb = b;
      double yb = f.evaluate (b);
      double ya = f.evaluate (a);
      double x = 0, y = 0;
      final int MAXITER = 1200;   // Maximum number of iterations
      final boolean DEBUG = false;
      tol += Num.DBL_EPSILON; // in case tol is too small

      if (DEBUG)
         System.out.println
         ("\niter              xa                   xb              f(x)");

      boolean fini = false;
      int i = 0;
      while (!fini) {
         x = (xa + xb) / 2.0;
         y = f.evaluate (x);
         if ((Math.abs (y) <= MINVAL) ||
             (Math.abs (xb - xa) <= tol * Math.abs (x)) ||
             (Math.abs (xb - xa) <= MINVAL)) {
            if (Math.abs(x) > MINVAL)
               return x;
            else
               return 0;
         }
         if (y * ya < 0.0)
            xb = x;
         else
            xa = x;
         ++i;
         if (DEBUG)
            System.out.printf("%3d    %18.12g     %18.12g    %14.4g%n",
                              i, xa, xb, y);
         if (i > MAXITER) {
            System.out.println ("***** bisection:  SEARCH DOES NOT CONVERGE");
            fini = true;
         }
      }
      return x;
   }\end{hide}
\end{code}
\begin{tabb} Computes a root $x$ of the function in \texttt{f} using the
    \emph{bisection} method. The interval $[a, b]$ must contain the root $x$.
    The calculations are done with an approximate relative precision
    \texttt{tol}.  Returns $x$ such that $f(x) = 0$.
 \end{tabb}
\begin{htmlonly}
   \param{a}{left endpoint of initial interval}
   \param{b}{right endpoint of initial interval}
   \param{f}{the function which is evaluated}
   \param{tol}{accuracy goal}
   \return{the root $x$}
\end{htmlonly}

\begin{code}\begin{hide}
}\end{hide}
\end{code}
