\defmodule {Num}

This class provides a few constants and some methods to compute numerical
quantities such as factorials, combinations, gamma functions, and so on.

\bigskip\hrule

\begin{code}
\begin{hide}
/*
 * Class:        Num
 * Description:  Provides methods to compute some special functions
 * Environment:  Java
 * Software:     SSJ
 * Copyright (C) 2001  Pierre L'Ecuyer and Universite de Montreal
 * Organization: DIRO, Universite de Montreal
 * @author
 * @since

 * SSJ is free software: you can redistribute it and/or modify it under
 * the terms of the GNU General Public License (GPL) as published by the
 * Free Software Foundation, either version 3 of the License, or
 * any later version.

 * SSJ is distributed in the hope that it will be useful,
 * but WITHOUT ANY WARRANTY; without even the implied warranty of
 * MERCHANTABILITY or FITNESS FOR A PARTICULAR PURPOSE.  See the
 * GNU General Public License for more details.

 * A copy of the GNU General Public License is available at
   <a href="http://www.gnu.org/licenses">GPL licence site</a>.
 */
\end{hide}
package umontreal.iro.lecuyer.util;
\begin{hide} import cern.jet.math.Bessel;\end{hide}

public class Num\begin{hide} {
   private static final double XBIG = 40.0;
   private static final double SQPI_2 = 0.88622692545275801; // Sqrt(Pi)/2
   private static final double RACPI = 1.77245385090551602729; // sqrt(Pi)
   private static final double LOG_SQPI_2 = -0.1207822376352453; // Ln(Sqrt(Pi)/2)
   private static final double LOG4 = 1.3862943611198906;   // Ln(4)
   private static final double LOG_PI = 1.14472988584940017413; // Ln(Pi)
   private static final double PIsur2 = Math.PI/2.0;

   private static final double UNSIX = 1.0/6.0;
   private static final double QUARAN = 1.0/42.0;
   private static final double UNTRENTE = 1.0 / 30.0;
   private static final double DTIERS = 2.0 / 3.0;
   private static final double CTIERS = 5.0 / 3.0;
   private static final double STIERS = 7.0 / 3.0;
   private static final double QTIERS = 14.0 / 3.0;


   private static final double[] AERF = {
      // used in erf(x)
      1.4831105640848036E0,
      -3.0107107338659494E-1,
      6.8994830689831566E-2,
      -1.3916271264722188E-2,
      2.4207995224334637E-3,
      -3.6586396858480864E-4,
      4.8620984432319048E-5,
      -5.7492565580356848E-6,
      6.1132435784347647E-7,
      -5.8991015312958434E-8,
      5.2070090920686482E-9,
      -4.2329758799655433E-10,
      3.188113506649174974E-11,
      -2.2361550188326843E-12,
      1.46732984799108492E-13,
      -9.044001985381747E-15,
      5.25481371547092E-16
   };

   private static final double[] AERFC = {
      // used in erfc(x)
       6.10143081923200418E-1,
      -4.34841272712577472E-1,
      1.76351193643605501E-1,
      -6.07107956092494149E-2,
      1.77120689956941145E-2,
      -4.32111938556729382E-3,
      8.54216676887098679E-4,
      -1.27155090609162743E-4,
      1.12481672436711895E-5,
      3.13063885421820973E-7,
      -2.70988068537762022E-7,
      3.07376227014076884E-8,
      2.51562038481762294E-9,
      -1.02892992132031913E-9,
      2.99440521199499394E-11,
      2.60517896872669363E-11,
      -2.63483992417196939E-12,
      -6.43404509890636443E-13,
      1.12457401801663447E-13,
      1.7281533389986098E-14,
      -4.2641016949424E-15,
      -5.4537197788E-16,
      1.5869760776E-16,
      2.08998378E-17,
      -0.5900E-17
      };


   private static final double[] AlnGamma = {
      /* Chebyshev coefficients for lnGamma (x + 3), 0 <= x <= 1 In Yudell
         Luke: The special functions and their approximations, Vol. II,
         Academic Press, p. 301, 1969. There is an error in the additive
         constant in the formula: (Ln (2)). */
      0.52854303698223459887,
      0.54987644612141411418,
      0.02073980061613665136,
     -0.00056916770421543842,
      0.00002324587210400169,
     -0.00000113060758570393,
      0.00000006065653098948,
     -0.00000000346284357770,
      0.00000000020624998806,
     -0.00000000001266351116,
      0.00000000000079531007,
     -0.00000000000005082077,
      0.00000000000000329187,
     -0.00000000000000021556,
      0.00000000000000001424,
     -0.00000000000000000095
   };

   private Num() {}\end{hide}
\end{code}

%%%%%%%%%%%%%%%%%%%%%%%
\subsubsection* {Constants}

\begin{code}

   public static final double DBL_EPSILON = 2.2204460492503131e-16;
\end{code}
  \begin{tabb} Difference between 1.0 and the smallest \texttt{double} greater than 1.0.
  \end{tabb}
\begin{code}

   public static final int DBL_MAX_EXP = 1024;
\end{code}
 \begin{tabb} Largest \texttt{int} $x$ such that $2^{x-1}$ is representable
  (approximately) as a \texttt{double}.
 \end{tabb}
\begin{code}

   public static final int DBL_MIN_EXP = -1021;
\end{code}
 \begin{tabb} Smallest \texttt{int} $x$ such that $2^{x-1}$ is representable
  (approximately) as a normalised \texttt{double}.
 \end{tabb}
\begin{code}

   public static final int DBL_MAX_10_EXP = 308;
\end{code}
 \begin{tabb} Largest \texttt{int} $x$ such that $10^x$ is representable
   (approximately) as a \texttt{double}.
 \end{tabb}
\begin{code}

   public static final double DBL_MIN = 2.2250738585072014e-308;
\end{code}
 \begin{tabb} Smallest normalized positive floating-point \texttt{double}.
 \end{tabb}
\begin{code}

   public static final double LN_DBL_MIN = -708.3964185322641;
\end{code}
 \begin{tabb} Natural logarithm of \texttt{DBL\_MIN}.
 \end{tabb}
\begin{code}

   public static final int DBL_DIG = 15;
\end{code}
 \begin{tabb} Number of decimal digits of precision in a \texttt{double}.
 \end{tabb}
\begin{code}

   public static final double EBASE = 2.7182818284590452354;
\end{code}
 \begin{tabb} The constant $e$.
 \end{tabb}
\begin{code}

   public static final double EULER = 0.57721566490153286;
\end{code}
 \begin{tabb} The Euler-Mascheroni constant.
 \end{tabb}
\begin{code}

   public static final double RAC2 = 1.41421356237309504880;
\end{code}
 \begin{tabb} The value of $\sqrt{2}$.
 \end{tabb}
\begin{code}

   public static final double IRAC2 = 0.70710678118654752440;
\end{code}
 \begin{tabb} The value of $1/\sqrt{2}$.
 \end{tabb}
\begin{code}

   public static final double LN2 = 0.69314718055994530941;
\end{code}
 \begin{tabb} The values of $\ln 2$.
 \end{tabb}
\begin{code}

   public static final double ILN2 = 1.44269504088896340737;
\end{code}
 \begin{tabb} The values of $1/\ln 2$.
 \end{tabb}
\begin{code}

   public static final double MAXINTDOUBLE = 9007199254740992.0;
\end{code}
  \begin{tabb} Largest integer $n_0 = 2^{53}$ such that any integer
  $n \le n_0$ is represented  exactly as a \texttt{double}.
  \end{tabb}
\begin{code}

   public static final double MAXTWOEXP = 64;
\end{code}
  \begin{tabb}   Powers of 2 up to \texttt{MAXTWOEXP} are stored exactly
    in the array \texttt{TWOEXP}.
  \end{tabb}
\begin{code}

   public static final double TWOEXP[]\begin{hide} = {
      1.0, 2.0, 4.0, 8.0, 1.6e1, 3.2e1,
      6.4e1, 1.28e2, 2.56e2, 5.12e2, 1.024e3,
      2.048e3, 4.096e3, 8.192e3, 1.6384e4, 3.2768e4,
      6.5536e4, 1.31072e5, 2.62144e5, 5.24288e5,
      1.048576e6, 2.097152e6, 4.194304e6, 8.388608e6,
      1.6777216e7, 3.3554432e7, 6.7108864e7,
      1.34217728e8, 2.68435456e8, 5.36870912e8,
      1.073741824e9, 2.147483648e9, 4.294967296e9,
      8.589934592e9, 1.7179869184e10, 3.4359738368e10,
      6.8719476736e10, 1.37438953472e11, 2.74877906944e11,
      5.49755813888e11, 1.099511627776e12, 2.199023255552e12,
      4.398046511104e12, 8.796093022208e12,
      1.7592186044416e13, 3.5184372088832e13,
      7.0368744177664e13, 1.40737488355328e14,
      2.81474976710656e14, 5.62949953421312e14,
      1.125899906842624e15, 2.251799813685248e15,
      4.503599627370496e15, 9.007199254740992e15,
      1.8014398509481984e16, 3.6028797018963968e16,
      7.2057594037927936e16, 1.44115188075855872e17,
      2.88230376151711744e17, 5.76460752303423488e17,
      1.152921504606846976e18, 2.305843009213693952e18,
      4.611686018427387904e18, 9.223372036854775808e18,
      1.8446744073709551616e19
     };
\end{hide}
\end{code}
 \begin{tabb} Contains the precomputed positive powers of 2.
   One has \texttt{TWOEXP[j]}$ = 2^j$, for $j=0,\dots,64$.
 \end{tabb}
\begin{code}

   public static final double TEN_NEG_POW[]\begin{hide} = {
      1.0, 1.0e-1, 1.0e-2, 1.0e-3, 1.0e-4, 1.0e-5, 1.0e-6, 1.0e-7, 1.0e-8,
      1.0e-9, 1.0e-10, 1.0e-11, 1.0e-12, 1.0e-13, 1.0e-14, 1.0e-15, 1.0e-16
     };
\end{hide}
\end{code}
 \begin{tabb} Contains the precomputed negative powers of 10.
   One has \texttt{TEN\_NEG\_POW[j]}$ = 10^{-j}$, for $j=0,\ldots,16$.
 \end{tabb}


%%%%%%%%%%%%%%%%%%%%%%%%%%
\subsubsection* {Methods}

\begin{code}

   public static int gcd (int x, int y)\begin{hide} {
      if (x < 0) x = -x;
      if (y < 0) y = -y;
      int r;
      while (y != 0) {
         r = x % y;
         x = y;
         y = r;
      }
      return x;
   }\end{hide}
\end{code}
 \begin{tabb} Returns the greatest common divisor (gcd) of $x$ and $y$.
 \end{tabb}
 \begin{htmlonly}
       \param{x}{integer}
       \param{y}{integer}
       \return{the GCD of $x$ and $y$}
 \end{htmlonly}
\begin{code}

   public static long gcd (long x, long y)\begin{hide} {
      if (x < 0) x = -x;
      if (y < 0) y = -y;
      long r;
      while (y != 0) {
         r = x % y;
         x = y;
         y = r;
      }
      return x;
   }\end{hide}
\end{code}
 \begin{tabb} Returns the greatest common divisor (gcd) of $x$ and $y$.
 \end{tabb}
 \begin{htmlonly}
       \param{x}{integer}
       \param{y}{integer}
       \return{the GCD of $x$ and $y$}
 \end{htmlonly}
\begin{code}

   public static double combination (int n, int s)\begin{hide} {
      final int NLIM = 100;      // pour eviter les debordements
      int i;
      if (s == 0 || s == n)
         return 1;
      if (s < 0) {
         System.err.println ("combination:   s < 0");
         return 0;
      }
      if (s > n) {
         System.err.println ("combination:   s > n");
         return 0;
      }
      if (s > (n/2))
         s = n - s;
      if (n <= NLIM) {
         double Res = 1.0;
         int Diff = n - s;
         for (i = 1; i <= s; i++) {
            Res *= (double)(Diff + i)/(double)i;
         }
         return Res;
      }
      else {
         double Res = (lnFactorial (n) - lnFactorial (s))
            - lnFactorial (n - s);
         return Math.exp (Res);
      }
   }
\end{hide}
\end{code}
  \begin{tabb} Returns the \begin{latexonly}value of $\binom{n}{s}$, the
   \end{latexonly} number of different combinations
   of $s$ objects amongst $n$. % Uses an algorithm that prevents overflows
  % (when computing factorials), if possible.
 \end{tabb}
 \begin{htmlonly}
      \param{n}{total number of objects}
      \param{s}{number of chosen objects on a combination}
      \return{the combination of $s$ objects amongst $n$}
 \end{htmlonly}
\begin{code}

   public static double lnCombination (int n, int s)\begin{hide} {
      if (s == 0 || s == n)
         return 0;
      if (s < 0 || s > n)
         return Double.NEGATIVE_INFINITY;

      double res = (lnFactorial (n) - lnFactorial (s)) - lnFactorial (n - s);
      return res;
   }
\end{hide}
\end{code}
  \begin{tabb} Returns the natural logarithm of\begin{latexonly} $\binom{n}{s}$, the
   \end{latexonly} number of different combinations
   of $s$ objects amongst $n$.
 \end{tabb}
 \begin{htmlonly}
      \param{n}{total number of objects}
      \param{s}{number of chosen objects on a combination}
      \return{the natural log of the combination}
 \end{htmlonly}
\begin{code}

   public static double factorial (int n)\begin{hide} {
      if (n < 0)
        throw new IllegalArgumentException ("factorial:   n < 0");
      double T = 1;
      for (int j = 2; j <= n; j++)
         T *= j;
      return T;
   }\end{hide}
\end{code}
 \begin{tabb} Returns the value of\latex{ $n!$}\html{ factorial $n$.}
 \end{tabb}
 \begin{htmlonly}
       \param{n}{the integer for which the factorial must be computed}
       \return{the value of $n!$}
 \end{htmlonly}
\begin{code}

   public static double lnFactorial (int n)\begin{hide} {
      return lnFactorial ((long) n);
   }\end{hide}
\end{code}
 \begin{tabb} Returns the value of\latex{ $\ln (n!)$,} the natural logarithm of
  factorial $n$. Gives 16 decimals of precision
  (relative error $< 0.5\times 10^{-15}$).
 \end{tabb}
 \begin{htmlonly}
       \param{n}{argument of the log-factorial}
       \return{natural logarithm of $n$ factorial}
 \end{htmlonly}
\begin{code}

   public static double lnFactorial (long n)\begin{hide} {
      final int NLIM = 14;

      if (n < 0)
        throw new IllegalArgumentException ("lnFactorial:   n < 0");

      if (n == 0 || n == 1)
         return 0.0;
      if (n <= NLIM) {
         long z = 1;
         long x = 1;
         for (int i = 2; i <= n; i++) {
            ++x;
            z *= x;
         }
         return Math.log (z);
      }
      else {
         double x = (double)(n + 1);
         double y = 1.0/(x*x);
         double z = ((-(5.95238095238E-4*y) + 7.936500793651E-4)*y -
            2.7777777777778E-3)*y + 8.3333333333333E-2;
         z = ((x - 0.5)*Math.log (x) - x) + 9.1893853320467E-1 + z/x;
         return z;
      }
   }\end{hide}
\end{code}
 \begin{tabb} Returns the value of\latex{ $\ln (n!)$,} the natural logarithm of
  factorial $n$. Gives 16 decimals of precision
  (relative error $< 0.5\times 10^{-15}$).
 \end{tabb}
 \begin{htmlonly}
       \param{n}{argument of the log-factorial}
       \return{natural logarithm of $n$ factorial}
 \end{htmlonly}
\begin{code}

   public static double factoPow (int n)\begin{hide} {
      if (n < 0)
        throw new IllegalArgumentException ("factoPow :   n < 0");
      double res = 1.0 / n;
      for (int i = 2; i <= n; i++) {
         res *= (double) i / n;
      }
      return res;
    }\end{hide}
\end{code}
 \begin{tabb} Returns the value of\latex{ $n!/n^n$.}\html{ factorial($n$)$/n^n$.}
 \end{tabb}
 \begin{htmlonly}
       \param{n}{integer}
       \return{the value of $n!/n^n$}
 \end{htmlonly}
\begin{code}

   public static double[][] calcMatStirling (int m, int n)\begin{hide} {
      int i, j, k;
      double[][] M = new double[m+1][n+1];

      for (i = 0; i <= m; i++)
         for (j = 0; j <= n; j++)
            M[i][j] = 0.0;

      M[0][0] = 1.0;
      for (j = 1; j <= n; j++) {
         M[0][j] = 0.0;
         if (j <= m) {
            k = j - 1;
            M[j][j] = 1.0;
         }
         else
            k = m;
         for (i = 1; i <= k; i++)
            M[i][j] = (double)i*M[i][j - 1] + M[i - 1][j - 1];
      }
      return M;
   }\end{hide}
\end{code}
 \begin{tabb} Computes and returns the Stirling numbers of the second kind
\begin{latexonly}
\eq
   M[i,j] = \left\{\begin{matrix}j \\ i\end{matrix}\right\}
     \quad \mbox { for $0\le i\le m$ and $0\le i\le j\le n$}.
                                                        \label{Stirling2}
\endeq
See \cite[Section 1.2.6]{iKNU73a}.
%  See D. E. Knuth, {\em The Art of Computer Programming\/}, vol.~1,
%  second ed., 1973, Section 1.2.6.
  The matrix $M$ is the transpose of Knuth's (1973).
\end{latexonly}
\end{tabb}
\begin{htmlonly}
       \param{m}{number of rows of the allocated matrix}
       \param{n}{number of columns of the allocated matrix}
       \return{the matrix of Stirling numbers}
 \end{htmlonly}
\begin{code}

   public static double log2 (double x)\begin{hide} {
     return ILN2*Math.log (x);
   }\end{hide}
\end{code}
 \begin{tabb} Returns $\log_2 ($\texttt{x}$)$.
 \end{tabb}
 \begin{htmlonly}
       \param{x}{the value for which the logarithm must be computed}
       \return{the value of $\log_2 ($\texttt{x}$)$}
 \end{htmlonly}
\begin{code}

   public static double lnGamma (double x)\begin{hide} {
      if (x <= 0.0)
         throw new IllegalArgumentException ("lnGamma:   x <= 0");
      if (Double.isNaN(x))
         return Double.NaN;
      final double XLIMBIG = 1.0/DBL_EPSILON;
      final double XLIM1 = 18.0;
      final double DK2 = 0.91893853320467274177;     // Ln (sqrt (2 Pi))
      final double DK1 = 0.9574186990510627;
      final int N = 15;              // Degree of Chebyshev polynomial
      double y, z;
      int i, k;

/*
      if (x <= 0.0) {
         double f = (1.0 - x) - Math.floor (1.0 - x);
         return LOG_PI - lnGamma (1.0 - x) - Math.log (Math.sin (Math.PI * f));
      }
*/
      if (x > XLIM1) {
         if (x > XLIMBIG)
            y = 0.0;
         else
            y = 1.0/(x*x);
         z = ((-(5.95238095238E-4*y) + 7.936500793651E-4)*y -
            2.7777777777778E-3)*y + 8.3333333333333E-2;
         z = ((x - 0.5)*Math.log (x) - x) + DK2 + z/x;
         return z;

      } else if (x > 4.0) {
         k = (int)x;
         z = x - k;
         y = 1.0;
         for (i = 3; i < k; i++)
            y *= z + i;
         y = Math.log (y);

      } else if (x <= 0.0)
         return Double.MAX_VALUE;

      else if (x < 3.0) {
         k = (int)x;
         z = x - k;
         y = 1.0;
         for (i = 2; i >= k; i--)
            y *= z + i;
         y = -Math.log (y);
      }
      else {           // 3 <= x <= 4
         z = x - 3.0;
         y = 0.0;
      }
      z = evalCheby (AlnGamma, N, 2.0*z - 1.0);
      return z + DK1 + y;
   }\end{hide}
\end{code}
  \begin{tabb} Returns the natural logarithm of the gamma function $\Gamma(x)$
   evaluated at \texttt{x}.
   Gives 16 decimals of precision, but is implemented only for $x>0$.
  \end{tabb}
  \begin{htmlonly}
       \param{x}{the value for which the lnGamma function must be computed}
       \return{the natural logarithm of the gamma function}
 \end{htmlonly}
\begin{code}

   public static double lnBeta (double lam, double nu)\begin{hide} {
      if (0. == lam || 0. == nu)
         return Double.POSITIVE_INFINITY;
      if (Double.isInfinite (lam) || Double.isInfinite (nu))
         return Double.NEGATIVE_INFINITY;
      return lnGamma (lam) + lnGamma (nu) - lnGamma (lam + nu);
   }\end{hide}
\end{code}
\begin{tabb} Computes the natural logarithm of the Beta function
 $B(\lambda, \nu)$.  It is defined in terms of the Gamma function as
 $$
  B(\lambda, \nu) = \frac{\Gamma(\lambda)\Gamma(\nu)}{\Gamma(\lambda + \nu)}
 $$
 with \texttt{lam} $=\lambda$ and  \texttt{nu} $=\nu$.
\end{tabb}
\begin{code}

   public static double digamma (double x)\begin{hide} {
      final double C7[][] = {
       {1.3524999667726346383e4, 4.5285601699547289655e4, 4.5135168469736662555e4,
        1.8529011818582610168e4, 3.3291525149406935532e3, 2.4068032474357201831e2,
        5.1577892000139084710, 6.2283506918984745826e-3},
       {6.9389111753763444376e-7, 1.9768574263046736421e4, 4.1255160835353832333e4,
          2.9390287119932681918e4, 9.0819666074855170271e3,
          1.2447477785670856039e3, 6.7429129516378593773e1, 1.0}
      };
      final double C4[][] = {
       {-2.728175751315296783e-15, -6.481571237661965099e-1, -4.486165439180193579,
        -7.016772277667586642, -2.129404451310105168},
       {7.777885485229616042, 5.461177381032150702e1,
        8.929207004818613702e1, 3.227034937911433614e1, 1.0}
      };

      if (Double.isNaN(x))
         return Double.NaN;
      double prodPj = 0.0;
      double prodQj = 0.0;
      double digX = 0.0;

      if (x >= 3.0) {
         double x2 = 1.0 / (x * x);
         for (int j = 4; j >= 0; j--) {
            prodPj = prodPj * x2 + C4[0][j];
            prodQj = prodQj * x2 + C4[1][j];
         }
         digX = Math.log (x) - (0.5 / x) + (prodPj / prodQj);

      } else if (x >= 0.5) {
         final double X0 = 1.46163214496836234126;
         for (int j = 7; j >= 0; j--) {
            prodPj = x * prodPj + C7[0][j];
            prodQj = x * prodQj + C7[1][j];
         }
         digX = (x - X0) * (prodPj / prodQj);

      } else {
         double f = (1.0 - x) - Math.floor (1.0 - x);
         digX = digamma (1.0 - x) + Math.PI / Math.tan (Math.PI * f);
      }

      return digX;
   }\end{hide}
\end{code}
\begin{tabb}
   Returns the value of the logarithmic derivative of the Gamma function
   $\psi(x) = \Gamma'(x) / \Gamma(x)$.
\end{tabb}
\begin{code}

   public static double trigamma (double x)\begin{hide} {
      double y, sum;
      if (Double.isNaN(x))
         return Double.NaN;

      if (x < 0.5) {
         y = (1.0 - x) - Math.floor (1.0 - x);
         sum = Math.PI / Math.sin (Math.PI * y);
         return  sum * sum - trigamma (1.0 - x);
      }

      if (x >= 40.0) {
         // Asymptotic series
         y = 1.0/(x*x);
         sum = 1.0 + y*(1.0/6.0 - y*(1.0/30.0 - y*(1.0/42.0 - 1.0/30.0*y)));
         sum += 0.5/x;
         return sum/x;
      }

      int i;
      int p = (int) x;
      y = x - p;
      sum = 0.0;

      if (p > 3) {
         for (i = 3; i < p; i++)
            sum -= 1.0/((y + i)*(y + i));

      } else if (p < 3) {
         for (i = 2; i >= p; i--)
            sum += 1.0/((y + i)*(y + i));
      }

      /* Chebyshev coefficients for trigamma (x + 3), 0 <= x <= 1. In Yudell
         Luke: The special functions and their approximations, Vol. II,
         Academic Press, p. 301, 1969. */
      final int N = 15;
      final double A[] = { 2.0*0.33483869791094938576, -0.05518748204873009463,
         0.00451019073601150186, -0.00036570588830372083,
         2.943462746822336e-5, -2.35277681515061e-6, 1.8685317663281e-7,
         -1.475072018379e-8, 1.15799333714e-9, -9.043917904e-11,
         7.029627e-12, -5.4398873e-13, 0.4192525e-13, -3.21903e-15, 0.2463e-15,
        -1.878e-17, 0., 0. };

      return sum + evalChebyStar (A, N, y);
   }\end{hide}
\end{code}
\begin{tabb}
   Returns the value of the trigamma function $d\psi(x)/dx$, the derivative of
   the digamma function, evaluated at $x$.
\end{tabb}
\begin{code}

   public static double tetragamma (double x)\begin{hide} {
      double y, sum;
      if (Double.isNaN(x))
         return Double.NaN;

      if (x < 0.5) {
         y = (1.0 - x) - Math.floor (1.0 - x);
         sum = Math.PI / Math.sin (Math.PI * y);
         return 2.0 * Math.cos (Math.PI * y) * sum * sum * sum +
               tetragamma (1.0 - x);
      }

      if (x >= 20.0) {
         // Asymptotic series
         y = 1.0/(x*x);
         sum = y*(0.5 - y*(1.0/6.0 - y*(1.0/6.0 - y*(0.3 - 5.0/6.0*y))));
         sum += 1.0 + 1.0/x;
         return -sum*y;
      }

      int i;
      int p = (int) x;
      y = x - p;
      sum = 0.0;

      if (p > 3) {
         for (i = 3; i < p; i++)
            sum += 1.0 / ((y + i) * (y + i) * (y + i));

      } else if (p < 3) {
         for (i = 2; i >= p; i--)
            sum -= 1.0 / ((y + i) * (y + i) * (y + i));
      }

      /* Chebyshev coefficients for tetragamma (x + 3), 0 <= x <= 1. In Yudell
         Luke: The special functions and their approximations, Vol. II,
         Academic Press, p. 301, 1969. */
      final int N = 16;
      final double A[] = { -0.11259293534547383037*2.0, 0.03655700174282094137,
         -0.00443594249602728223, 0.00047547585472892648,
         -4.747183638263232e-5, 4.52181523735268e-6, -4.1630007962011e-7,
         3.733899816535e-8, -3.27991447410e-9, 2.8321137682e-10,
         -2.410402848e-11, 2.02629690e-12, -1.6852418e-13, 1.388481e-14,
         -1.13451e-15, 9.201e-17, -7.41e-18, 5.9e-19, -5.0e-20 };

      return 2.0 * sum + evalChebyStar (A, N, y);
   }\end{hide}
\end{code}
\begin{tabb}
   Returns the value of the tetragamma function $d^{2}\psi(x)/d^{2}x$, the second
   derivative of the digamma function, evaluated at $x$.
\end{tabb}
\begin{code}

   public static double gammaRatioHalf (double x)\begin{hide} {
      if (x <= 0.0)
         throw new IllegalArgumentException ("gammaRatioHalf:   x <= 0");
      if (Double.isNaN(x))
         return Double.NaN;

      if (x <= 10.0) {
         double y = lnGamma (x + 0.5) - lnGamma (x);
         return Math.exp(y);
      }

      double sum;
      if (x <= 300.0) {
         // The sum converges very slowly for small x, but faster as x increases
         final double EPSILON = 1.0e-15;
         double term = 1.0;
         sum = 1.0;
         int i = 1;
         while (term > EPSILON*sum) {
            term *= (i - 1.5)*(i - 1.5) /(i*(x + i - 1.5));
            sum += term;
            i++;
         }
         return Math.sqrt ((x - 0.5)*sum);
      }

      // Asymptotic series for Gamma(x + 0.5) / (Gamma(x) * Sqrt(x))
      // Comparer la vitesse de l'asymptotique avec la somme ci-dessus !!!
      double y = 1.0 / (8.0*x);
      sum = 1.0 + y*(-1.0 + y*(0.5 + y*(2.5 - y*(2.625 + 49.875*y))));
      return sum*Math.sqrt(x);
   }\end{hide}
\end{code}
\begin{tabb}
Returns the value of the ratio $\Gamma(x+1/2)/\Gamma(x)$ of two gamma
functions. This ratio is evaluated in a numerically stable way.
Restriction: $x>0$.
\end{tabb}
\begin{code}

   public static double sumKahan (double[] A, int n) \begin{hide} {
      if (A.length < n)
         n = A.length;
      double sum = 0;
      double c = 0;
      double y, t;

      for (int i = 0; i < n; i++) {
        y = A[i] - c;
        t = sum + y;
        c = (t - sum) - y;
        sum = t;
      }

      return sum;
   }\end{hide}
\end{code}
\begin{tabb}
 Implementation of the Kahan summation algorithm.
Sums the first $n$ elements of $A$ and returns the sum.
This algorithm is more precise than the naive algorithm.
See  \url{http://en.wikipedia.org/wiki/Kahan_summation_algorithm}.
\end{tabb}
\begin{code}

   public static double harmonic (long n)\begin{hide} {
      if (n < 1)
         throw new IllegalArgumentException ("n < 1");
      return digamma(n + 1) + EULER;
   }\end{hide}
\end{code}
\begin{tabb} Computes the $n$-th harmonic number $H_n  = \sum_{j=1}^n 1/j$.
\end{tabb}
\begin{code}

   public static double harmonic2 (long n)\begin{hide} {
      if (n <= 0)
         throw new IllegalArgumentException ("n <= 0");
      if (1 == n)
         return 0.0;
      if (2 == n)
         return 1.0;

      long k = n / 2;
      if ((n & 1) == 1)
         return  2.0*harmonic(k);         // n odd
      return  1.0/k + 2.0*harmonic(k-1);  // n even
   }\end{hide}
\end{code}
\begin{tabb} Computes the sum
\[
\sideset{}{'}\sum_{-n/2<j\le n/2}\; \frac 1{|j|},
\]
 where the symbol $\sum^\prime$ means that the term with $j=0$ is excluded
 from the sum.
\end{tabb}
\begin{code}

   public static double volumeSphere (double p, int t)\begin{hide} {
      final double EPS = 2.0*DBL_EPSILON;
      int pLR = (int)p;
      double kLR = (double)t;
      double Vol;
      int s;

      if (p < 0)
         throw new IllegalArgumentException ("volumeSphere:   p < 0");

      if (Math.abs (p - pLR) <= EPS) {
         switch (pLR) {
         case 0:
            return TWOEXP[t];
         case 1:
            return TWOEXP[t]/(double)factorial (t);
         case 2:
            if ((t % 2) == 0)
               return Math.pow (Math.PI, kLR/2.0)/(double)factorial (t/2);
            else {
               s = (t + 1)/2;
               return Math.pow (Math.PI, (double)s - 1.0)*factorial (s)*
                  TWOEXP[2*s]/(double)factorial (2*s);
            }
          default:
         }
      }
      Vol = kLR*(LN2 + lnGamma (1.0 + 1.0/p)) -
      lnGamma (1.0 + kLR/p);
      return Math.exp (Vol);
   }\end{hide}
\end{code}
 \begin{tabb} Returns the volume $V$ of a sphere of radius 1 in $t$ dimensions
  using the norm $L_p$. It is given by the formula
\begin{htmlonly}
\[
   V = ([2\Gamma (1 + 1/p)]^t)/\Gamma(1 + t/p), \qquad p > 0,
\]
\end{htmlonly}
\begin{latexonly}
\[
       V = \frac{\left[2 \Gamma (1 + 1/p)\right]^t}
             {\Gamma\left (1 + t/p\right)}, \qquad p > 0,
\]
\end{latexonly}%
where $\Gamma$ is the gamma function.
  The case of the sup norm $L_\infty$ is  obtained by choosing $p=0$.
  Restrictions: $p\ge 0$ and $t\ge 1$.
  \end{tabb}
  \begin{htmlonly}
       \param{p}{index of the used norm}
       \param{t}{number of dimensions}
       \return{the volume of a sphere}
 \end{htmlonly}
\begin{code}

   public static double bernoulliPoly (int n, double x) \begin{hide} {
      switch (n) {
      case 0:
         return 1.0;
      case 1:
         return x - 0.5;
      case 2:
         return x*(x - 1.0) + UNSIX;
      case 3:
         return ((2.0*x - 3.0) * x + 1.0)*x*0.5;
      case 4:
         return ((x - 2.0) * x + 1.0)*x*x - UNTRENTE;
      case 5:
         return (((x - 2.5) * x + CTIERS) *x*x - UNSIX) * x;
      case 6:
         return (((x - 3.0) * x + 2.5) * x*x - 0.5) * x*x + QUARAN;
      case 7:
         return ((((x - 3.5) * x + 3.5) *x*x - 7.0/6.0) * x*x + UNSIX) * x;
      case 8:
         return ((((x - 4.0) * x +
                  QTIERS) * x*x - STIERS) * x*x + DTIERS) * x*x - UNTRENTE;
      default:
         throw new IllegalArgumentException("n must be <= 8");
      }
    //  return 0;
    }\end{hide}
\end{code}
\begin{tabb} Evaluates the Bernoulli polynomial $B_n(x)$ of degree $n$
  at $x$. Only degrees $n\le 8$ are programmed for now.
 The first Bernoulli polynomials of even degree are:
\begin{eqnarray}
B_0(x) &=& 1 \nonumber \\
B_2(x) &=& x^2-x+1/6 \nonumber \\
B_4(x) &=& x^4-2x^3+x^2-1/30 \label{bernoulli}\\
B_6(x) &=& x^6-3x^5+5x^4/2-x^2/2+1/42 \nonumber \\
B_8(x) &=& x^8-4x^7+14x^6/3 - 7x^4/3 +2x^2/3-1/30. \nonumber
\end{eqnarray}
\end{tabb}
\begin{code}

   public static double evalCheby (double a[], int n, double x) \begin{hide} {
      if (Math.abs (x) > 1.0)
         System.err.println ("Chebychev polynomial evaluated "+
                               "at x outside [-1, 1]");
      final double xx = 2.0*x;
      double b0 = 0.0;
      double b1 = 0.0;
      double b2 = 0.0;
      for (int j = n; j >= 0; j--) {
         b2 = b1;
         b1 = b0;
         b0 = (xx*b1 - b2) + a[j];
      }
      return (b0 - b2)/2.0;
   }\end{hide}
\end{code}
  \begin{tabb} Evaluates a series of Chebyshev polynomials $T_j$ at
  $x$ over the basic interval $[-1, \;1]$\latex{, using}\html{. It uses}
   the method of Clenshaw
  \cite{mCLE62a}, i.e., computes and  returns
  \[
    y = \frac{a_0}2 + \sum_{j=1}^n a_j T_j (x).
  \]
\begin{htmlonly}
       \param{a}{coefficients of the polynomials}
       \param{n}{largest degree of polynomials}
       \param{x}{the parameter of the $T_j$ functions}
       \return{ the value of a series of Chebyshev polynomials $T_j$.}
\end{htmlonly}
  \end{tabb}
\begin{code}

   public static double evalChebyStar (double a[], int n, double x) \begin{hide} {
      if (x > 1.0 || x < 0.0)
         System.err.println ("Shifted Chebychev polynomial evaluated " +
                             "at x outside [0, 1]");
      final double xx = 2.0*(2.0*x - 1.0);
      double b0 = 0.0;
      double b1 = 0.0;
      double b2 = 0.0;
      for (int j = n; j >= 0; j--) {
         b2 = b1;
         b1 = b0;
         b0 = xx*b1 - b2 + a[j];
      }
      return (b0 - b2)/2.0;
   }\end{hide}
\end{code}
  \begin{tabb} Evaluates a series of shifted Chebyshev polynomials $T_j^*$
   at $x$ over the basic interval $ [0, \;1]$\latex{, using}\html{. It uses}
   the method of Clenshaw \cite{mCLE62a}, i.e., computes and  returns
  \[
    y = \frac{a_0}2 + \sum_{j=1}^n a_j T_j^* (x).
  \]
\begin{htmlonly}
       \param{a}{coefficients of the polynomials}
       \param{n}{largest degree of polynomials}
       \param{x}{the parameter of the $T_j^*$ functions}
       \return{ the value of a series of Chebyshev polynomials $T_j^*$.}
\end{htmlonly}
  \end{tabb}
\begin{code}

   public static double besselK025 (double x)\begin{hide} {
      if (Double.isNaN(x))
         return Double.NaN;
      if (x < 1.E-300)
         return Double.MIN_VALUE;

      final int DEG = 6;
      final double c[] = {
         32177591145.0,
         2099336339520.0,
         16281990144000.0,
         34611957596160.0,
         26640289628160.0,
         7901666082816.0,
         755914244096.0
      };

      final double b[] = {
         75293843625.0,
         2891283595200.0,
         18691126272000.0,
         36807140966400.0,
         27348959232000.0,
         7972533043200.0,
         755914244096.0
      };

      /*------------------------------------------------------------------
       * x > 0.6 => approximation asymptotique rationnelle dans Luke:
       * Yudell L.Luke "Mathematical functions and their approximations",
       * Academic Press Inc. New York, 1975, p.371
       *------------------------------------------------------------------*/
      if (x >= 0.6) {
         double B = b[DEG];
         double C = c[DEG];
         for (int j = DEG; j >= 1; j--) {
            B = B * x + b[j - 1];
            C = C * x + c[j - 1];
         }
         double Res1 = Math.sqrt(Math.PI / (2.0*x)) * Math.exp(-x)*(C/B);
         return Res1;
      }

      /*------------------------------------------------------------------
       * x < 0.6 => la serie de K_{1/4} = Pi/Sqrt (2) [I_{-1/4} - I_{1/4}]
       *------------------------------------------------------------------*/
      double xx = x * x;
      double rac = Math.pow (x/2.0, 0.25);
      double Res = (((xx/1386.0 + 1.0/42.0)*xx + 1.0/3.0)*xx + 1.0)/
              (1.225416702465177*rac);
      double temp = (((xx/3510.0 + 1.0/90.0)*xx + 0.2)*xx + 1.0)*rac/
                     0.906402477055477;
      Res = Math.PI*(Res - temp)/RAC2;
      return Res;
   }\end{hide}
\end{code}
  \begin{tabb} Returns the value of $K_{1/4}(x)$, where \begin{latexonly}
 $K_{\nu}$\end{latexonly}\begin{htmlonly}$K_a$ \end{htmlonly} is the modified
  Bessel's function of the second kind.
  The relative error on the returned value is less than
  $0.5\times 10^{-6}$ for $x > 10^{-300}$.
 \end{tabb}
\begin{htmlonly}
       \param{x}{value at which the function is calculated}
       \return{the value of $K_{1/4}(x)$}
\end{htmlonly}
\begin{code}

   public static double expBesselK1 (double x, double y)\begin{hide} {
      if (Double.isNaN(x) || Double.isNaN(y))
         return Double.NaN;
      if (y > 500.0) {
           double sum = 1 + 3.0/(8.0*y) - 15.0 / (128.0*y*y) + 105.0 /(1024.0*y*y*y);
           return Math.sqrt(PIsur2/ y) * sum * Math.exp(x - y);
      } else if (Math.abs(x) > 500.0) {
         double b = Bessel.k1(y);
         return Math.exp(x + Math.log(b));
      } else {
         return Math.exp(x) * Bessel.k1(y);
      }
   }\end{hide}
\end{code}
\begin{tabb}
Returns the value of $e^x K_1(y)$, where $ K_1$ is the modified Bessel
function of the second kind of order 1. Restriction: $y > 0$.
\end{tabb}
\begin{code}

   public static double erf (double x)\begin{hide} {
      if (Double.isNaN(x))
         return Double.NaN;
      if (x < 0.0)
         return -erf(-x);
      if (x >= 6.0)
         return 1.0;
      if (x >= 2.0)
         return 1.0 - erfc(x);

      double t = 0.5*x*x - 1.0;
      double y = Num.evalCheby (AERF, 16, t);
      return x*y;
   }\end{hide}
\end{code}
\begin{tabb}
Returns the value of \texttt{erf}($x$), the error function. It is defined as
\begin{latexonly}
\[
\mbox{erf}(x) = \frac2{\sqrt\pi}\int_0^x dt\, e^{-t^2}.
\]
\end{latexonly}
\begin{htmlonly}
\[
\mbox{erf}(x) = 2/[\sqrt\pi]\int_0^x dt\, e^{-t^2}.
\]
\end{htmlonly}
\begin{htmlonly}
       \param{x}{value at which the function is calculated}
       \return{the value of \texttt{erf}$(x)$}
\end{htmlonly}
\end{tabb}
\begin{code}

   public static double erfc (double x)\begin{hide} {
      if (Double.isNaN(x))
         return Double.NaN;
      if (x < 0.0)
         return 2.0 - erfc(-x);
      if (x >= XBIG)
         return 0.0;
      double t = (x - 3.75)/(x + 3.75);
      double y = Num.evalCheby (AERFC, 24, t);
      y *= Math.exp(-x*x);
      return y;
   }\end{hide}
\end{code}
 \begin{tabb}
Returns the value of \texttt{erfc}($x$), the complementary error function.
It is defined as
\begin{latexonly}
\[
\mbox{erfc}(x) = \frac2{\sqrt\pi}\int_x^\infty dt\, e^{-t^2}.
\]
\end{latexonly}
\begin{htmlonly}
\[
\mbox{erfc}(x) = 2/[\sqrt\pi]\int_x^\infty dt\, e^{-t^2}.
\]
\end{htmlonly}
\end{tabb}
\begin{htmlonly}
       \param{x}{value at which the function is calculated}
       \return{the value of \texttt{erfc}$(x)$}
\end{htmlonly}
\begin{code}\begin{hide}

    private static final double[] InvP1 = {
        0.160304955844066229311e2,
       -0.90784959262960326650e2,
        0.18644914861620987391e3,
       -0.16900142734642382420e3,
        0.6545466284794487048e2,
       -0.864213011587247794e1,
        0.1760587821390590
    };

    private static final double[] InvQ1 = {
        0.147806470715138316110e2,
       -0.91374167024260313396e2,
        0.21015790486205317714e3,
       -0.22210254121855132366e3,
        0.10760453916055123830e3,
       -0.206010730328265443e2,
        0.1e1
    };

    private static final double[] InvP2 = {
       -0.152389263440726128e-1,
        0.3444556924136125216,
       -0.29344398672542478687e1,
        0.11763505705217827302e2,
       -0.22655292823101104193e2,
        0.19121334396580330163e2,
       -0.5478927619598318769e1,
        0.237516689024448000
    };

    private static final double[] InvQ2 = {
      -0.108465169602059954e-1,
       0.2610628885843078511,
      -0.24068318104393757995e1,
       0.10695129973387014469e2,
      -0.23716715521596581025e2,
       0.24640158943917284883e2,
      -0.10014376349783070835e2,
       0.1e1
    };

    private static final double[] InvP3 = {
        0.56451977709864482298e-4,
        0.53504147487893013765e-2,
        0.12969550099727352403,
        0.10426158549298266122e1,
        0.28302677901754489974e1,
        0.26255672879448072726e1,
        0.20789742630174917228e1,
        0.72718806231556811306,
        0.66816807711804989575e-1,
       -0.17791004575111759979e-1,
        0.22419563223346345828e-2
    };

    private static final double[] InvQ3 = {
        0.56451699862760651514e-4,
        0.53505587067930653953e-2,
        0.12986615416911646934,
        0.10542932232626491195e1,
        0.30379331173522206237e1,
        0.37631168536405028901e1,
        0.38782858277042011263e1,
        0.20372431817412177929e1,
        0.1e1
    };\end{hide}

   public static double erfInv (double u) \begin{hide} {
      if (Double.isNaN(u))
         return Double.NaN;
      if (u < 0.0)
         return -erfInv (-u);

      if (u > 1.0)
         throw new IllegalArgumentException ("u is not in [-1, 1]");
      if (u >= 1.0)
         return Double.POSITIVE_INFINITY;

      double t, z, v, w;

      if (u <= 0.75) {
         t = u * u - 0.5625;
         v = Misc.evalPoly (InvP1, 6, t);
         w = Misc.evalPoly (InvQ1, 6, t);
         z = (v / w) * u;

      } else if (u <= 0.9375) {
         t = u * u - 0.87890625;
         v = Misc.evalPoly (InvP2, 7, t);
         w = Misc.evalPoly (InvQ2, 7, t);
         z = (v / w) * u;

      } else {
         t = 1.0 / Math.sqrt (-Math.log (1.0 - u));
         v = Misc.evalPoly (InvP3, 10, t);
         w = Misc.evalPoly (InvQ3, 8, t);
         z = (v / w) / t;
      }

      return z;
   }\end{hide}
\end{code}
\begin{tabb}
Returns the value of \texttt{erf}${}^{-1}(u)$, the inverse of the error
function. If $u =\ $\texttt{erf}$(x)$, then $x =\ $\texttt{erf}${}^{-1}(u)$.
\end{tabb}
\begin{htmlonly}
       \param{u}{value at which the function is calculated}
       \return{the value of {erfInv}$(u)$}
\end{htmlonly}
\begin{code}

   public static double erfcInv (double u)\begin{hide} {
      if (Double.isNaN(u))
         return Double.NaN;
      if (u < 0 || u > 2)
         throw new IllegalArgumentException ("u is not in [0, 2]");

      if (u > 0.005)
         return erfInv (1.0 - u);

      if (u <= 0)
         return Double.POSITIVE_INFINITY;

      double x, w;

      // first term of asymptotic series
      x = Math.sqrt (-Math.log (RACPI * u));
      x = Math.sqrt (-Math.log (RACPI * u * x));

      // Newton's method
      for (int i = 0; i < 3; ++i) {
         w = -2.0*Math.exp(-x * x) / RACPI;
         x += (u - erfc(x))/w;
      }
      return x;
   }\end{hide}
\end{code}
\begin{tabb} Returns the value of \texttt{erfc}${}^{-1}(u)$, the inverse of the
complementary error function. If $u =\ $\texttt{erfc}$(x)$,
then $x =\ $\texttt{erfc}${}^{-1}(u)$.
\end{tabb}
\begin{htmlonly}
       \param{u}{value at which the function is calculated}
       \return{the value of {erfcInv}$(u)$}
\end{htmlonly}

\begin{code}\begin{hide}
}\end{hide}
\end{code}
