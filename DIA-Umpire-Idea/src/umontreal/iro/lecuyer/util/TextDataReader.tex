\defmodule{TextDataReader}

Provides static methods to read data from text files.

\bigskip\hrule
\begin{code}
\begin{hide}
/*
 * Class:        TextDataReader
 * Description:  Provides static methods to read data from text files
 * Environment:  Java
 * Software:     SSJ 
 * Copyright (C) 2001  Pierre L'Ecuyer and Universite de Montreal
 * Organization: DIRO, Universite de Montreal
 * @author       
 * @since

 * SSJ is free software: you can redistribute it and/or modify it under
 * the terms of the GNU General Public License (GPL) as published by the
 * Free Software Foundation, either version 3 of the License, or
 * any later version.

 * SSJ is distributed in the hope that it will be useful,
 * but WITHOUT ANY WARRANTY; without even the implied warranty of
 * MERCHANTABILITY or FITNESS FOR A PARTICULAR PURPOSE.  See the
 * GNU General Public License for more details.

 * A copy of the GNU General Public License is available at
   <a href="http://www.gnu.org/licenses">GPL licence site</a>.
 */
\end{hide}
package umontreal.iro.lecuyer.util;\begin{hide}

import java.io.LineNumberReader;
import java.io.File;
import java.io.InputStreamReader;
import java.io.FileReader;
import java.io.IOException;
import java.io.Reader;
import java.net.URL;
import java.util.logging.Logger;
\end{hide}


public class TextDataReader\begin{hide} {
   private static Logger log = Logger.getLogger ("umontreal.iro.lecuyer.util");
\end{hide}

   public static double[] readDoubleData (Reader input) throws IOException\begin{hide} {
      LineNumberReader inb = new LineNumberReader (input);
      double[] data = new double[5];
      int n = 0;
      String li;
      while ((li = inb.readLine()) != null) {
        li = li.trim();
        if (li.startsWith ("#"))
           continue;

         // look for the first non-digit character on the read line
         int index = 0;
         while (index < li.length() &&
            (li.charAt (index) == '+' || li.charAt (index) == '-' ||
             li.charAt (index) == 'e' || li.charAt (index) == 'E' ||
             li.charAt (index) == '.' || Character.isDigit (li.charAt (index))))
           ++index; 

         // truncate the line
         li = li.substring (0, index);
         if (!li.equals ("")) {
            try {
               data[n++] = Double.parseDouble (li);
               if (n >= data.length) {
                  double[] newData = new double[2*n];
                  System.arraycopy (data, 0, newData, 0, data.length);
                  data = newData;
               }
            }
            catch (NumberFormatException nfe) {
               log.warning ("Invalid line " + inb.getLineNumber() + ": " + li);
            }
         }
      }
      if (data.length != n) {
         double[] data2 = new double[n];
         System.arraycopy (data, 0, data2, 0, n);
         return data2;
      }
      return data;
   }\end{hide}
\end{code}
\begin{tabb}   Reads an array of double-precision values from the
 reader \texttt{input}.
 For each line of text obtained from the
 given reader, this method
 trims whitespaces, and parses the
 remaining text as a double-precision value.
 This method ignores every character
 other than the digits, the plus and minus signs,
 the period (\texttt{.}),
 and the letters \texttt{e} and \texttt{E}.
 Moreover, lines starting with a pound sign (\texttt{\#})
 are considered as comments and thus skipped.
 The method returns an array containing
 all the parsed values.
\end{tabb}
\begin{htmlonly}
   \param{input}{the reader to obtain data from.}
   \return{the obtained array of double-precision values.}
   \exception{IOException}{if an I/O error occurs.}
\end{htmlonly}
\begin{code}

   public static double[] readDoubleData (URL url) throws IOException\begin{hide} {
      Reader reader = new InputStreamReader (url.openStream());
      try {
         return readDoubleData (reader);
      }
      finally {
         reader.close();
      }
   }\end{hide}
\end{code}
\begin{tabb}   Connects to the URL referred to by the URL object \texttt{url},
 and calls \method{readDoubleData}{(Reader)} to
 obtain an array of double-precision values from
 the resource.
\end{tabb}
\begin{htmlonly}
   \param{url}{the URL object representing the resource to read.}
   \return{the obtained array of double-precision values.}
   \exception{IOException}{if an I/O error occurs.}
\end{htmlonly}
\begin{code}

   public static double[] readDoubleData (File file) throws IOException\begin{hide} {
      FileReader reader = new FileReader (file);
      try {
         return readDoubleData (reader);
      }
      finally {
         reader.close();
      }
   }\end{hide}
\end{code}
\begin{tabb}   Opens the file referred to by the file object \texttt{file},
 and calls \method{readDoubleData}{(Reader)} to
 obtain an array of double-precision values from
 the file.
\end{tabb}
\begin{htmlonly}
   \param{file}{the file object representing the file to read.}
   \return{the obtained array of double-precision values.}
   \exception{IOException}{if an I/O error occurs.}
\end{htmlonly}
\begin{code}

   public static double[] readDoubleData (String file) throws IOException\begin{hide} {
      FileReader reader = new FileReader (file);
      try {
         return readDoubleData (reader);
      }
      finally {
         reader.close();
      }
   }\end{hide}
\end{code}
\begin{tabb}   Opens the file with name \texttt{file},
 and calls \method{readDoubleData}{(Reader)} to
 obtain an array of double-precision values from
 the file.
\end{tabb}
\begin{htmlonly}
   \param{file}{the name of the file to read.}
   \return{the obtained array of double-precision values.}
   \exception{IOException}{if an I/O error occurs.}
\end{htmlonly}
\begin{code}

   public static int[] readIntData (Reader input) throws IOException\begin{hide} {
      LineNumberReader inb = new LineNumberReader (input);
      int[] data = new int[5];
      int n = 0;
      String li;
      while ((li = inb.readLine()) != null) {
        li = li.trim();
        if (li.startsWith ("#"))
           continue;

         // look for the first non-digit character on the read line
         int index = 0;
         while (index < li.length() &&
            (li.charAt (index) == '+' || li.charAt (index) == '-' ||
             Character.isDigit (li.charAt (index))))
           ++index; 

         // truncate the line
         li = li.substring (0, index);
         if (!li.equals ("")) {
            try {
               data[n++] = Integer.parseInt (li);
               if (n >= data.length) {
                  int[] newData = new int[2*n];
                  System.arraycopy (data, 0, newData, 0, data.length);
                  data = newData;
               }
            }
            catch (NumberFormatException nfe) {
               log.warning ("Invalid line " + inb.getLineNumber() + ": " + li);
            }
         }
      }
      if (data.length != n) {
         int[] data2 = new int[n];
         System.arraycopy (data, 0, data2, 0, n);
         return data2;
      }
      return data;
   }\end{hide}
\end{code}
\begin{tabb}   This is equivalent to \method{readDoubleData}{(Reader)},
 for reading integers.
\end{tabb}
\begin{htmlonly}
   \param{input}{the reader to obtain data from.}
   \return{the obtained array of integers.}
   \exception{IOException}{if an I/O error occurs.}
\end{htmlonly}
\begin{code}

   public static int[] readIntData (URL url) throws IOException\begin{hide} {
      Reader reader = new InputStreamReader (url.openStream());
      try {
         return readIntData (reader);
      }
      finally {
         reader.close();
      }
   }\end{hide}
\end{code}
\begin{tabb}   Connects to the URL referred to by the URL object \texttt{url},
 and calls \method{readIntData}{(Reader)} to
 obtain an array of integers from
 the resource.
\end{tabb}
\begin{htmlonly}
   \param{url}{the URL object representing the resource to read.}
   \return{the obtained array of integers.}
   \exception{IOException}{if an I/O error occurs.}
\end{htmlonly}
\begin{code}

   public static int[] readIntData (File file) throws IOException\begin{hide} {
      FileReader reader = new FileReader (file);
      try {
         return readIntData (reader);
      }
      finally {
         reader.close();
      }
   }\end{hide}
\end{code}
\begin{tabb}   This is equivalent to \method{readDoubleData}{(File)},
   for reading integers.
\end{tabb}
\begin{htmlonly}
   \param{file}{the file object represented to file to read.}
   \return{the array of integers.}
   \exception{IOException}{if an I/O error occurs.}
\end{htmlonly}
\begin{code}

   public static int[] readIntData (String file) throws IOException\begin{hide} {
      FileReader reader = new FileReader (file);
      try {
         return readIntData (reader);
      }
      finally {
         reader.close();
      }
   }\end{hide}
\end{code}
\begin{tabb}   This is equivalent to \method{readDoubleData}{(String)},
   for reading integers.
\end{tabb}
\begin{htmlonly}
   \param{file}{the name of the file to read.}
   \return{the array of integers.}
   \exception{IOException}{if an I/O error occurs.}
\end{htmlonly}
\begin{code}

   public static String[] readStringData (Reader input) throws IOException\begin{hide} {
      LineNumberReader inb = new LineNumberReader (input);
      String[] data = new String[5];
      int n = 0;
      String li;
      while ((li = inb.readLine()) != null) {
        li = li.trim();
        if (li.startsWith ("#"))
           continue;

        data[n++] = li;
        if (n >= data.length) {
           String[] newData = new String[2*n];
           System.arraycopy (data, 0, newData, 0, data.length);
           data = newData;
        }
      }
      if (data.length != n) {
         String[] data2 = new String[n];
         System.arraycopy (data, 0, data2, 0, n);
         return data2;
      }
      return data;
   }\end{hide}
\end{code}
\begin{tabb}   Reads an array of strings from the
 reader \texttt{input}.
 For each line of text obtained from the
 given reader, this method
 trims leading and trailing whitespaces, and stores the remaining string.
 Lines starting with a pound sign (\texttt{\#})
 are considered as comments and thus skipped.
 The method returns an array containing
 all the read strings.
\end{tabb}
\begin{htmlonly}
   \param{input}{the reader to obtain data from.}
   \return{the obtained array of strings.}
   \exception{IOException}{if an I/O error occurs.}
\end{htmlonly}
\begin{code}

   public static String[] readStringData (URL url) throws IOException\begin{hide} {
      Reader reader = new InputStreamReader (url.openStream());
      try {
         return readStringData (reader);
      }
      finally {
         reader.close();
      }
   }\end{hide}
\end{code}
\begin{tabb}   Connects to the URL referred to by the URL object \texttt{url},
 and calls \method{readStringData}{(Reader)} to
 obtain an array of integers from
 the resource.
\end{tabb}
\begin{htmlonly}
   \param{url}{the URL object representing the resource to read.}
   \return{the obtained array of strings.}
   \exception{IOException}{if an I/O error occurs.}
\end{htmlonly}
\begin{code}

   public static String[] readStringData (File file) throws IOException\begin{hide} {
      FileReader reader = new FileReader (file);
      try {
         return readStringData (reader);
      }
      finally {
         reader.close();
      }
   }\end{hide}
\end{code}
\begin{tabb}   This is equivalent to \method{readDoubleData}{(File)},
   for reading strings.
\end{tabb}
\begin{htmlonly}
   \param{file}{the file object represented to file to read.}
   \return{the array of strings.}
   \exception{IOException}{if an I/O error occurs.}
\end{htmlonly}
\begin{code}

   public static String[] readStringData (String file) throws IOException\begin{hide} {
      FileReader reader = new FileReader (file);
      try {
         return readStringData (reader);
      }
      finally {
         reader.close();
      }
   }\end{hide}
\end{code}
\begin{tabb}   This is equivalent to \method{readDoubleData}{(String)},
   for reading strings.
\end{tabb}
\begin{htmlonly}
   \param{file}{the name of the file to read.}
   \return{the array of strings.}
   \exception{IOException}{if an I/O error occurs.}
\end{htmlonly}
\begin{code}

   public static double[][] readDoubleData2D (Reader input)
                                              throws IOException\begin{hide} {
      LineNumberReader inb = new LineNumberReader (input);
      double[][] data = new double[5][];
      int n = 0;
      String li;
      String number;

      while ((li = inb.readLine()) != null) {
         li = li.trim();
         if (li.startsWith ("#"))
            continue;

         if (li.equals(";")) {
            data[n++] = new double[0];
         }
         else {

            int index = 0;
            int begin = 0;
            boolean end = false;

            double[] row = new double[5];
            int k = 0;

            while (index < li.length() && (! end))
            {
               while (index < li.length() &&
                  (li.charAt (index) == '+' || li.charAt (index) == '-' ||
                   li.charAt (index) == 'e' || li.charAt (index) == 'E' ||
                   li.charAt (index) == '.' || Character.isDigit (li.charAt (index))))
                  ++index;

               if (index >= li.length() || (Character.isWhitespace (li.charAt (index))))
               {
                  number = li.substring (begin, index);
                  begin = ++index;

                  if (! number.equals("")) {
                     try {
                        row[k++] = Double.parseDouble (number);
                        if (k >= row.length) {
                           double[] newRow = new double[2*k];
                           System.arraycopy (row, 0, newRow, 0, row.length);
                           row = newRow;
                        }
                     }
                     catch (NumberFormatException nfe) {
                        log.warning ("Invalid column " + k + " at line " + inb.getLineNumber() + ": " + number);
                     }
                  }
               }
               else {
                  end = true;
               }
            }

            if (k > 0) {
               data[n] = new double[k];
               System.arraycopy (row, 0, data[n], 0, k);
               n++;
            }
            else {
               log.warning ("Invalid line " + inb.getLineNumber() + ": " + li);
            }
         }

         if (n == data.length) {
            double[][] newData = new double[2*n][];
            System.arraycopy (data, 0, newData, 0, n);
            data = newData;
         }
      }

      double[][] data2 = new double[n][];
      System.arraycopy (data, 0, data2, 0, n);
      return data2;
   }\end{hide}
\end{code}
\begin{tabb}   Uses the reader \texttt{input} to obtain
 a 2-dimensional array of double-precision values.
 For each line of text obtained from the
 given reader, this method
 trims whitespaces, and parses the
 remaining text as an array of double-precision values.
 Every character
 other than the digits, the plus (\texttt{+}) and minus (\texttt{-}) signs,
 the period (\texttt{.}),
 and the letters \texttt{e} and \texttt{E} are
 ignored and can be used to separate
 numbers on a line.
 Moreover, lines starting with a pound sign (\texttt{\#})
 are considered as comments and thus skipped. The lines
 containing only a semicolon sign (\texttt{;}) are considered
 as empty lines.
 The method returns a 2D array containing
 all the parsed values.
 The returned array is not always rectangular.
\end{tabb}
\begin{htmlonly}
   \param{input}{the reader to obtain data from.}
   \return{the 2D array of double-precison values.}
   \exception{IOException}{if an I/O error occurs.}
\end{htmlonly}
\begin{code}

   public static double[][] readDoubleData2D (URL url) throws IOException\begin{hide} {
      Reader reader = new InputStreamReader (url.openStream());
      try {
         return readDoubleData2D (reader);
      }
      finally {
         reader.close();
      }
   }\end{hide}
\end{code}
\begin{tabb}   Connects to the URL referred to by the URL object \texttt{url},
 and calls \method{readDoubleData2D}{(Reader)} to
 obtain a matrix of double-precision values from
 the resource.
\end{tabb}
\begin{htmlonly}
   \param{url}{the URL object representing the resource to read.}
   \return{the obtained matrix of double-precision values.}
   \exception{IOException}{if an I/O error occurs.}
\end{htmlonly}
\begin{code}

   public static double[][] readDoubleData2D (File file) throws IOException\begin{hide} {
      FileReader reader = new FileReader (file);
      try {
         return readDoubleData2D (reader);
      }
      finally {
         reader.close();
      }
   }\end{hide}
\end{code}
\begin{tabb}   Opens the file referred to by the file object \texttt{file},
 and calls \method{readDoubleData2D}{(Reader)} to
 obtain a matrix of double-precision values from
 the file.
\end{tabb}
\begin{htmlonly}
   \param{file}{the file object representing the file to read.}
   \return{the obtained matrix of double-precision values.}
   \exception{IOException}{if an I/O error occurs.}
\end{htmlonly}
\begin{code}

   public static double[][] readDoubleData2D (String file)
                                              throws IOException\begin{hide} {
      FileReader reader = new FileReader (file);
      try {
         return readDoubleData2D (reader);
      }
      finally {
         reader.close();
      }
   }\end{hide}
\end{code}
\begin{tabb}   Opens the file with name \texttt{file},
 and calls \method{readDoubleData2D}{(Reader)} to
 obtain a matrix of double-precision values from
 the file.
\end{tabb}
\begin{htmlonly}
   \param{file}{the name of the file to read.}
   \return{the obtained matrix of double-precision values.}
   \exception{IOException}{if an I/O error occurs.}
\end{htmlonly}
\begin{code}

   public static int[][] readIntData2D (Reader input) throws IOException\begin{hide} {
      LineNumberReader inb = new LineNumberReader (input);
      int[][] data = new int[5][];
      int n = 0;
      String li;
      String number;

      while ((li = inb.readLine()) != null) {
         li = li.trim();
         if (li.startsWith ("#"))
            continue;

         if (li.equals(";")) {
            data[n++] = new int[0];
         }
         else {

            int index = 0;
            int begin = 0;
            boolean end = false;

            int[] row = new int[5];
            int k = 0;

            while (index < li.length() && (! end))
            {
               while (index < li.length() &&
                  (li.charAt (index) == '+' || li.charAt (index) == '-' ||
                   Character.isDigit (li.charAt (index))))
                  ++index;

               if (index >= li.length() || (Character.isWhitespace (li.charAt (index))))
               {
                  number = li.substring (begin, index);
                  begin = ++index;

                  if (! number.equals("")) {
                     try {
                        row[k++] = Integer.parseInt (number);
                        if (k >= row.length) {
                           int[] newRow = new int[2*k];
                           System.arraycopy (row, 0, newRow, 0, row.length);
                           row = newRow;
                        }
                     }
                     catch (NumberFormatException nfe) {
                        log.warning ("Invalid column " + k + " at line " + inb.getLineNumber() + ": " + number);
                     }
                  }
               }
               else {
                  end = true;
               }
            }

            if (k > 0) {
               data[n] = new int[k];
               System.arraycopy (row, 0, data[n], 0, k);
               n++;
            }
            else {
               log.warning ("Invalid line " + inb.getLineNumber() + ": " + li);
            }
         }

         if (n == data.length) {
            int[][] newData = new int[2*n][];
            System.arraycopy (data, 0, newData, 0, n);
            data = newData;
         }
      }

      int[][] data2 = new int[n][];
      System.arraycopy (data, 0, data2, 0, n);
      return data2;
   }\end{hide}
\end{code}
\begin{tabb}   This is equivalent to \method{readDoubleData2D}{(Reader)},
 for reading integers.
\end{tabb}
\begin{htmlonly}
   \param{input}{the reader to obtain data from.}
   \return{the obtained 2D array of integers.}
   \exception{IOException}{if an I/O error occurs.}
\end{htmlonly}
\begin{code}

   public static int[][] readIntData2D (URL url) throws IOException\begin{hide} {
      Reader reader = new InputStreamReader (url.openStream());
      try {
         return readIntData2D (reader);
      }
      finally {
         reader.close();
      }
   }\end{hide}
\end{code}
\begin{tabb}   Connects to the URL referred to by the URL object \texttt{url},
 and calls \method{readDoubleData}{(Reader)} to
 obtain a matrix of integers from
 the resource.
\end{tabb}
\begin{htmlonly}
   \param{url}{the URL object representing the resource to read.}
   \return{the obtained matrix of integers.}
   \exception{IOException}{if an I/O error occurs.}
\end{htmlonly}
\begin{code}

   public static int[][] readIntData2D (File file) throws IOException\begin{hide} {
      FileReader reader = new FileReader (file);
      try {
         return readIntData2D (reader);
      }
      finally {
         reader.close();
      }
   }\end{hide}
\end{code}
\begin{tabb}   This is equivalent to \method{readDoubleData2D}{(File)},
   for reading integers.
\end{tabb}
\begin{htmlonly}
   \param{file}{the file object represented to file to read.}
   \return{the obtained matrix of integer values.}
   \exception{IOException}{if an I/O error occurs.}
\end{htmlonly}
\begin{code}

   public static int[][] readIntData2D (String file) throws IOException\begin{hide} {
      FileReader reader = new FileReader (file);
      try {
         return readIntData2D (reader);
      }
      finally {
         reader.close();
      }
   }\end{hide}
\end{code}
\begin{tabb}   This is equivalent to \method{readDoubleData2D}{(String)},
   for reading integers.
\end{tabb}
\begin{htmlonly}
   \param{file}{the name of the file to read.}
   \return{the obtained matrix of integer values.}
   \exception{IOException}{if an I/O error occurs.}
\end{htmlonly}
\begin{code}

   public static String[][] readCSVData (Reader input, char colDelim,
                                         char stringDelim)
                                         throws IOException\begin{hide} {
      // Using a buffered reader is important here for performance
      // LineNumberReader is a subclass of BufferedReader
      LineNumberReader inb = new LineNumberReader (input);
      StringBuffer sb = new StringBuffer();
      boolean stringMode = false;
      String[][] data = new String[5][];
      int numRows = 0;
      int numColumns = 0;
      boolean newRow = false, newColumn = false;
      int ich = -1;
      char ch = ' ';
      boolean readDone = false;
      while (!readDone) {
         if (ich == -2)
            // A character is pending
            ich = 0;
         else {
            ich = inb.read();
            if (ich == -1)
               // End of stream: process the last column and row, and exit
               newRow = newColumn = readDone = true;
            else
               ch = (char)ich;
         }
         if (ich != -1) {
            if (stringMode) {
               if (ch == stringDelim) {
                  // Check if there is a second string delimiter
                  int ichNext = inb.read();
                  if (ichNext >= 0) {
                     char chNext = (char)ichNext;
                     if (chNext == stringDelim)
                        // Append the quoted string delimiter
                        sb.append (stringDelim);
                     else {
                        // Indicate the end of the string, and a new pending character
                        stringMode = false;
                        ich = -2;
                        ch = chNext;
                     }
                  }
               }
               else
                  sb.append (ch);
            }
            else {
              if (ch == '\n' || ch == '\r') {
                 int ichNext = inb.read();
                 if (ichNext >= 0) {
                    char chNext = (char)ichNext;
                    if (ch == '\r' && chNext == '\n') {
                       ichNext = inb.read();
                       if (ichNext >= 0) {
                          chNext = (char)ichNext;
                          ich = -2;
                          ch = chNext;
                          newRow = true;
                       }
                    }
                    else {
                       ich = -2;
                       ch = chNext;
                       newRow = true;
                    }
                 }
              }
              else if (ch == colDelim)
                 newColumn = true;
              else if (ch == stringDelim)
                 stringMode = true;
              else
                 sb.append (ch);
            }
         }
         if (newColumn || newRow) {
            if (numColumns == 0) {
               ++numRows;
               numColumns = 1;
            }
            else
               ++numColumns;
            if (data.length < numRows) {
               String[][] newData = new String[2*data.length][];
               System.arraycopy (data, 0, newData, 0, data.length);
               data = newData;
            }
            if (data[numRows - 1] == null)
               data[numRows - 1] = new String[5];
            else if (data[numRows - 1].length < numColumns) {
               String[] newData = new String[2*data[numRows - 1].length];
               System.arraycopy (data[numRows - 1], 0, newData, 0, data[numRows - 1].length);
               data[numRows - 1] = newData;
            }
            data[numRows - 1][numColumns - 1] = sb.toString();
            sb.delete (0, sb.length());
            newColumn = false;
         }
         if (newRow) {
            if (data[numRows - 1].length != numColumns) {
               String[] data2 = new String[numColumns];
               System.arraycopy (data[numRows - 1], 0, data2, 0, numColumns);
               data[numRows - 1] = data2;
            }
            numColumns = 0;
            newRow = false;
         }
      }

      if (stringMode)
         throw new IllegalArgumentException ("Too many string delimiters " + stringDelim);
      if (data.length != numRows) {
         String[][] data2 = new String[numRows][];
         System.arraycopy (data, 0, data2, 0, numRows);
         return data2;
      }
      return data;
   }\end{hide}
\end{code}
\begin{tabb}   Reads comma-separated values (CSV) from reader \texttt{input}, and
  returns a 2D array of strings corresponding to the read data.
  Lines are delimited using line separators \texttt{\bs r},
  \texttt{\bs n}, and \texttt{\bs r\bs n}.
  Each line contains one or more values, separated by the column
  delimiter \texttt{colDelim}.
  If a string of characters is surrounded with the string delimiter
  \texttt{stringDelim}, any line separator and column separator appear in
  the string.  The string delimiter can be inserted in such a string by
  putting it twice.
  Usually, the column delimiter is the comma, and the string delimiter is
  the quotation mark. The following example uses these default
  delimiters.
\begin{verbatim}
         "One","Two","Three"
          1,2,3
         "String with "" delimiter",n,m
\end{verbatim}
This produces a matrix of strings with dimensions $3\times 3$.
The first row contains the strings \texttt{One}, \texttt{Two}, and \texttt{Three}
while the second row contains the strings \texttt{1}, \texttt{2}, and \texttt{3}.
The first column of the last row contains the string
\texttt{String with " delimiter}.
\end{tabb}
\begin{htmlonly}
   \param{input}{the reader to obtain data from.}
   \param{colDelim}{the column delimiter.}
   \param{stringDelim}{the string delimiter.}
   \return{the obtained 2D array of strings.}
   \exception{IOException}{if an I/O error occurs.}
\end{htmlonly}
\begin{code}

   public static String[][] readCSVData (URL url, char colDelim,
                                         char stringDelim)
                                         throws IOException\begin{hide} {
      Reader reader = new InputStreamReader (url.openStream());
      try {
         return readCSVData (reader, colDelim, stringDelim);
      }
      finally {
         reader.close();
      }
   }\end{hide}
\end{code}
\begin{tabb}   Connects to the URL referred to by the URL object \texttt{url},
 and calls \method{readCSVData}{(Reader,char,char)} to
 obtain a matrix of strings from
 the resource.
\end{tabb}
\begin{htmlonly}
   \param{url}{the URL object representing the resource to read.}
   \param{colDelim}{the column delimiter.}
   \param{stringDelim}{the string delimiter.}
   \return{the obtained matrix of strings.}
   \exception{IOException}{if an I/O error occurs.}
\end{htmlonly}
\begin{code}

   public static String[][] readCSVData (File file, char colDelim,
                                         char stringDelim)
                                         throws IOException\begin{hide} {
      FileReader reader = new FileReader (file);
      try {
         return readCSVData (reader, colDelim, stringDelim);
      }
      finally {
         reader.close();
      }
   }\end{hide}
\end{code}
\begin{tabb}   This is equivalent to \method{readDoubleData2D}{(File)},
   for reading strings.
\end{tabb}
\begin{htmlonly}
   \param{file}{the file object represented to file to read.}
   \param{colDelim}{the column delimiter.}
   \param{stringDelim}{the string delimiter.}
   \return{the obtained matrix of string values.}
   \exception{IOException}{if an I/O error occurs.}
\end{htmlonly}
\begin{code}

   public static String[][] readCSVData (String file, char colDelim,
                                         char stringDelim)
                                         throws IOException\begin{hide} {
      FileReader reader = new FileReader (file);
      try {
         return readCSVData (reader, colDelim, stringDelim);
      }
      finally {
         reader.close();
      }
   }\end{hide}
\end{code}
\begin{tabb}   This is equivalent to \method{readDoubleData2D}{(String)},
   for reading strings.
\end{tabb}
\begin{htmlonly}
   \param{file}{the name of the file to read.}
   \param{colDelim}{the column delimiter.}
   \param{stringDelim}{the string delimiter.}
   \return{the obtained matrix of string values.}
   \exception{IOException}{if an I/O error occurs.}
\end{htmlonly}
\begin{code}\begin{hide}
}\end{hide}
\end{code}
