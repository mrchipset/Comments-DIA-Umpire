\defmodule {BoxSeriesCollection}

This class stores data used in a 
\externalclass{umontreal.iro.lecuyer.charts}{CategoryChart}.
It also provides complementary tools to draw
 box-and-whisker plots; for example, one may
add or remove plots series and modify plot style. This class is linked with 
the JFreeChart \texttt{DefaultBoxAndWhiskerCategoryDataset} class to store 
data plots, and linked with the JFreeChart
\texttt{BoxAndWhiskerRenderer} to render the plots.

\bigskip\hrule
\begin{code}
\begin{hide}
/*
 * Class:        BoxSeriesCollection
 * Description:  
 * Environment:  Java
 * Software:     SSJ 
 * Copyright (C) 2001  Pierre L'Ecuyer and Universite de Montreal
 * Organization: DIRO, Universite de Montreal
 * @author       
 * @since

 * SSJ is free software: you can redistribute it and/or modify it under
 * the terms of the GNU General Public License (GPL) as published by the
 * Free Software Foundation, either version 3 of the License, or
 * any later version.

 * SSJ is distributed in the hope that it will be useful,
 * but WITHOUT ANY WARRANTY; without even the implied warranty of
 * MERCHANTABILITY or FITNESS FOR A PARTICULAR PURPOSE.  See the
 * GNU General Public License for more details.

 * A copy of the GNU General Public License is available at
   <a href="http://www.gnu.org/licenses">GPL licence site</a>.
 */
\end{hide}
package umontreal.iro.lecuyer.charts;\begin{hide}

import   org.jfree.chart.renderer.category.BoxAndWhiskerRenderer;
import   org.jfree.data.statistics.DefaultBoxAndWhiskerCategoryDataset;

import   java.util.List;
import   java.util.ArrayList;
import   java.util.Locale;
import   java.util.Formatter;
\end{hide}

public class BoxSeriesCollection extends SSJCategorySeriesCollection \begin{hide} {
   final double BARWIDTH = 0.1;
\end{hide}
\end{code}

%%%%%%%%%%%%%%%%%%%%%%%%%%%
\subsubsection*{Constructors}

\begin{code}
   public BoxSeriesCollection () \begin{hide} {
      renderer = new BoxAndWhiskerRenderer();
      seriesCollection = new DefaultBoxAndWhiskerCategoryDataset ();
      ((BoxAndWhiskerRenderer)renderer).setMaximumBarWidth(BARWIDTH);
   }\end{hide}
\end{code}
\begin{tabb}
   Creates a new \texttt{BoxSeriesCollection} instance with an empty dataset.
\end{tabb}
\begin{code}

   public BoxSeriesCollection (double[] data, int numPoints) \begin{hide} {
      renderer = new BoxAndWhiskerRenderer();

      ((BoxAndWhiskerRenderer)renderer).setMaximumBarWidth(BARWIDTH);
      seriesCollection = new DefaultBoxAndWhiskerCategoryDataset ();

      DefaultBoxAndWhiskerCategoryDataset tempSeriesCollection =
            (DefaultBoxAndWhiskerCategoryDataset)seriesCollection;

      final List<Double> list = new ArrayList<Double>();
      for (int i = 0; i < numPoints; i ++)
         list.add(data[i]);

      tempSeriesCollection.add(list, 0, 0);
   }\end{hide}
\end{code}
\begin{tabb}
 Creates a new \texttt{BoxSeriesCollection} instance with default parameters
 and input series \texttt{data}. Only \emph{the first} \texttt{numPoints}
 of \texttt{data} will taken into account.
\end{tabb}
\begin{htmlonly}
   \param{data}{point sets.}
   \param{numPoints}{Number of points}
\end{htmlonly}
\begin{code}

   public BoxSeriesCollection (double[]... data) \begin{hide} {
      renderer = new BoxAndWhiskerRenderer();
      seriesCollection = new DefaultBoxAndWhiskerCategoryDataset ();

      DefaultBoxAndWhiskerCategoryDataset tempSeriesCollection =
            (DefaultBoxAndWhiskerCategoryDataset)seriesCollection;

      for (int i = 0; i < data.length; i ++) {
         if (data[i].length == 0)
            throw new IllegalArgumentException("Unable to render the plot. data["
                                                   + i +"] contains no row");
         final List<Double> list = new ArrayList<Double>();
         for (int j = 0; j < data[i].length-1; j ++)
            list.add(data[i][j]);
         tempSeriesCollection.add(list, 0, "Serie " + i);
         list.clear();
      }
      ((BoxAndWhiskerRenderer)renderer).setMaximumBarWidth(BARWIDTH);
   }\end{hide}
\end{code}
\begin{tabb}
   Creates a new \texttt{BoxSeriesCollection} instance with default
   parameters and given data series. The input parameter represents series
   of point sets.
\end{tabb}
\begin{htmlonly}
   \param{data}{series of point sets.}
\end{htmlonly}
\begin{code}

   public BoxSeriesCollection (DefaultBoxAndWhiskerCategoryDataset data) \begin{hide} {
      renderer = new BoxAndWhiskerRenderer();
      ((BoxAndWhiskerRenderer)renderer).setFillBox(false);
      seriesCollection = data;
      ((BoxAndWhiskerRenderer)renderer).setMaximumBarWidth(BARWIDTH);
   }\end{hide}
\end{code}
\begin{tabb}
   Creates a new \texttt{BoxSeriesCollection} instance with default parameters and given data series.
   The input parameter represents a \texttt{DefaultBoxAndWhiskerCategoryDataset}.
\end{tabb}
\begin{htmlonly}
   \param{data}{series of point sets.}
\end{htmlonly}


%%%%%%%%%%%%%%%%%%%%%%%%%%%
\subsubsection*{Data control methods}

\begin{code}

   public int add (double[] data) \begin{hide} {
      return add(data, data.length);
   }\end{hide}
\end{code}
\begin{tabb}
   Adds a data series into the series collection. Vector \texttt{data} represents
   a point set.
\end{tabb}
\begin{htmlonly}
   \param{data}{point sets.}
   \return{Integer that represent the new point set's position in the JFreeChart \texttt{DefaultBoxAndWhiskerXYDataset} object.}
\end{htmlonly}
\begin{code}

   public int add (double[] data, int numPoints) \begin{hide} {
      DefaultBoxAndWhiskerCategoryDataset tempSeriesCollection =
            (DefaultBoxAndWhiskerCategoryDataset)seriesCollection;

      final List<Double> list = new ArrayList<Double>();
      for (int i = 0; i < numPoints; i ++)
         list.add(data[i]);

      int count = tempSeriesCollection.getColumnCount();
      tempSeriesCollection.add(list, 0, "Serie " + count);
      return count;
   }\end{hide}
\end{code}
\begin{tabb}
   Adds a data series into the series collection. Vector \texttt{data} represents
   a point set. Only \emph{the first} \texttt{numPoints} of 
   \texttt{data} will be added to the new series.
\end{tabb}
\begin{htmlonly}
   \param{data}{Point set}
   \param{numPoints}{Number of points to add}
   \return{Integer that represent the new point set's position in the JFreeChart \texttt{DefaultBoxAndWhiskerXYDataset} object.}
\end{htmlonly}
\begin{code}

   public String getName (int series) \begin{hide} {
      return (String)((DefaultBoxAndWhiskerCategoryDataset)seriesCollection).getColumnKey(series);
   }\end{hide}
\end{code}
\begin{tabb}
   Gets the current name of the selected series.
\end{tabb}
\begin{htmlonly}
   \param{series}{series index.}
   \return{current name of the series.}
\end{htmlonly}
\begin{code}

   public double[] getRangeBounds() \begin{hide} {
      double max=0, min=0;
      DefaultBoxAndWhiskerCategoryDataset tempSeriesCollection =
            (DefaultBoxAndWhiskerCategoryDataset)seriesCollection;

      if(tempSeriesCollection.getColumnCount() != 0 && tempSeriesCollection.getRowCount() != 0) {
         max = tempSeriesCollection.getItem(0, 0).getMaxOutlier().doubleValue() ;
         min = tempSeriesCollection.getItem(0, 0).getMinOutlier().doubleValue() ;
      }

      for(int i = 0; i < tempSeriesCollection.getRowCount(); i++) {
         for( int j = 0; j < tempSeriesCollection.getColumnCount(); j++) {
            max = Math.max(max, tempSeriesCollection.getItem(i, j).getMaxOutlier().doubleValue() );
            min = Math.min(min, tempSeriesCollection.getItem(i, j).getMinOutlier().doubleValue() );
         }
      }

      double[] retour = {min, max};
      return retour;
   }\end{hide}
\end{code}
\begin{tabb}
   Returns the range ($y$-coordinates) min and max values.
\end{tabb}
\begin{htmlonly}
   \return{range min and max values.}
\end{htmlonly}
\begin{code}

   public String toString() \begin{hide} {
      Formatter formatter = new Formatter(Locale.US);
      for(int i = 0; i < seriesCollection.getRowCount(); i++) {
         formatter.format(" Series " + i + " : %n");
         for(int j = 0; j < seriesCollection.getColumnCount(); j++)
            formatter.format(",%15e%n", seriesCollection.getValue(i, j));
      }
      return formatter.toString();
   }\end{hide}
\end{code}
\begin{tabb}
  Returns in a \texttt{String} all data contained in the current object.
\end{tabb}
\begin{htmlonly}
   \return{All data contained in the current object as a \class{String}.}
\end{htmlonly}

%%%%%%%%%%%%%%%%%%%%%%%%%%%
% \subsubsection*{Rendering methods}
\begin{code}

   public String toLatex (double YScale, double YShift, 
                          double ymin, double ymax) \begin{hide} {
      throw new UnsupportedOperationException(" NOT implemented yet");
/*         
      // Calcule les bornes reelles du graphique, en prenant en compte la position des axes
      
      ymin = Math.min(YShift, ymin);
      ymax = Math.max(YShift, ymax);

      DefaultBoxAndWhiskerCategoryDataset tempSeriesCollection = (DefaultBoxAndWhiskerCategoryDataset)seriesCollection;
      Formatter formatter = new Formatter(Locale.US);
//       double var;
//       double margin = ((BoxAndWhiskerRenderer)renderer).getMargin();

//       for (int i = tempSeriesCollection.getColumnCount() - 1; i >= 0; i--) {
//          List temp = tempSeriesCollection.getBins(i);
//          ListIterator iter = temp.listIterator();
//          
//          Color color = (Color)renderer.getSeriesPaint(i);
//          String colorString = detectXColorClassic(color);
//          if (colorString == null) {
//             colorString = "color"+i;
//             formatter.format( "\\definecolor{%s}{rgb}{%.2f, %.2f, %.2f}%n",
//                               colorString, color.getRed()/255.0, color.getGreen()/255.0, color.getBlue()/255.0);
//          }
//          
//          HistogramBin currentBin=null;
//          while(iter.hasNext()) {
//             double currentMargin;
//             currentBin = (HistogramBin)iter.next();
//             currentMargin = ((margin*(currentBin.getEndBoundary()-currentBin.getStartBoundary()))-XShift)*XScale;
//             if ((currentBin.getStartBoundary() >= xmin && currentBin.getStartBoundary() <= xmax) 
//                && (currentBin.getCount() >= ymin && currentBin.getCount() <= ymax) )
//             {
//                var = Math.min( currentBin.getEndBoundary(), xmax);
//                if (filled[i]) {
//                   formatter.format("\\filldraw [line width=%.2fpt, opacity=%.2f, color=%s] ([xshift=%.4f] %.4f, %.4f) rectangle ([xshift=-%.4f] %.4f, %.4f); %%%n",
//                         lineWidth[i], (color.getAlpha()/255.0), colorString,
//                         currentMargin, (currentBin.getStartBoundary()-XShift)*XScale, 0.0,
//                         currentMargin, (var-XShift)*XScale, (currentBin.getCount()-YShift)*YScale);
//               }
//               else {
//                   formatter.format("\\draw [line width=%.2fpt, color=%s] ([xshift=%.4f] %.4f, %.4f) rectangle ([xshift=-%.4f] %.4f, %.4f); %%%n",
//                         lineWidth[i], colorString,
//                         currentMargin, (currentBin.getStartBoundary()-XShift)*XScale, 0.0,
//                         currentMargin, (var-XShift)*XScale, (currentBin.getCount()-YShift)*YScale);
//               }
//             }
//             else if (   (currentBin.getStartBoundary() >= xmin && currentBin.getStartBoundary() <= xmax) 
//                         && (currentBin.getCount() >= ymin && currentBin.getCount() > ymax) )
//             { // Cas ou notre rectangle ne peut pas etre affiche en entier (trop haut)
//                var = Math.min( currentBin.getEndBoundary(), xmax);
//                if (filled[i]) {
//                   formatter.format("\\filldraw [line width=%.2fpt,  opacity=%.2f, color=%s] ([xshift=%.4f] %.4f, %.4f) rectangle ([xshift=-%.4f] %.4f, %.4f); %%%n",
//                         lineWidth[i], (color.getAlpha()/255.0), colorString,
//                         currentMargin, (currentBin.getStartBoundary()-XShift)*XScale, 0.0,
//                         currentMargin, (var-XShift)*XScale, (ymax-YShift)*YScale);
//               formatter.format("\\draw [line width=%.2fpt, color=%s, style=dotted] ([xshift=%.4f] %.4f, %.4f) rectangle ([yshift=3mm, xshift=-%.4f] %.4f, %.4f); %%%n",
//                         lineWidth[i], colorString,
//                         currentMargin, (currentBin.getStartBoundary()-XShift)*XScale, (ymax-YShift)*YScale,
//                         currentMargin, (var-XShift)*XScale, (ymax-YShift)*YScale);
//                }
//                else {
//                   formatter.format("\\draw [line width=%.2fpt, color=%s] ([xshift=%.4f] %.4f, %.4f) rectangle ([xshift=-%.4f] %.4f, %.4f); %%%n",
//                         lineWidth[i], colorString,
//                         currentMargin, (currentBin.getStartBoundary()-XShift)*XScale, 0.0,
//                         currentMargin, (var-XShift)*XScale, (ymax-YShift)*YScale);
//                   
//               formatter.format("\\draw [line width=%.2fpt, color=%s, style=dotted] ([xshift=%.4f] %.4f, %.4f) rectangle ([yshift=3mm, xshift=-%.4f] %.4f, %.4f); %%%n",
//                         lineWidth[i], colorString,
//                         currentMargin, (currentBin.getStartBoundary()-XShift)*XScale, (ymax-YShift)*YScale,
//                         currentMargin, (var-XShift)*XScale, (ymax-YShift)*YScale);
//                }
//             }
//          }
//       }
      return formatter.toString();
*/
}\end{hide}
\end{code}
\begin{tabb}
  NOT IMPLEMENTED: To do.
\end{tabb}
\begin{htmlonly}
   \param{ymin}{}
   \param{ymax}{}
   \return{LaTeX source code}
\end{htmlonly}
\begin{code}
\begin{hide}
}\end{hide}
\end{code}
